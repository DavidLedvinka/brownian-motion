% please run using
% lualatex --shell-escape ...tex

\documentclass{article}
\usepackage{latexsym, a4wide}
\usepackage[utf8]{inputenc}
\usepackage[T1]{fontenc}

\usepackage{color}
\usepackage{amsfonts,amssymb,mathtools,amsthm,mathrsfs,bbm}
\newtheorem{theorem}{Theorem}
\newtheorem{proposition}{Proposition}[section]
\newtheorem{lemma}[proposition]{Lemma}
\newtheorem{corollary}[proposition]{Corollary}
\newtheorem{definition}[proposition]{Definition}
\theoremstyle{definition}
\newtheorem{remark}[proposition]{Remark}
\newtheorem{example}[proposition]{Example}
\newtheoremstyle{step}{3pt}{0pt}{}{}{\bf}{}{.5em}{}
\theoremstyle{step} \newtheorem{step}{Step}

\DeclareMathAlphabet{\mathpzc}{OT1}{pzc}{m}{it}
\newcommand{\smallu}{\mathpzc{u}}

\usepackage{minted}
\setminted{encoding=utf-8}
\usepackage{fontspec}
%\setmainfont{dejavu-serif}
\setmonofont[Scale=0.85]{dejavu-sans}
%\setmainfont{FreeSerif}
%\setmonofont{FreeMono}
\usepackage{xcolor, graphics}
\definecolor{mygray}{rgb}{0.97,0.97,0.97}
\definecolor{LightCyan}{rgb}{0.8784,1,1}
\newcommand{\leanline}[1]{\mintinline[breaklines, breakafter=_., breaklines]{Lean}{#1}}%\textsuperscript{\textcolor{gray}{\tiny(m)}}}
\newcommand{\leanlinecolor}{\mintinline[breaklines, breakafter=_., breaklines, bgcolor=LightCyan]{Lean}}
%\usepackage{refcheck}

\usepackage[notcite,notref]{showkeys}
\usepackage{hyperref}

\begin{document}
\title{\LARGE Formalizing Brownian motion}

\maketitle Our goal is to write down the steps necessary in order to
formalize Brownian motions (or $\mathbb R^d$-valued Gaussian
processes) in some generality using {\tt mathlib}.

\begin{remark}[Notation]
  We will write $(E,r)$ for some extended pseudo-metric space, $\mathcal
    P(E)$ for the set of probability measures on the Borel
  $\sigma$-algebra on $E$, $\mathbbm k \in \{\mathbb R, \mathbb C\}$,
  and $\mathcal C_{b}(E, \mathbbm k)$ the set of $\mathbbm k$-valued
  bounded continuous functions on $E$. For some $\mathbf P \in \mathcal
    P(E)$ and $f \in \mathcal C_{b}(E, \mathbbm k)$, we let $\mathbf P[f]
    := \int f(x) \mathbf P(dx) \in \mathbbm k$ be the expectation.
\end{remark}

\setcounter{section}{-1}
\section{Some simple probability results}
The following is a simple consequence of dominated convergence, and
often needed in probability theory.

\begin{definition}
  Let $E$ be some set and $f, f_1, f_2,...: E \to \mathbbm k$. We say
  that $f_1,f_2,..:$ converges to $f$ boundedly pointwise if $f_n
    \xrightarrow{n\to\infty} f$ pointwise and $\sup_n ||f_n|| <
    \infty$. We write $f_n \xrightarrow{n\to\infty}_{bp} f$
\end{definition}

\begin{lemma}\label{lemma:bp}
  Let $(\Omega, \mathcal A, \mathbf P)$ be a probability (or finite)
  measure space, and $X, X_1, X_2,... : \Omega \to \mathbbm k$ such
  that $X_n \xrightarrow{n\to\infty}_{bp} X$. Then, $\mathbf E[X_n]
    \xrightarrow{n\to\infty} \mathbf E[X]$.
\end{lemma}

\begin{proof}
  Note that the constant function $x \mapsto \sup_n ||f_n||$ is
  integrable (since $\mathbf P$ is finite), so the result follows from
  dominated convergence.
\end{proof}

\begin{definition}
  Let $X,X_1,X_2,...$, all $E$-valued random variables.
  \begin{enumerate}
    \item We say that $X_n \xrightarrow{n\to\infty} X$ almost everywhere
          if $\mathbf P(\lim_{n\to\infty} X_n = X) = 1$. We also write
          $X_n\xrightarrow{n\to\infty}_{ae} X$.
    \item We say that $X_n \xrightarrow{n\to\infty} X$ in probability
          (or in measure) if, for all $\varepsilon>0$,
          $$ \lim_{n\to\infty} \mathbf P(r(X_n, X) > \varepsilon) = 0.$$
  \end{enumerate}
\end{definition}
The two notions here are denoted \leanline{∀ᵐ (x : α) ∂P,
  Filter.Tendsto (fun n => X n x) Filter.atTop (nhds (X x))} and
\leanline{MeasureTheory.TendstoInMeasure}, respectively.

\begin{lemma}\label{l:aep}
  Let $X,X_1,X_2,...$ be $E$-valued random variables with $X_n
    \xrightarrow{n\to\infty}_{ae} X$. Then, $X_n
    \xrightarrow{n\to\infty}_{p} X$.
\end{lemma}

This result is called
\leanline{MeasureTheory.tendstoInMeasure_of_tendsto_ae} in
\leanline{mathlib}. We also need the (almost sure) uniquess of the
limit in measure, which is not formalized in \leanline{mathlib} yet:

\begin{lemma}[Uniqueness of a limit in probability]\label{l:puni}
  Let $X,Y,X_1,X_2,...$ be $E$-valued random variables with $X_n
    \xrightarrow{n\to\infty}_{p} X$ and $X_n
    \xrightarrow{n\to\infty}_{p} Y$. Then, $X=Y$, almost surely.
\end{lemma}

\begin{proof}
  We write, using monotone convergence and Lemma~\ref{l:aep}
  \begin{align*}
    \mathbf P(X\neq Y) & = \lim_{\varepsilon \downarrow 0} \mathbf
    P(r(X,Y)>\varepsilon) \leq \lim_{\varepsilon \downarrow 0}
    \lim_{n\to\infty}\mathbf P(r(X,X_n)>\varepsilon/2) + \mathbf
    P(r(Y,X_n)>\varepsilon/2) = 0.
  \end{align*}
\end{proof}

\begin{lemma}\label{l:supsum}
  Let $I$ be some (finite or infinite) set and $(X_t)_{t\in I}$ be a
  family of random variables with values in $[0,\infty)$. Then,
  $\sup_{t\in I} X_t \leq \sum_{t\in I} X_t.$
\end{lemma}

\section{Separating algebras and characteristic functions}

\begin{definition}[Separating class of functions]
  \mbox{} Let $\mathcal M \subseteq \mathcal C_{b}(E,\mathbbm k)$.
  \begin{enumerate}
    \item If, for all $x,y\in E$ with $x\neq y$, there is
          $f\in\mathcal M$ with $f(x)\neq f(y)$, we say that $\mathcal M$
          separates points.
    \item
          If, for all $\mathbf P, \mathbf Q\in\mathcal P(E)$,
          $$ \mathbf P = \mathbf Q \quad \text{iff}\quad \mathbf P[f] =
            \mathbf Q[f] \text{ for all } f\in\mathcal M,$$ we say that
          $\mathcal M$ is separating in $\mathcal P(E)$.
    \item If (i) $1\in\mathcal M$ and (ii) if $\mathcal M$ is closed
          under sums and products, then we call $\mathcal M$ a
          (sub-)algebra.  If $\mathbbm k = \mathbb C$, and (iii) if
          $\mathcal M$ is closed under complex conjugation, we call
          $\mathcal M$ a star-(sub-)algebra.
  \end{enumerate}
\end{definition}

In \leanline{mathlib}, 1.\ and 3.\ of the above definition are already
implemented:

\begin{minted}[highlightlines={}, mathescape, numbersep=5pt, framesep=5mm, bgcolor=mygray]{Lean}
  structure Subalgebra (R : Type u) (A : Type v) [CommSemiring R] [Semiring A]
    [Algebra R A] extends Subsemiring :
  Type v

  abbrev Subalgebra.SeparatesPoints {α : Type u_1} [TopologicalSpace α]
    {R : Type u_2} [CommSemiring R] {A : Type u_3} [TopologicalSpace A]
    [Semiring A] [Algebra R A] [TopologicalSemiring A] (s : Subalgebra R C(α, A))
    : Prop
\end{minted}

The latter is an extension of \leanline{Set.SeparatesPoints}, which
works on any set of functions.

\noindent
For the first result, we already need that $(E,r)$ has a metric
structure. There is a formalization of this result in
\url{https://github.com/pfaffelh/some_probability/tree/master}.

\begin{lemma}\label{l:unique}
  $\mathcal M := \mathcal C_b(E, \mathbbm k)$ is separating.
\end{lemma}

\begin{proof}
  We restrict ourselves to $\mathbbm k = \mathbb R$, since the result
  for $\mathbbm k = \mathbb C$ follows by only using functions with
  vanishing imaginary part. Let $\mathbf P, \mathbf Q \in \mathcal
    P(E)$. We will prove that $\mathbf P(A) = \mathbf Q(A)$ for all $A$
  closed. Since the set of closed sets is a $\pi$-system generating
  the Borel-$\sigma$-algebra, this suffices for $\mathbf P = \mathbf
    Q$. So, let $A$ be closed and $g = 1_A$ be the indicator
  function. Let $g_n(x) := (1 - n r(A,x))^+$ (where $r(A,y) := \inf_{y\in
      A}r(y,x)$) and note that $g_n(x) \xrightarrow{n\to\infty}
    1_A(x)$. Then, we have by dominated convergence
  \begin{align*}
    \mathbf P(A) & = \lim_{n\to\infty} \mathbf P[g_n] =
    \lim_{n\to\infty} \mathbf Q[g_n] = \mathbf Q(A),
  \end{align*}
  and we are done.
\end{proof}

We will use the Stone-Weierstrass Theorem below. Here is its version
in \leanline{mathlib}. Note that this requires $E$ to be compact.

\begin{minted}[highlightlines={0}, mathescape, numbersep=5pt, framesep=5mm, bgcolor=mygray]{Lean}
  theorem ContinuousMap.starSubalgebra_topologicalClosure_eq_top_of_separatesPoints
  {𝕜 : Type u_2} {X : Type u_1} [IsROrC 𝕜] [TopologicalSpace X] [CompactSpace X]
  (A : StarSubalgebra 𝕜 C(X, 𝕜)) (hA : Subalgebra.SeparatesPoints A.toSubalgebra) :
  StarSubalgebra.topologicalClosure A = ⊤
\end{minted}

We also need (as proved in the last project):

\begin{minted}[highlightlines={0}, mathescape, numbersep=5pt, framesep=5mm, bgcolor=mygray]{Lean}
  theorem innerRegular_isCompact_isClosed_measurableSet_of_complete_countable
  [PseudoEMetricSpace α] [CompleteSpace α] [SecondCountableTopology α] [BorelSpace α]
  (P : Measure α) [IsFiniteMeasure P] :
  P.InnerRegular (fun s => IsCompact s ∧ IsClosed s) MeasurableSet
\end{minted}

The proof of the following result follows
\cite[Theorem~3.4.5]{EthierKurtz1986}.

\begin{theorem}[Algebras separating points and separating algebras]
  \label{T:wc3}\mbox{}\\ \sloppy Let $(E,r)$ be a complete and separable extended pseudo-metric space, and
  $\mathcal M \subseteq\mathcal C_b(E,\mathbbm k)$ be a
  star-sub-algebra that separates points. Then, $\mathcal M$ is
  separating.
\end{theorem}

\begin{proof}
  Let $\mathbf P, \mathbf Q\in\mathcal P(E)$, $\varepsilon>0$ and $K$
  compact, such that $\mathbf P(K)>1-\varepsilon$, $\mathbf
    Q(K)>1-\varepsilon$, and $g \in \mathcal C_b(E,\mathbbm k)$. According to the
  Stone-Weierstrass Theorem, there is $(g_n)_{n=1,2,\dots}$ in
  $\mathcal M$ with
  \begin{align}
    \label{eq:wc9}
    \sup_{x\in K} |g_n(x) - g(x)| \xrightarrow{n\to\infty} 0.
  \end{align}
  So, (note that $C := \sup_{x\geq 0} xe^{-x^2} < \infty$)
  \begin{align*}
    \big|\mathbf P[ge^{-\varepsilon g^2}] - \mathbf Q[ge^{-\varepsilon
    g^2}] \big| & \leq \big|\mathbf P[ge^{-\varepsilon g^2}] -
    \mathbf P[ge^{-\varepsilon g^2};K] \big|                   \\ & \quad {} + \big|\mathbf
    P[ge^{-\varepsilon g^2};K] - \mathbf P[g_ne^{-\varepsilon
    g_n^2};K] \big|                                            \\ & \quad {} + \big| \mathbf P[g_ne^{-\varepsilon
    g_n^2};K] - \mathbf P[g_ne^{-\varepsilon g_n^2}] \big|     \\ &
    \quad {} + |\mathbf P[g_ne^{-\varepsilon g_n^2}] - \mathbf
    Q[g_ne^{-\varepsilon g_n^2}] \big|                         \\ & \quad {} + \big|\mathbf
    Q[g_ne^{-\varepsilon g_n^2}] - \mathbf Q[g_ne^{-\varepsilon
    g_n^2};K] \big|                                            \\ & \quad {}  + \big|\mathbf Q[g_ne^{-\varepsilon
    g_n^2}] - \mathbf Q[ge^{-\varepsilon g^2};K] \big|         \\ & \quad{} {}  +
    \big|\mathbf Q[ge^{-\varepsilon g^2};K] - \mathbf
    Q[ge^{-\varepsilon g^2}] \big|
  \end{align*}
  We bound the first term by
  $$\big|\mathbf P[ge^{-\varepsilon g^2}] - \mathbf P[ge^{-\varepsilon
            g^2};K] \big| \leq \frac{C}{\sqrt{\varepsilon}} \mathbf P(K^c)
    \leq C\sqrt{\varepsilon},$$ and analogously for the third, fifth and
  last. The second and second to last vanish for $n\to\infty$ due to
  \eqref{eq:wc9}. Since $\mathcal M$ is an algebra, we can
  approximate, using dominated convergence,
  \begin{align*}
    \mathbf P[g_ne^{-\varepsilon g_n^2}] = \lim_{m \to \infty} \mathbf
    P[\underbrace{g_n \Big(1 - \frac{\varepsilon
            g_n^2}{m}\Big)^m}_{\in\mathcal M}\Big] = \lim_{m \to \infty}
    \mathbf Q[\underbrace{g_n \Big(1 - \frac{\varepsilon
            g_n^2}{m}\Big)^m}_{\in\mathcal M}\Big] = \mathbf
    Q[g_ne^{-\varepsilon g_n^2}],
  \end{align*}
  so the fourth term vanishes for $n\to\infty$ as well. Concluding,
  \begin{align*}
    \big|\mathbf P[g] - \mathbf Q[g] \big| = \lim_{\varepsilon\to 0}
    \big|\mathbf P[ge^{-\varepsilon g^2}] - \mathbf Q[ge^{-\varepsilon
            g^2}] \big| \leq 4C \lim_{\varepsilon \to 0}\sqrt{\varepsilon} =
    0.
  \end{align*}
  Since $g$ was arbitrary and $\mathcal C_b(E,\mathbbm k)$ is
  separating by Lemma~\ref{l:unique}, we find $\mathbf P = \mathbf Q$.
\end{proof}

\noindent
We now come to characteristic functions and Laplace transforms.

\begin{proposition}[Charakteristic functions determine distributions uniquely]
  \label{Pr:char1}\mbox{}\\
  A probability measure $\mathbf P \in\mathcal P(\mathbb R^d)$ is
  uniquely given by its characteristic function.  \\ In other words,
  if $\mathbf P, \mathbf Q \in\mathcal P(\mathbb R^d)$ are such that
  $\int e^{itx} \mathbf P(dx) = \int e^{itx} \mathbf Q(dx)$ for all
  $t\in\mathbb R^d$. Then, $\mathbf P = \mathbf Q$.
\end{proposition}

\begin{proof}
  The set
  $$\mathcal M:= \Big\{ x\mapsto \sum_{k=1}^n a_k e^{i t_k x}; n \in
    \mathbb N, a_1,...,a_n \in \mathcal C, t_1,...,1_n\in\mathbb
    R^d\Big\}$$ separates points in $\mathbb R^d$. Since $\mathcal M
    \subseteq \mathcal C_b(\mathbb R^d,\mathbbm k)$ contains~1, is
  closed under sums and products, and closed under complex
  conjugation, it is a star-subalgebra of $\mathcal C_b(E,\mathbb
    C)$. So, the assertion directly follows from Theorem~\ref{T:wc3}.
\end{proof}

\begin{remark}\label{rem:proj}
  We also need to show the following: For $J \subseteq I$, where $I$
  is finite, let $\psi$ be the characteristic function for some
  distribution on $\mathbb R^I$. Then, for the projection $\pi_J$, the
  characteristic function of the image measure under $\pi_J$ is given
  by $\psi \circ g_J$, where $(g_J(t)_j) = t_j$ for $j\in J$ and
  $(g(t)_j) = 0$ otherwise. In other words, when computing the
  characteristic function of a projection, just set the coordinates in
  $t \mapsto \psi(t)$ which need to be projected out to $0$.
\end{remark}

\section{Gaussian random variables}
Define an arbitrary family of Gaussian rvs with values in $\mathbb
  R^d$ by (i) defining a standard normal distribution on $\mathbb R$
with the correct density, (ii) show that its characteristic function
is given by $\psi(t) = e^{-t^2/2}$, (iii) define an independent finite
family of standard normal Gaussians using finite product measures and
(iv) define a general independent family by taking some symmetric,
positive definite $C \in \mathbb R^{d\times d}$, some\footnote{In
  order to see that such an $A$ exists, consider some orthogonal $O$
  and a diagonal matrix $D$ with $C = O^\top DO$ and set $A :=
    \sqrt{D} O$, where $\sqrt{D}$ is the diagonal matrix with entries
  $\sqrt{\lambda_i}$ for all eigenvalues $\lambda_i$ of $C$. Then,
  $A\top A = O^\top \sqrt{D}\sqrt{D} O = O^\top D O = C$.} $A \in
  \mathbb R^{d\times d}$ with $C = A^\top A$, and define the Gaussian
measure as the image measure of the independent family $Y$ under the
map $X = A Y + \mu$. Show that
\begin{align*}
  \mathbb E[e^{itX}] & = \mathbb E[e^{it(\mu + AY)}] = e^{it\mu}
  \mathbb E[e^{itAY}] = e^{it\mu} \mathbb E\Big[\exp\Big(i \sum_{kl}
    t_k A_{kl} Y_l\Big)\Big]                                     \\ & = e^{it\mu} \prod_{l} \mathbb
  E\Big[\exp\Big(i \big(\sum_{k} t_k A_{kl}\big) Y_l\Big)\Big] =
  e^{it\mu} \prod_{l} \mathbb E[e^{i (tA_{.l}) Y_l}]             \\ & =
  e^{it\mu} \prod_{l} e^{-(tA_{.l})^2/2} = e^{it\mu} e^{-\sum_l
      (tA_{.l})(A^\top_{l.}t^\top)/2} = e^{it\mu -tCt^\top/2}.
\end{align*}
In particular, this shows that the distribution does not depend on the
choice of $A$ as long as $A^\top A = C$. Together with
Proposition~\ref{Pr:char1}, this shows that there is a unique
probability measure on $\mathbb R^d$ with characteristic function
$t\mapsto e^{it\mu -tCt^\top/2}$ for any vector $\mu$ and symmetric
and positive definite matrix $C$.

In the concrete application of the finite dimensional distribution of
Brownian Motion, consider $0\leq t_1 \leq \cdots \leq t_n$ and $C =
(t_i \wedge t_j)_{1\leq i,j\leq n}$. In order to show that $C$ is
positive semi-definite, there are two paths:

\begin{enumerate}
\item Find $A$ with $A^\top A = C$: In fact, this $A$ can be given
  explicitely by
  $$ A = \left( \begin{matrix}\sqrt{t_1} & \sqrt{t_1} & \sqrt{t_1} &
    \cdots \\ 0 & \sqrt{t_2 - t_1} & \sqrt{t_2 - t_1} & \cdots \\ 0 &
    0 & \sqrt{t_3-t_2}\\ \cdots & \cdots & \cdots &
    \ddots \end{matrix}\right),$$ such that
  $$ A^\top A = \left( \begin{matrix} t_1 & t_1 & t_1 & \cdots \\ t_1
    & t_2 & t_2 & \cdots \\ t_1 & t_2 & t_3\\ \cdots & \cdots & \cdots
    & \ddots\end{matrix}\right).$$
\item Use induction: Apparently (and this is implemented in
  \leanline{mathlib}), if $X$ and $Y$ are positively semidefinite,
  then $X+Y$ is positively semidefinite. We write
  \begin{align*}
    C_3 := \left( \begin{matrix} t_1 & t_1 & t_1 \\ t_1 & t_2 & t_2 \\ t_1
      & t_2 & t_3\end{matrix}\right) = \left( \begin{matrix} t_1 & t_1 &
        t_1 \\ t_1 & t_2 & t_2 \\ t_1 & t_2 & t_2\end{matrix}\right) +
        \left( \begin{matrix} 0 & 0 & 0 \\ 0 & 0 & 0 \\ 0 & 0 & t_3 -
          t_2\end{matrix}\right) =: X_3 + Y_3,
  \end{align*}
  where $Y_3$ is obviously positively semidefinite. Now, for $x =
  (x_1, x_2, x_3)$, we find with $y := (x_1, x_2 + x_3)$, summing over
  all entries of the matrix
  \begin{align*}
    x X_3 x^\top & = \sum \begin{matrix} x_1 t_1 x_1 & x_1 t_1x_2 &
      x_1 t_1 x_3 \\ x_2 t_1 x_1 & x_2 t_2 x_2 & x_2 t_2x_3 \\ x_3 t_1
      x_1 & x_3 t_2 x_2& x_3 t_2x_3\end{matrix} \\ & =
      \sum \begin{matrix} x_1 t_1 x_1 & x_1 t_1x_2 & x_1 t_1 x_3
        \\ (x_2 + x_3) t_1 x_1 & (x_2 + x_3) t_2 x_2 & (x_2 +
        x_3)t_2x_3 \end{matrix} \\ & = \sum \begin{matrix} x_1 t_1 x_1
        & x_1 t_1(x_2 + x_3) \\ (x_2 + x_3) t_1 x_1 & (x_2 + x_3) t_2
        (x_2 + x_3) \end{matrix} = y C_2 y^\top.
  \end{align*}
  Therefore, using $y C_2 y^\top \geq 0$ as the induction hypothesis,
  this shows that $X_3$ is positively semidefinite, and the same
  applies to $C_3$.
\end{enumerate}


\section{Projectivity}\label{S:proj}
For projectivity of finite-dimensional distributions of the BM,
proceed as follows: (i) For $I = \{s_1,...,s_n\} \subseteq \mathbb
R^d$ (with $s_1 < ... < s_d$), define $P_J$ as the unique probability
measure with characteristic function $\psi_I(t) = e^{-tC_It^\top/2}$
with $C_{ij} = s_i \wedge s_j$. For $J\subseteq I$, we then have that
the characteristic function of the projection to coordinates in $J$ is
(see Remark~\ref{rem:proj}) $\psi_I \circ g_J = e^{- g_J(.)  C_I
  g_J(.)^\top} = e^{- . C_J -/2} = \psi_J$. In other words, this is
the required projectivity of $(P_I)_{I \subseteq_f [0,\infty)}$.

\section{The Kolmogorov-Chentsov criterion}
In this section, let $(D, r_D), (E, r_E)$ be extended pseudo-metric
spaces. In addition, we will only have a single probability measure in
this section, so we write $\mathbf P(.)$ for probabilities and
$\mathbf E[.]$ for its expectations.

\begin{definition}[Local Hölder]
  Let $f: D\to E$ and $s\in D$. If there is $\tau>0$ and some $C <
    \infty$ with $r_E(f(s), f(t)) \leq C r_D(s,t)^\gamma$ for all $t$
  with $r_D(s,t) < \tau$, we call $f$ \emph{locally Hö\"lder of order
    $\gamma$} at $s$.
\end{definition}

Hölder is implemented as \leanline{HolderOnWith} (on a set) and
\leanline{HolderWith}. Moreover, locally Hölder at a point is used for
$\gamma=1$ (i.e.\ Lipschitz continuity) e.g.\ in
\leanline{continuousAt_of_locally_lipschitz} (Every function, which is
locally Lipschitz at a point, is continuous.)

\begin{lemma}\label{l:holderext}
  Let $D, E$ be extended pseudo-metric spaces and $f: D\to E$ and
  $s\in D$.
  \begin{enumerate}
    \item If $f$ is locally Hölder at $x$, it is continuous at $x$.
    \item If $E$ is complete, $A \subseteq D$ is dense, and $g : A \to
            E$ is Hölder, it can be extended to a Hölder-$\gamma$-function
          on $D$.
  \end{enumerate}
\end{lemma}

\begin{proof}
  1. Since $f$ is locally Hölder at $s$, choose $\tau>0$ and
  $C<\infty$ such that $r_E(f(s), f(t)) \leq Cr(s,t)^\gamma$ for all
  $t$ with $r_D(s,t) < \tau$. For $\varepsilon>0$, there is
  $\delta'>0$ such $r_D(s,t)^\gamma < \varepsilon / C$ for all $t \in
    B_{\delta'}(s)$. Choose $\delta := \tau \wedge \delta'$ in order to
  see, for $t \in B_\delta(s)$
  $$ r_E(f(s), f(t)) \leq C r(s,t)^\gamma < \varepsilon.$$ 2. For $s
    \in D$, choose $s_1,s_2,...\in A$ with $s_n \xrightarrow{n\to\infty}
    s$. Then, note that $r_E(f(s_n), f(s_M)) \leq C r_D(s_n, s_m)
    \xrightarrow{m,n\to\infty} 0$, so $(f(s_n))_{n=1,2,...}$ is a
  Cauchy-sequence in $E$. We define $f(s)$ to be its limit. Then, for
  $s,t\in D$ and the sequences $s_1, s_2,...\in D, t_1, t_2,...\in D$
  with $s_n \xrightarrow{n\to\infty}s, t_n \xrightarrow{n\to\infty}t$,
  $$ r_E(f(s), f(t)) = \lim_{n\to\infty}r_E(f(s_n), f(t_n)) \leq
    \lim_{n\to\infty} Cr_D(s_n, t_n) = Cr_D(s,t).$$
\end{proof}

For 1., \leanline{continuousAt_of_locally_lipschitz} must be adapted
for Hölder instead of Lipschitz, i.e.\ for $\gamma<1$.

For 2., there is \leanline{LipschitzOnWith.extend_real}, which does
not require the set $A$ to be dense, but $\gamma=1$ and $E=\mathbb R$.
Also, there is \leanline{DenseInducing.continuous_extend} which gives
a condition when a function can continuously be extended. (It needs a
\leanline{DenseInducing} function, which in our case is $i : A \to D,
  x\mapsto x$.)


\begin{lemma}\label{l:gauss}
  For $x\in\mathbb R$,let
  $$ \lfloor x \rfloor := \max\{n \in \mathbb N: n\leq x\}.$$ The
  following holds:
  \begin{enumerate}
    \item $0\leq x - \lfloor x \rfloor < 1$;
    \item   If $|x-y| \leq 1$, then $|\lfloor x\rfloor - \lfloor y \rfloor| \leq
            1$.
    \item $|2 \lfloor x\rfloor - \lfloor 2x \rfloor| \leq 1$.
  \end{enumerate}
\end{lemma}

\begin{proof}
  1. The first inequality is clear that $\lfloor x\rfloor$ is defined
  as a maximum over a set of numbers bounded above by $x$. The second
  inequality holds since otherwise we would have $\lfloor x\rfloor + 1
    \leq x$, in contradiction to the definition of $\lfloor
    x\rfloor$. \\ 2. Without loss of generality, assume that $y\leq x$
  (which implies that $\lfloor y \rfloor \leq \lfloor x\rfloor$). The
  proof is by contradition, so assume that $\lfloor x\rfloor - \lfloor
    y \rfloor > 1$. So, we find $n := \lfloor x\rfloor\in\mathbb N$ such
  that $y < n-1 < n \leq x$. This means that $x-y > n - (n-1) = 1$, in
  contradiction to $|x-y| \leq 1$.  \\ 3.  If $x - \lfloor x\rfloor <
    1/2$, then $2x - 2 \lfloor x\rfloor < 1$, which implies that
  $\lfloor 2x\rfloor = 2\lfloor x\rfloor$. Last, if $1/2 \leq x -
    \lfloor x\rfloor < 1$, then $1 \leq 2x - 2\lfloor x\rfloor < 2$, so
  $\lfloor 2x\rfloor = 2\lfloor x\rfloor +1$ and the result follows.
\end{proof}


\begin{lemma}
  Let $I = [0,1]^d$ and $|s-t| := \max_{i=1,...,d} |s_i - t_i|$ for
  $s,t\in I$. Let
  \begin{itemize}
    \item $D_n := \{0,1,...,2^n\}^n \cdot 2^{-n} \subseteq I$ for
          $n=0,1,...$, and $D =\bigcup_{n=0}^\infty D_n$;
    \item $m \in \mathbb N$ and $s,t \in D$ with $|t-s| \leq 2^{-m}$.
  \end{itemize}
  Then, there is $n \geq m$ and $s_m,...,s_n, t_m,...,t_n$ such that
  \begin{enumerate}
    \item $s_k, t_k \in D_k$ with $|s-s_k|, |t-t_k| \leq 2^{-k}$ for all
          $k=m,...,n$
    \item $|s_k - s_{k-1}|, |t_k - t_{k-1}| \leq 2^{-k}$,
    \item $|t_m - s_m| \leq 2^{-m}$,
    \item $s_n=s, t_n=t$.
  \end{enumerate}
\end{lemma}

\begin{proof}
  Since $s,t \in D = \bigcup_n D_n$, and $D_n \subseteq D_m$ for
  $n\geq m$, there is some $n \geq m$ with $s,t\in D_n$. For $k \in
    m,...,n$, we set
  $$s_k := \lfloor s2^k\rfloor 2^{-k}, \qquad t_k := \lfloor
    t2^k\rfloor 2^{-k} \in D_k.$$ 1.  Since $|x - \lfloor x \rfloor |
    \leq 1$ for all $x \in \mathbb R^d$ by Lemma~\ref{l:gauss}.1, we have that
  $$|s - s_k| = 2^{-k}|s2^k - \lfloor s2^k\rfloor| \leq 2^{-k}, \qquad
    k=m,...,n.$$ \\ 2. Using Lemma~\ref{l:gauss}.3, write
  $$ |s_k - s_{k-1}| = 2^{-k} | \lfloor 2s2^{k-1}\rfloor - 2\lfloor
    s2^{k-1}\rfloor| \leq 2^{-k}.$$ \\ 3. Since  $|t-s| \leq
    2^{-m}$, we have $|2^mt - s^ms| \leq 1$, so by Lemma~\ref{l:gauss}.2
  $$ |t_m - s_m| = 2^{-m} | \lfloor t2^m\rfloor - \lfloor s2^m
    \rfloor| \leq 2^{-m}.$$ \\ 4. We have $s2^n, t2^n \in \mathbb Z^d$
  since $s,t \in D_n$, so $s_n = 2^{-n} \lfloor s2^n\rfloor = 2^{-n} s
    2^n = s$ and $t_n = t$.
\end{proof}

\begin{remark}\label{rem1}
  Assume that $r(x_s, x_t) \leq 2^{-\gamma k}$ for all $s,t$ with
  $|t-s| = 2^{-k}$ for $k\geq m$. Then, for some $s,t\in D$ with
  $|t-s| \leq 2^{-m}$, with $s_k, t_k$ as in the above result and the
  triangle inequality,
  \begin{align*}
    t & = t_n = s_n + \Big(\sum_{k=m+1}^n t_k - t_{k-1} -(s_k -
    s_{k-1})\Big) + t_m - s_m,                                  \\ r(x_t, x_s) & \leq
    \Big(\sum_{k=m+1}^n r(x_{t_k}, x_{t_{k-1}}) + r(x_{s_k},
    x_{s_{k-1}}\Big) + r(x_{t_m}, x_{s_m})                      \\ & \leq 2 \sum_{k=m}^n
    2^{-\gamma k} \leq \tfrac{1}{1-2^{-\gamma}} 2^{-\gamma m}.
  \end{align*}
\end{remark}

\noindent
The proof of the continuity theorem follows the version in
\cite{karatzas1991brownian}.

\begin{theorem}[Continuous version; Kolmogorov, Chentsov]
  \label{T:kolchen}
  For some $d\in\mathbb N$ and
  $\sigma_1,\tau_1,...,\sigma_d,\tau_d>0$, let $I = \prod_{i=1}^d
    [\sigma_i,\tau_i]$, and $X = (X_t)_{t\in I}$ an $E$-valued
  stochastic process. Assume that there are $\alpha, \beta, C>0$ with
  $$ \mathbf E[r(X_s, X_t)^\alpha] \leq C|t-s|^{d+\beta}, \qquad 0\leq
    s,t\leq \tau.$$ There there exists a version $Y = (Y_t)_{t\in I}$ of
  $X$ such that, for some random variables $H>0$ and $K<\infty$,
  $$ \mathbf P\Big(\sup_{s\neq t, |t-s| \leq H} r(Y_s,
    Y_t)/|t-s|^\gamma \leq K\Big) = 1,
  $$ for every $\gamma\in(0,\beta/\alpha)$. In particular, $Y$ almost
  surely is locally Hölder of all orders $\gamma\in(0,\beta/\alpha)$,
  and has continuous paths.
\end{theorem}

\begin{proof}
  It suffices to show the assertion for $I=[0,1]^d$. The general case
  then follows by some scaling of $I$. We consider the set of times
  $$ D_n := \{0,1,...,2^n\}^n \cdot 2^{-n}$$ for $n=0,1,...$, $D
    =\bigcup_{n=0}^\infty D_n$. Using the Markov inequality, we write
  for any $n\in\mathbb N$ (note that $|\{s,t\in D_n, |t-s| = 2^{-n}\}|
    \leq d2^{nd}$), using Lemma~\ref{l:supsum},
  \begin{align*}
    \mathbf P\Big( & \sup_{s,t\in D_n, |t-s| = 2^{-n}} r(X_s, X_t)
    \geq 2^{-\gamma n} \Big) \leq 2^{\gamma \alpha n}\mathbf
    E\Big[\sup_{s,t\in D_n, |t-s| = 2^{-n}} r(X_s, X_t)^\alpha \Big]
    \\ & \leq \sum_{s,t\in D_n, |t-s| = 2^{-n}} 2^{\gamma\alpha
        n}\mathbf E[r(X_t,X_s)^\alpha] \leq Cd2^{nd} 2^{\gamma\alpha n}
    2^{-(d + \beta) n} = Cd 2^{(\gamma\alpha - \beta)n},
  \end{align*}
  and we see that the right hand side is summable. By the
  Borel-Cantelli Lemma,
  $$ N := \max \Big\{n: \sup_{s,t\in D_n, |t-s| = 2^{-n}} r(X_s, X_t)
    \geq 2^{-\gamma n}\Big\} + 1$$ is finite, almost surely. From this
  and Remark~\ref{rem1}, we conclude with $C' =
    \frac{1}{1-2^{-\gamma}}$,
  \begin{align*}
    \sup_{s,t\in D, s\neq t, |t-s| \leq 2^{-N}} r(X_s, X_t) & \leq
    \sup_{m \geq N} \Big( \sup_{s,t\in D, |t-s|\leq 2^{-m}} r(X_s,
    X_t)\Big) \leq C' \sup_{m \geq N} 2^{-\gamma m} = C' 2^{-\gamma
        N}.
  \end{align*}
  In other words, we see with $H = 2^{-N}$ and $K = C 2^{-\gamma N}$,
  $$ \mathbf P\Big[\sup_{s,t\in D, s\neq t, |t-s| \leq H} r(X_s,
      X_t)/|t-s|^\gamma \leq K\Big] = 1,
  $$
  i.e.\ $X$ is locally Hölder-$\gamma$ on $D$.

  With this and Lemma~\ref{l:holderext}.2, we can extend $X$
  Hölder-continuously on $I$, and call this extension $ Y =
    (Y_t)_{t\in I}$. In order to show that $Y$ is a modification of $X$,
  fix $t\in I$ and consider a sequence $t_1,t_2,...\in D$ with $t_n\to
    t$ as $n\to \infty$. Then, for all $\varepsilon>0$,
  $$\mathbf P(r(X_{t_n}, X_t) > \varepsilon) \leq \mathbf E[r(X_{t_n},
        X_t)^\alpha]/\varepsilon^\alpha \xrightarrow{n\to\infty} 0,$$
  i.e.\ $X_{t_n} \xrightarrow{n\to\infty}_p X_t$. Moreover, due to
  continuity of $Y$, we have $Y_{t_n} \xrightarrow{n\to\infty}_{fs}
    Y_t$. In particular, since $X_{t_n} = Y_{t_n}$ for all $n$, we have
  $\mathbf P(X_t= Y_t)=1$ by Lemma~\ref{l:puni} , which concludes the
  proof.
\end{proof}

\begin{lemma}\label{l:chain}
  Let $(I,q)$ and $(E,r)$ be metric spaces, and $f: I\to E$. Moreover, let $J \subseteq I$ be finite, $a,b,c\in\mathbb R_+$ with $a\geq 1$ and $n \in \{1,2,...\}$ such that $|J| \leq b a^n$. Then, there is $K \subseteq J^2$ such that
  \begin{align}
    |K|                                          & \leq a |J|, \label{eq:chain1}                 \\
    (s,t) \in K                                  & \Rightarrow q(s,t) \leq cn, \label{eq:chain2} \\
    \sup_{s,t\in J, q(s,t) \leq c} |f(t) - f(s)| & \leq 2 \sup_{(s,t) \in K} |f(s) - f(t)|.
    \label{eq:chain3}
  \end{align}
\end{lemma}

\begin{proof}
  Start with $V_1 = J$ and some $t_1 \in V_1$. We iteratively construct tuples $(V_\ell, t_\ell \in I, r_\ell \in \{1,...,d\}, B_\ell \subseteq V_\ell, C_\ell \subseteq V_\ell, K_\ell \subseteq V_\ell^2)$ such that $V_{\ell+1} = V_\ell \setminus B_\ell$ (hence, $\ell \mapsto V_\ell$ is decreasing), $t_\ell \in V_\ell$ is some arbitrary element, and $(r_\ell, B_\ell, C_\ell, K_\ell)$ are given by first finding $r_\ell \in \{1,2,...\}$ minimal with
  $$ |C_\ell| \leq ba^{r_\ell} \text{ for } C_\ell := \{s \in V_\ell: q(s, t_\ell) \leq r_\ell c \}$$
  (since $|V_\ell| \leq ba^d$, this $r_\ell$ exists uniquely) and
  $$ B_\ell := \{s\in V_\ell: r(s,t_\ell) \leq (r_\ell-1)c\} \subseteq C_\ell, \qquad K_\ell := \{t_\ell\} \times C_\ell. $$
  Note that this implies
  $$ |B_\ell| \geq ba^{r_\ell-1}, \qquad |K_\ell| = |C_\ell| \leq ba^{r_\ell} $$
  by definition of $r_\ell$, and since $t_\ell \in B_\ell$ in all cases. We continue this construction until $V_m = \emptyset$.
  \\
  We claim that
  $$ K := \bigcup_{\ell=1}^m K_\ell = \{(t_\ell,s): s \in V_\ell, q(t_\ell,s) \leq cr_\ell \text{ for some } \ell=1,2,...\}$$
  satisfies \eqref{eq:chain1}, \eqref{eq:chain2} and \eqref{eq:chain3}.
  In order to show \eqref{eq:chain1}, we have, since $B_1, B_2,...$ are disjoint,
  $$ \sum_\ell ba^{r_\ell-1} \leq \sum_\ell |B_\ell| \leq |J|.$$
  Hence, \eqref{eq:chain1} follows from
  $$ |K| \leq \sum_\ell |K_\ell| = \sum_\ell |C_\ell| \leq \sum_\ell ba^{r_\ell} \leq a|J|.$$
  For \eqref{eq:chain2}, we have for $(t_\ell,s) \in K_\ell \subseteq K$,
  $$ q(t_\ell, s) \leq cr_\ell \leq cd.$$
  Last, for \eqref{eq:chain3}, consider $(s,t) \in J$ with $q(s,t) \leq c$. (Recall that $\ell \mapsto V_\ell$ is decreasing with $V_1=J$ and $V_m = \emptyset$.) Find $\ell$ maximal with $s,t \in V_\ell$. Assume wlog that $s \notin V_{\ell+1}$, which implies $s \in B_\ell$ (since $V_{\ell+1} = V_\ell \setminus B_\ell$), which further implies $q(s, t_\ell) \leq (r_\ell-1)c$. (This implies $(t_\ell,s) \in K_\ell$.) Since $q(s,t) \leq c$, this gives $q(t, t_\ell) \leq q(t,s) + q(s, t_\ell) \leq r_\ell c$, so $s,t \in C_\ell$. From here, $(t_\ell, s), (t_\ell, t) \in K_\ell \subseteq K$, hence
  $$ q(f(s), f(t)) \leq q(f(s), f(t_\ell)) + q(f(t_\ell), f(t)) \leq 2\sup_{s', t' \in K} q(f(s'), f(t'))$$
  and we are done by taking $\sup_{s,t \in J. q(s,t) \leq c}$ on the left hand side.
\end{proof}


\begin{definition}[$\varepsilon$-cover]
  Let $\varepsilon>0$ and $(D,r)$ be some pseudometric space.\\ A set $D' \subseteq D$ is said to be an $\varepsilon$-cover of $D$ if $D = \bigcup_{x\in D'} B_\varepsilon(x)$. It is called minimal, if $D' \setminus \{x\}$ is no $\varepsilon$-cover for all $x \in D'$.
\end{definition}

\begin{lemma}
  Let $\varepsilon>0$, $(D,r)$ be some pseudometric space and $D_\varepsilon \subseteq D$ a minimal $\varepsilon$-cover of $D$. In addition, for $n=1,2,...$, set $D_n := D_{2^{-n}}$.
  \begin{enumerate}
    \item For any $x\in D$ there is $x' \in D_\varepsilon$ such that $r(x,x') < \varepsilon$.
    \item If $x \in D_\varepsilon$ is not isolated in $D$, there is $\varepsilon>0$ and $x' \neq x$ with $x'\in D_\varepsilon$ and $r(x,x') < 3 \varepsilon$.
    \item Let $m\leq k$ and $x \in D_k$. Then, there is a sequence $x_k :=x \in D_k, x_{k-1} \in D_{k-1},...,x_{m+1} \in D_{m+1}$ with $r(x_\ell, x_{\ell+1}) < 2^{-\ell-1}$ for $\ell=k,...,m$.
    \item  Let $m \leq k,\ell$, $x \in D_k, y\in D_\ell$ with $r(x,y) < 2^{-m}$. Then, there are sequences $x_k :=x, x_{k-1} \in D_{k-1},...,x_{m} \in D_{m}$ and $y_\ell :=y, y_{\ell-1} \in D_{\ell-1},...,y_{m} \in D_{m}$ as in 2.\ with $r(x_{m}, y_{m}) < 3\cdot 2^{-m}$.
  \end{enumerate}
\end{lemma}

\begin{proof}
  1. This follows from the definition of a $\varepsilon$-cover.
  \\ 2. Since $x$ is not isolated, there is $y \in D$ with $r(x,y) \in (\varepsilon, 2\varepsilon)$ for some $\varepsilon>0$. Let $x' \in D_\varepsilon$ with $r(x',y) < \varepsilon$, which exists by 1. Then, $r(x,x') \leq r(x,y) + r(y,x') < 3\varepsilon$.
  \\ 3. As 1.\ shows, there is $x_{k+1} \in D_{k+1}$ with $r(x, x_{k+1}) < 2^{-k-1}$. The assertion follows inductively.
  \\ 4. Using the triangle inequality,
  \begin{align*}
    r(x_{m}, y_{m}) & \leq r(x,y) + \sum_{i=m}^{k-1} r(x_i, x_{i+1}) + \sum_{i=m}^{\ell-1} r(y_i, y_{i+1}) \\ & < 2^{-m} + \sum_{i=m}^\infty 2^{-i-1} + \sum_{i=m}^\infty 2^{-i-1}  = 3\cdot 2^{-m}
  \end{align*}
\end{proof}

\begin{remark}
  Ok, this is simple, but, for $d>0$
  $$ \sum_{j=0}^n 2^{dj} = \frac{2^{d(n+1)}-1}{2^d-1} \leq \frac{2^d}{2^d-1}2^{dn}. $$
\end{remark}

\begin{theorem}[Continuous version; Kolmogorov, Chentsov]
  \label{T:kolchen_general}
  Let $(I, q)$ be a compact metric space and for $\varepsilon>0$, let $I_\varepsilon$ be a finite $\varepsilon$-cover of $I$. Assume that, for some $d \in \{1,2,...\}$, we have $c_1>0$ with
  $$|I_\varepsilon| \leq c_1\varepsilon^{-d}$$
  for $\varepsilon$ small enough. Assume that $X = (X_t)_{t\in I}$ is an $E$-valued stochastic process and there are $\alpha, \beta, c_2>0$ with
  $$ \mathbf E[r(X_s, X_t)^\alpha] \leq c_2 q(s,t)^{d+\beta}, \qquad
    s,t\in I.$$ Then, there exists a version $Y = (Y_t)_{t\in I}$ of
  $X$ such that, for some random variables $H>0$ and $K<\infty$,
  $$ \mathbf P\Big(\sup_{s\neq t, q(s,t) \leq H} r(Y_s,
    Y_t)/q(s,t)^\gamma \leq K\Big) = 1,
  $$ for every $\gamma\in(0,\beta/\alpha)$. In particular, $Y$ almost
  surely is locally Hölder of all orders $\gamma\in(0,\beta/\alpha)$,
  and has continuous paths.
\end{theorem}

\begin{proof}
  With a slight abuse of notation, we set $I_n := I_{2^{-n}}$. (So, $|I_n| \leq c_1 2^{dn}$.) In addition, $J_n := \bigcup_{j=0}^n I_j$, hence $|J_n| \leq c_1 \sum_{i=0}^n 2^{di} \leq c_3 2^{dn}$ with $c_3 = \frac{2^d}{2^d-1}c_1$. We use Lemma~\ref{l:chain} with $J = J_n$, $a = 2^d$, $b = c_3$, $c = 2^{-n}$ and $n=n$. This gives some $K_n \subseteq J_n^2$ with $|K_n| \leq c_3 2^d 2^{dn}$ such that $(s,t) \in K_n \Rightarrow q(s,t) \leq n\cdot 2^{-n}$ and
  $$\sup_{s,t \in J_n, q(s,t) \leq 2^{-n}} r(X_s, X_t) \leq 2 \sup_{(s,t) \in K_n}  r(X_s, X_t).$$
  Using the Markov inequality, we write for any $n\in\mathbb N$, using Lemma~\ref{l:supsum},
  \begin{align*}
    \mathbf P\Big( & \sup_{s,t\in J_n, q(s,t) < 2^{-n}} r(X_s, X_t)
    \geq 2^{-\gamma n} \Big) \leq
    \mathbf P\Big( 2\sup_{(s,t)\in K_n} r(X_s, X_t)
    \geq 2^{-\gamma n} \Big)                                               \\ & = \mathbf P\Big( \sup_{(s,t)\in K_n} r(X_s, X_t)^\alpha
    \geq 2^{-\alpha} 2^{-\gamma \alpha n} \Big) \leq 2^\alpha 2^{\gamma \alpha n}\mathbf
    E\Big[\sup_{(s,t) \in K_n} r(X_s, X_t)^\alpha \Big]
    \\ & \leq \sum_{(s,t)\in K_n} 2^\alpha 2^{\gamma\alpha
        n}\mathbf E[r(X_s,X_t)^\alpha] \leq c_3 2^d 2^{nd} 2^\alpha 2^{\gamma\alpha n}
    c_2 (3n\cdot 2^{-n})^{d + \beta} = c n^{d + \beta} 2^{(\gamma\alpha - \beta)n}
  \end{align*}
  with $c = c_3 2^d c_2$. So, we see that the right hand side is summable. By the
  Borel-Cantelli Lemma,
  $$ N := \max \Big\{n: \sup_{s,t\in J_n, q(s,t) < 2^{-n}} r(X_s, X_t)
    \geq 2^{-\gamma n}\Big\} + 1$$ 
  is finite, almost surely. We set $J := \bigcup_n J_n$ and $H_m := 2^{-(N+m)}$ (with $H_m>0$, almost surely). For $s\in J_k$ and $k>N+m$, let $s_k := s, s_{k-1},...,s_{N+m}$ be as in Lemma xxx, and analogously for $t\in K_\ell$ with $\ell > N+m$. From this and Remark~\ref{rem1}, we conclude with $c' = 1 + \frac{2}{1-2^{-\gamma}}$ and
  \begin{align*}
    & \sup_{s,t\in J, q(s,t) \leq H_m} r(X_s, X_t)
    \\ & \leq
    \sup_{k, \ell \geq N+m} \sup_{s \in K_k, t\in K_\ell, q(s,t) \leq H_m} r(x_{s_{N+m}}, y_{t_{N+m}}) + \sum_{i=N+m}^{k-1} r(X_{s_i}, X_{s_{i+1}}) +
    \sum_{i=N+m}^{\ell-1} r(X_{t_i}, X_{t_{i+1}})    \\ &
    \leq \sup_{s,t \in J_{N+m}, q(s,t) \leq 3\cdot H_m} r(X_s, X_t) +  2^{-\gamma N} + \sum_{i=N}^\infty 2^{-\gamma i} + \sum_{i=N}^\infty 2^{-\gamma i}
    = c' 2^{-\gamma N}.
  \end{align*}
  In other words, we see with $H = 3\cdot 2^{-N}$ and $K = c' 2^{-\gamma N}$,
  $$ \mathbf P\Big[\sup_{s,t\in J, s\neq t, q(s,t) \leq H} r(X_s,
      X_t)/q(s,t)^\gamma \leq K\Big] = 1,
  $$
  i.e.\ $X$ is locally Hölder-$\gamma$ on $K$.

  Then,
  \begin{align*}
    \sup & \Big\{\frac{r(X_s, X_t)}{q(s,t)^\gamma}: {s,t\in J, q(s,t) \leq 3\cdot 2^{-N}} \Big\} \\ & = \sup_{m=0,1,2,...} \sup \Big\{\frac{r(X_s, X_t)}{q(s,t)^\gamma}: s,t\in J, H_{m+1}  < q(s,t) \leq H_m\Big\} \\ & \leq \sup_{m=0, 1,...} 2^{(N+m+1)\gamma}\sup \big\{r(X_s, X_t): s,t\in J, q(s,t) \leq H_m\big\} \\ & \leq \sup_{m=0,1,...} 2^{(N+m+1)\gamma} c' 2^{-\gamma (N+m)} \\ & = 2^{\gamma}c'.
  \end{align*}


  With this and Lemma~\ref{l:holderext}.2, we can extend $X$
  Hölder-continuously on $I$, and call this extension $ Y =
    (Y_t)_{t\in I}$. In order to show that $Y$ is a modification of $X$,
  fix $t\in I$ and consider a sequence $t_1,t_2,...\in D$ with $t_n\to
    t$ as $n\to \infty$. Then, for all $\varepsilon>0$,
  $$\mathbf P(r(X_{t_n}, X_t) > \varepsilon) \leq \mathbf E[r(X_{t_n},
        X_t)^\alpha]/\varepsilon^\alpha \xrightarrow{n\to\infty} 0,$$
  i.e.\ $X_{t_n} \xrightarrow{n\to\infty}_p X_t$. Moreover, due to
  continuity of $Y$, we have $Y_{t_n} \xrightarrow{n\to\infty}_{fs}
    Y_t$. In particular, since $X_{t_n} = Y_{t_n}$ for all $n$, we have
  $\mathbf P(X_t= Y_t)=1$ by Lemma~\ref{l:puni} , which concludes the
  proof.
\end{proof}



\bibliographystyle{alpha}
\bibliography{main}


\end{document}
