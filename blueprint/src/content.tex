% In this file you should put the actual content of the blueprint.
% It will be used both by the web and the print version.
% It should *not* include the \begin{document}
%
% If you want to split the blueprint content into several files then
% the current file can be a simple sequence of \input. Otherwise It
% can start with a \section or \chapter for instance.

\part{Brownian motion}

\textbf{Overview}

This part of the blueprint is a guide to the formalization of a standard Brownian motion in Lean using Mathlib. There are two main parts to this formalization:
\begin{itemize}
  \item a development of the theory of Gaussian distributions, the construction of a projective family of Gaussian distributions and its projective limit by the Kolmogorov extension theorem,
  \item a proof of a Kolmogorov-Chentsov continuity theorem, following \cite{kratschmer2023kolmogorov}.
\end{itemize}

Putting the two sides together, we then build a stochastic process that fits the definition of a Brownian motion on the real line.

\textbf{Status} The formalization is complete.

\textbf{Formalization authors} Rémy Degenne, Markus Himmel, David Ledvinka, Etienne Marion, Peter Pfaffelhuber.

With additional contributions from Jonas Bayer, Lorenzo Loccioli, Pietro Monticone, Alessio Rondelli and Jérémy Scanvic.

\chapter{Characteristic functions}
\label{chap:characteristic_function}


\begin{definition}[Characteristic function]\label{def:charFunCLM}
The characteristic function of a measure $\mu$ on a normed space $E$ is the function $E^* \to \mathbb{C}$ defined by
\begin{align*}
  \hat{\mu}(L) = \int_E e^{i L(x)} \: d\mu(x) \: .
\end{align*}
\end{definition}


\begin{theorem}\label{thm:ext_of_charFunCLM}
  \uses{def:charFunCLM}
In a separable Banach space, if two finite measures have same characteristic function, they are equal.
\end{theorem}

\begin{proof}

\end{proof}


\begin{definition}[Characteristic function]\label{def:charFun}
  \mathlibok
  \lean{MeasureTheory.charFun}
The characteristic function of a measure $\mu$ on an inner product space $E$ is the function $E \to \mathbb{C}$ defined by
\begin{align*}
  \hat{\mu}(t) = \int_E e^{i \langle t, x \rangle} \: d\mu(x) \: .
\end{align*}
This is equal to the normed space version of the characteristic function applied to the linear map $x \mapsto \langle t, x \rangle$.
\end{definition}


\begin{theorem}\label{thm:ext_of_charFun}
  \uses{def:charFun}
  \mathlibok
  \lean{MeasureTheory.Measure.ext_of_charFun}
In a separable Hilbert space, if two finite measures have same characteristic function, they are equal.
\end{theorem}

\begin{proof}\leanok

\end{proof}

\chapter{Stochastic processes}
\label{chap:process}

Let $T$ be an index set and $\Omega$ a measurable space, with measure $\mathbb{P}$.
A stochastic process is a function $X : T \to \Omega \to E$, where $E$ is another measurable space, such that for all $t \in T$, $X_t : \Omega \to E$ is $\mathbb{P}$-a.e. measurable.


\begin{definition}[Law of a stochastic process]\label{def:processLaw}
  \leanok
The law of a stochastic process $X$ is the measure on the measurable space $E^T$ obtained by pushing forward the measure $\mathbb{P}$ by the map $\omega \mapsto X(\cdot, \omega)$.
\end{definition}

\textbf{Lean remark}: we don't use a Lean definition for the law, but write the map in full.

\begin{definition}[Modification]\label{def:modification}
  \leanok
We say that a stochastic process $Y$ is a \emph{modification} of another stochastic process $X$ if for all $t \in T$, $Y_t =_{\mathbb{P}\text{-a.e.}} X_t$.
\end{definition}

\textbf{Lean remark}: we don't use a Lean definition for being a modification, but write explicitly the condition $\forall t \in T,\ Y_t =_{\mathbb{P}\text{-a.e.}} X_t$~.

\begin{definition}[Indistinguishable]\label{def:indistinguishable}
We say that a stochastic processes $Y$ is a \emph{indistinguishable} from $X$ if $\mathbb{P}$-a.e., for all $t \in T$, $X_t = Y_t$.
\end{definition}

A summary of the next few lemmas is this:
\begin{itemize}
  \item indistinguishable $\implies$ modification $\implies$ same law,
  \item modification and continuous with $T$ separable $\implies$ indistinguishable.
\end{itemize}


\begin{lemma}\label{lem:Indistinguishable.Modification}
  \uses{def:indistinguishable, def:modification}
If $Y$ is indistinguishable from $X$, then $Y$ is a modification of $X$.
\end{lemma}

\begin{proof}
Obvious.
\end{proof}


\begin{lemma}\label{lem:map_eq_of_modification}
  \uses{def:modification}
Let $X, Y : T \to \Omega \to E$ be two stochastic processes that are modifications of each other.
Then for all $t_1, \ldots, t_n \in T$, the random vector $(X_{t_1}, \ldots, X_{t_n})$ has the same distribution as the random vector $(Y_{t_1}, \ldots, Y_{t_n})$.
That is, $X$ and $Y$ have same finite-dimensional distributions.
\end{lemma}

\begin{proof}
By the modification property, almost surely $X_{t_i} = Y_{t_i}$ for all $i \in [n]$.
Thus the function $\omega \mapsto (X_{t_1}(\omega), \ldots, X_{t_n}(\omega))$ is equal to $\omega \mapsto (Y_{t_1}(\omega), \ldots, Y_{t_n}(\omega))$ almost surely, hence the maps of $\mathbb{P}$ by these two functions are equal.
\end{proof}


\begin{lemma}\label{lem:map_eq_iff}
  \uses{def:processLaw}
Let $X, Y : T \to \Omega \to E$ be two stochastic processes.
Then $X$ and $Y$ have same finite-dimensional distributions if and only if they have the same law.
\end{lemma}

\begin{proof}
TODO: consider the $\pi$-system of cylinder sets.
\end{proof}


\begin{lemma}\label{lem:indistinguishable_of_modification_of_continuous}
  \uses{def:modification, def:indistinguishable}
Let $T$ and $E$ be topological spaces and suppose that $T$ is separable Hausdorff.
Let $X, Y : T \to \Omega \to E$ be two stochastic processes that are modifications of each other and are almost surely continuous.
Then $X$ and $Y$ are indistinguishable.
\end{lemma}

\begin{proof}
Since $T$ is separable, it has a countable dense subset $D$.
Since $D$ is countable,
\begin{align*}
  (\forall t \in D, \mathbb{P}\text{-a.e.}, X_t = Y_t)
  \iff (\mathbb{P}\text{-a.e.}, \forall t \in D, X_t = Y_t)
\end{align*}
Hence by the modification property we have that almost surely, for all $t \in D$, $X_t = Y_t$.
Then almost surely $X$ and $Y$ are continuous functions which are equal on a dense subset of $T$: those two functions are equal everywhere.
\end{proof}

\chapter{Gaussian distributions}
\label{chap:gaussian}

\section{Gaussian measures}
\label{sec:gaussian_measures}

\subsection{Real Gaussian measures}

\begin{definition}[Real Gaussian measure]\label{def:gaussianReal}
  \mathlibok
  \lean{ProbabilityTheory.gaussianReal}
  The real Gaussian measure with mean $\mu \in \mathbb{R}$ and variance $\sigma^2 > 0$ is the measure on $\mathbb{R}$ with density $\frac{1}{\sqrt{2 \pi \sigma^2}} \exp\left(-\frac{(x - \mu)^2}{2 \sigma^2}\right)$ with respect to the Lebesgue measure.
  The real Gaussian measure with mean $\mu \in \mathbb{R}$ and variance $0$ is the Dirac measure $\delta_\mu$.
  We denote this measure by $\mathcal{N}(\mu, \sigma^2)$.
\end{definition}


\begin{lemma}\label{lem:charFun_gaussianReal}
  \uses{def:gaussianReal, def:charFun}
  \mathlibok
  \lean{ProbabilityTheory.charFun_gaussianReal}
The characteristic function of a real Gaussian measure with mean $\mu$ and variance $\sigma^2$ is given by
$x \mapsto \exp\left(i \mu x - \frac{\sigma^2 x^2}{2}\right)$.
\end{lemma}

\begin{proof}\leanok

\end{proof}


\begin{lemma}\label{lem:centralMoment_two_mul_gaussianReal}
  \uses{def:gaussianReal}
The central moment of order $2n$ of a real Gaussian measure $\mathcal{N}(\mu, \sigma^2)$ is given by
\begin{align*}
  \mathbb{E}[(X - \mu)^{2n}] = \sigma^{2n} (2n - 1)!! \: ,
\end{align*}
in which $(2n - 1)!! = (2n - 1)(2n - 3) \cdots 3 \cdot 1$ is the double factorial of $2n - 1$.
\end{lemma}

\begin{proof}

\end{proof}


\subsection{Gaussian measures on a Banach space}

That kind of generality is not needed for this project, but we happen to have results about Gaussian measures on a Banach space in Mathlib, so we will use them.
The main reference for this section is \cite{hairer2009introduction}.

Let $F$ be a separable Banach space.

\begin{definition}[Gaussian measure]\label{def:IsGaussian}
  \uses{def:gaussianReal}
  \mathlibok
  \lean{ProbabilityTheory.IsGaussian}
A measure $\mu$ on $F$ is Gaussian if for every continuous linear form $L \in F^*$, the pushforward measure $L_* \mu$ is a Gaussian measure on $\mathbb{R}$.
\end{definition}


\begin{lemma}\label{lem:IsGaussian.IsProbabilityMeasure}
  \uses{def:IsGaussian}
  \mathlibok
A Gaussian measure is a probability measure.
\end{lemma}

\begin{proof}\leanok

\end{proof}


\begin{definition}[Centered measure]\label{def:IsCentered}
A measure $\mu$ on $F$ is centered if for every continuous linear form $L \in F^*$, $\mu[L] = 0$.
\end{definition}


\begin{theorem}\label{thm:isGaussian_iff_charFunDual_eq}
  \uses{def:IsGaussian, def:charFunDual}
  \mathlibok
  \lean{ProbabilityTheory.isGaussian_iff_charFunDual_eq}
A finite measure $\mu$ on $F$ is Gaussian if and only if for every continuous linear form $L \in F^*$, the characteristic function of $\mu$ at $L$ is
\begin{align*}
  \hat{\mu}(L) = \exp\left(i \mu[L] - \mathbb{V}_\mu[L] / 2\right) \: ,
\end{align*}
in which $\mathbb{V}_\mu[L]$ is the variance of $L$ with respect to $\mu$.
\end{theorem}

\begin{proof}\uses{thm:ext_of_charFunDual, lem:charFun_gaussianReal}\leanok

\end{proof}



\paragraph{Transformations of Gaussian measures}

\begin{lemma}\label{lem:isGaussian_map}
  \uses{def:IsGaussian}
  \mathlibok
  \lean{ProbabilityTheory.isGaussian_map}
Let $F, G$ be two Banach spaces, let $\mu$ be a Gaussian measure on $F$ and let $T : F \to G$ be a continuous linear map.
Then $T_*\mu$ is a Gaussian measure on $G$.
\end{lemma}

\begin{proof}\leanok

\end{proof}


\begin{lemma}\label{lem:isGaussian_add_const}
  \uses{def:IsGaussian}
  \leanok
  % This is an instance without name in the code, hence we don't give a \lean{...}.
Let $\mu$ be a Gaussian measure on $F$ and let $c \in F$.
Then the measure $\mu$ translated by $c$ (the map of $\mu$ by $x \mapsto x + c$) is a Gaussian measure on $F$.
\end{lemma}

\begin{proof}\leanok

\end{proof}


\begin{lemma}\label{lem:isGaussian_conv}
  \uses{def:IsGaussian}
  \mathlibok
  %\lean{ProbabilityTheory.isGaussian_conv} -- need a Mathlib update
The convolution of two Gaussian measures is a Gaussian measure.
\end{lemma}

\begin{proof}\leanok

\end{proof}



\paragraph{Fernique's theorem}


\begin{theorem}\label{thm:exists_integrable_exp_sq_of_map_rotation_eq_self}
  \leanok
  % In a Mathlib PR
Let $\mu$ be a finite measure on $F$ such that $\mu \times \mu$ is invariant under the rotation of angle $-\frac{\pi}{4}$.
Then there exists $C > 0$ such that the function $x \mapsto \exp (C \Vert x \Vert ^ 2)$ is integrable with respect to $\mu$.
\end{theorem}

\begin{proof}\leanok

\end{proof}


\begin{lemma}\label{lem:IsGaussian.map_rotation_eq_self}
  \uses{def:IsGaussian}
  \leanok
  % In a Mathlib PR
For a Gaussian measure $\mu$, $\mu \times \mu$ is invariant by rotation.
\end{lemma}

\begin{proof}\leanok
  \uses{lem:isGaussian_conv}

\end{proof}


\begin{theorem}[Fernique's theorem]\label{thm:IsGaussian.exists_integrable_exp_sq}
  \uses{def:IsGaussian}
  \leanok
  \lean{ProbabilityTheory.IsGaussian.exists_integrable_exp_sq}
For a Gaussian measure, there exists $C > 0$ such that the function $x \mapsto \exp (C \Vert x \Vert ^ 2)$ is integrable.
\end{theorem}

\begin{proof}\leanok
  \uses{thm:isGaussian_iff_charFunDual_eq, lem:IsGaussian.IsProbabilityMeasure, thm:exists_integrable_exp_sq_of_map_rotation_eq_self, lem:IsGaussian.map_rotation_eq_self}

\end{proof}


\begin{lemma}\label{lem:IsGaussian.memLp_id}
  \uses{def:IsGaussian}
  \leanok
  \lean{ProbabilityTheory.IsGaussian.memLp_id}
A Gaussian measure $\mu$ has finite moments of all orders.
In particular, there is a well defined mean $m_\mu := \mu[\mathrm{id}]$, and for all $L \in F^*$, $\mu[L] = L(m_\mu)$.
\end{lemma}

\begin{proof}\leanok
  \uses{thm:IsGaussian.exists_integrable_exp_sq}

\end{proof}


\begin{definition}[Covariance]\label{def:covarianceBilin}
  \mathlibok
  \lean{ProbabilityTheory.covarianceBilin}
The covariance bilinear form of a measure $\mu$ with finite second moment is the continuous bilinear form $C_\mu : F^* \times F^* \to \mathbb{R}$ with
\begin{align*}
  C_\mu(L_1, L_2) = \int_x (L_1(x) - L_1(m_\mu)) (L_2(x) - L_2(m_\mu)) \: d\mu(x) \: .
\end{align*}
\end{definition}

A Gaussian measure has finite second moment by Lemma~\ref{lem:IsGaussian.memLp_id}, hence its covariance bilinear form is well defined.

\begin{lemma}\label{lem:covarianceBilin_same_eq_variance}
  \uses{def:covarianceBilin}
  \mathlibok
  \lean{ProbabilityTheory.covarianceBilin_same_eq_variance}
For $\mu$ a measure on $F$ with finite second moment and $L \in F^*$, $C_\mu(L, L) = \mathbb{V}_\mu[L]$.
\end{lemma}

\begin{proof}\leanok

\end{proof}


\subsection{Gaussian measures on a finite dimensional Hilbert space}

We specialize directly from Banach space to finite dimensional Hilbert space since that's what we need in this project, although there are results for Gaussian measures on infinite dimensional Hilbert spaces that would worth stating.

\begin{lemma}\label{lem:isGaussian_iff_charFunDual_inner_eq}
  \uses{def:IsGaussian, def:charFunDual, def:charFun}
A finite measure $\mu$ on a separable Hilbert space $E$ is Gaussian if and only if for every $t \in E$, the characteristic function of $\mu$ at $t$ is
\begin{align*}
  \hat{\mu}(t) =  \exp\left(i \mu[\langle t, \cdot \rangle] - \mathbb{V}_\mu[\langle t, \cdot \rangle] / 2\right) \: .
\end{align*}
\end{lemma}

\begin{proof}
  \uses{thm:isGaussian_iff_charFunDual_eq}
By Theorem~\ref{thm:isGaussian_iff_charFunDual_eq}, $\mu$ is Gaussian iff for every continuous linear form $L \in E^*$, the characteristic function of $\mu$ at $L$ is
\begin{align*}
  \hat{\mu}(L) = \exp\left(i \mu[L] - \mathbb{V}_\mu[L] / 2\right) \: .
\end{align*}
Every continuous linear form $L \in E^*$ can be written as $L(x) = \langle t, x \rangle$ for some $t \in E$, hence we have that $\mu$ is Gaussian iff for every $t \in E$,
\begin{align*}
  \hat{\mu}(t) = \exp\left(i \mu[\langle t, \cdot \rangle] - \mathbb{V}_\mu[\langle t, \cdot \rangle] / 2\right) \: .
\end{align*}
\end{proof}

Let $E$ be a finite dimensional Hilbert space. We denote by $\langle \cdot, \cdot \rangle$ the inner product on $E$ and by $\Vert \cdot \Vert$ the associated norm.


\begin{definition}[Covariance matrix]\label{def:covMatrix}
  \uses{def:IsGaussian}
The covariance matrix of a Gaussian measure $\mu$ on $E$ is the positive semidefinite matrix $\Sigma_\mu$ such that for $u, v \in E$,
\begin{align*}
  \langle u, \Sigma_\mu v\rangle = \mu[\langle u, x - m_\mu \rangle \langle x - m_\mu, v \rangle] \: .
\end{align*}
\end{definition}


\begin{lemma}\label{lem:covarianceBilin_map}
  \uses{def:covarianceBilin}
Let $E$ and $F$ be two Hilbert spaces, $\mu$ a measure on $E$ with finite second moment, and $L : E \to F$ a continuous linear map.
Then the covariance bilinear form of the measure $L_*\mu$ is given by
\begin{align*}
  C_{L_*\mu}(\langle u, \cdot\rangle, \langle v, \cdot\rangle)
  &= C_\mu(\langle L^\dagger(u), \cdot\rangle, \langle L^\dagger(v), \cdot\rangle)
  \: ,
\end{align*}
in which $L^\dagger : F \to E$ is the adjoint of $L$.
\end{lemma}

\begin{proof}
\begin{align*}
  C_{L_*\mu}(\langle u, \cdot\rangle, \langle v, \cdot\rangle)
  &= (\pi_{IJ*}\mu)\left[\langle u, x - m_{L_*\mu}\rangle \langle x - m_{L_*\mu}, v \rangle\right]
  \\
  &= \mu\left[\langle u, L(x) - L(m_\mu)\rangle \langle L(x) - L(m_\mu), v \rangle \right]
  \\
  &= \mu\left[\langle L^\dagger(u), x - m_\mu\rangle \langle x - m_\mu, L^\dagger(v) \rangle \right]
  \\
  &= C_\mu(\langle L^\dagger(u), \cdot\rangle, \langle L^\dagger(v), \cdot\rangle)
  \: .
\end{align*}
\end{proof}


\begin{lemma}\label{lem:covMatrix_map}
  \uses{def:covMatrix}
Let $E$ and $F$ be two finite dimensional Hilbert spaces, $\mu$ a measure on $E$ with finite second moment, and $L : E \to F$ a continuous linear map.
Then the covariance matrix of the measure $L_*\mu$ has entries
\begin{align*}
  \langle e_i, \Sigma_{L_*\mu} e_j\rangle
  &= \langle L^\dagger(e_i), \Sigma_\mu L^\dagger(e_j)\rangle
  \: ,
\end{align*}
in which $L^\dagger : F \to E$ is the adjoint of $L$.
\end{lemma}

\begin{proof}
  \uses{lem:covarianceBilin_map}

\end{proof}


\begin{lemma}\label{lem:IsGaussian.charFun_eq}
  \uses{def:IsGaussian, def:charFun, def:covMatrix}
The characteristic function of a Gaussian measure $\mu$ on $E$ is given by
\begin{align*}
  \hat{\mu}(t) = \exp\left(i \langle t, m_\mu \rangle - \frac{1}{2} \langle t, \Sigma_\mu t \rangle\right) \: .
\end{align*}
\end{lemma}

\begin{proof}
  \uses{lem:isGaussian_iff_charFunDual_inner_eq, lem:IsGaussian.memLp_id, lem:covarianceBilin_same_eq_variance}
By Lemma~\ref{lem:isGaussian_iff_charFunDual_inner_eq}, for every $t \in E$,
\begin{align*}
  \hat{\mu}(t) = \exp\left(i \mu[\langle t, \cdot \rangle] - \mathbb{V}_\mu[\langle t, \cdot \rangle] / 2\right) \: .
\end{align*}
By Lemma~\ref{lem:IsGaussian.memLp_id}, $\mu$ has finite first moment and $\mu[\langle t, \cdot \rangle] = \langle t, m_\mu \rangle$.

TODO: the second moment is also finite and we can get to the covariance matrix.
\end{proof}


\begin{lemma}\label{lem:isGaussian_iff_charFun_eq}
  \uses{def:IsGaussian, def:charFun, def:covMatrix}
A finite measure $\mu$ on $E$ is Gaussian if and only if there exists $m \in E$ and $\Sigma$ positive semidefinite such that for all $t \in E$, the characteristic function of $\mu$ at $t$ is
\begin{align*}
  \hat{\mu}(t) = \exp\left(i \langle t, m \rangle - \frac{1}{2} \langle t, \Sigma t \rangle\right) \: ,
\end{align*}
If that's the case, then $m = m_\mu$ and $\Sigma = \Sigma_\mu$.
\end{lemma}

Note that this lemma does not say that there exists a Gaussian measure for any such $m$ and $\Sigma$.
We will prove that later.

\begin{proof}\uses{lem:IsGaussian.charFun_eq, lem:charFun_map_eq_charFunDual_smul, thm:ext_of_charFun}
Lemma~\ref{lem:IsGaussian.charFun_eq} states that the characteristic function of a Gaussian measure has the wanted form.

Suppose now that there exists $m \in E$ and $\Sigma$ positive semidefinite such that for all $t \in E$, $\hat{\mu}(t) = \exp\left(i \langle t, m \rangle - \frac{1}{2} \langle t, \Sigma t \rangle\right)$.

We need to show that for all $L \in E^*$, $L_*\mu$ is a Gaussian measure on $\mathbb{R}$.
Such an $L$ can be written as $\langle u, \cdot \rangle$ for some $u \in E$.
Let then $u \in E$. We compute the characteristic function of $\langle u, \cdot\rangle_*\mu$ at $x \in \mathbb{R}$ with Lemma~\ref{lem:charFun_map_eq_charFunDual_smul}:
\begin{align*}
  \widehat{\langle u, \cdot\rangle_*\mu}(x)
  &= \hat{\mu}(x \cdot u)
  \\
  &= \exp\left(i x \langle u, m \rangle - \frac{1}{2} x^2 \langle u, \Sigma u \rangle\right)
  \: .
\end{align*}
This is the characteristic function of a Gaussian measure on $\mathbb{R}$ with mean $\langle u, m \rangle$ and variance $\langle u, \Sigma u \rangle$.
By Theorem~\ref{thm:ext_of_charFun}, $\langle u, \cdot\rangle_*\mu$ is Gaussian, hence $\mu$ is Gaussian.
\end{proof}


\begin{definition}[Standard Gaussian measure]\label{def:stdGaussian}
  \uses{def:gaussianReal}
  \leanok
  \lean{ProbabilityTheory.stdGaussian}
Let $(e_1, \ldots, e_d)$ be an orthonormal basis of $E$ and let $\mu$ be the standard Gaussian measure on $\mathbb{R}$.
The standard Gaussian measure on $E$ is the pushforward measure of the product measure $\mu \times \ldots \times \mu$ by the map $x \mapsto \sum_{i=1}^d x_i \cdot e_i$.
\end{definition}

The fact that this definition does not depend on the choice of basis will be a consequence of the fact that its characteristic function does not depend on the basis.

\begin{lemma}\label{lem:isCentered_stdGaussian}
  \uses{def:stdGaussian, def:IsCentered}
The standard Gaussian measure on $E$ is centered, i.e., $\mu[L] = 0$ for every $L \in E^*$.
\end{lemma}

\begin{proof}

\end{proof}


\begin{lemma}\label{lem:isProbabilityMeasure_stdGaussian}
  \uses{def:stdGaussian}
  \leanok
  \lean{ProbabilityTheory.isProbabilityMeasure_stdGaussian}
The standard Gaussian measure is a probability measure.
\end{lemma}

\begin{proof}\leanok

\end{proof}


\begin{lemma}\label{lem:charFun_stdGaussian}
  \uses{def:stdGaussian, def:charFun}
The characteristic function of the standard Gaussian measure on $E$ is given by
\begin{align*}
  \hat{\mu}(t) = \exp\left(-\frac{1}{2} \Vert t \Vert^2 \right) \: .
\end{align*}
\end{lemma}

\begin{proof}
  \uses{lem:charFun_gaussianReal, lem:isCentered_stdGaussian}

\end{proof}


\begin{lemma}\label{lem:isGaussian_stdGaussian}
  \uses{def:stdGaussian, def:IsGaussian}
  \leanok
  \lean{ProbabilityTheory.isGaussian_stdGaussian}
The standard Gaussian measure on $E$ is a Gaussian measure.
\end{lemma}

\begin{proof}
  \uses{lem:isGaussian_iff_charFun_eq, lem:charFun_stdGaussian, lem:isProbabilityMeasure_stdGaussian}
Since the standard Gaussian is a probability measure (hence finite), we can apply Lemma~\ref{lem:isGaussian_iff_charFun_eq} that states that it suffices to show that the characteristic function has a particular form.
That form is given by Lemma~\ref{lem:charFun_stdGaussian}.
\end{proof}


\begin{lemma}\label{lem:integral_id_stdGaussian}
  \uses{def:stdGaussian}
The mean of the standard Gaussian measure is $0$.
\end{lemma}

\begin{proof}
  \uses{lem:isCentered_stdGaussian}

\end{proof}


\begin{lemma}\label{lem:covMatrix_stdGaussian}
  \uses{def:stdGaussian}
The covariance matrix of the standard Gaussian measure is the identity matrix.
\end{lemma}

\begin{proof}

\end{proof}


\begin{definition}[Multivariate Gaussian]\label{def:multivariateGaussian}
  \uses{def:stdGaussian}
  \leanok
  \lean{ProbabilityTheory.multivariateGaussian}
The multivariate Gaussian measure on $\mathbb{R}^d$ with mean $m \in \mathbb{R}^d$ and covariance matrix $\Sigma \in \mathbb{R}^{d \times d}$, with $\Sigma$ positive semidefinite, is the pushforward measure of the standard Gaussian measure on $\mathbb{R}^d$ by the map $x \mapsto m + \Sigma^{1/2} x$.
We denote this measure by $\mathcal{N}(m, \Sigma)$.
\end{definition}


\begin{lemma}\label{lem:integral_id_multivariateGaussian}
  \uses{def:multivariateGaussian}
The mean of the multivariate Gaussian measure $\mathcal{N}(m, \Sigma)$ is $m$.
\end{lemma}

\begin{proof}
  \uses{lem:integral_id_stdGaussian}

\end{proof}


\begin{lemma}\label{lem:covMatrix_multivariateGaussian}
  \uses{def:multivariateGaussian}
The covariance matrix of the multivariate Gaussian measure $\mathcal{N}(m, \Sigma)$ is $\Sigma$.
\end{lemma}

\begin{proof}
  \uses{lem:covMatrix_stdGaussian}

\end{proof}


\begin{lemma}\label{lem:isGaussian_multivariateGaussian}
  \uses{def:multivariateGaussian, def:IsGaussian}
  \leanok
  \lean{ProbabilityTheory.isGaussian_multivariateGaussian}
A multivariate Gaussian measure is a Gaussian measure.
\end{lemma}

\begin{proof}
  \uses{lem:isGaussian_stdGaussian, lem:isGaussian_add_const, lem:isGaussian_map}
The multivariate Gaussian measure is the pushforward of the standard Gaussian measure by an affine map, and is thus Gaussian by Lemma~\ref{lem:isGaussian_add_const} and Lemma~\ref{lem:isGaussian_map}.
\end{proof}


\begin{theorem}\label{thm:charFun_multivariateGaussian}
  \uses{def:multivariateGaussian, def:charFun}
The characteristic function of a multivariate Gaussian measure $\mathcal{N}(m, \Sigma)$ is given by
\begin{align*}
  \hat{\mu}(t) = \exp\left(i \langle m, t \rangle - \frac{1}{2} \langle t, \Sigma t \rangle\right)
  \: .
\end{align*}
\end{theorem}

\begin{proof}
  \uses{lem:isGaussian_multivariateGaussian, lem:IsGaussian.charFun_eq, lem:integral_id_multivariateGaussian, lem:covMatrix_multivariateGaussian}
Since the multivariate Gaussian measure is a Gaussian measure, we can apply Lemma~\ref{lem:IsGaussian.charFun_eq} to it.
It suffices then to show that the mean and the covariance matrix of the multivariate Gaussian measure are equal to $m$ and $\Sigma$, respectively.
This is given by Lemma~\ref{lem:integral_id_multivariateGaussian} and Lemma~\ref{lem:covMatrix_multivariateGaussian}.
\end{proof}


\section{Gaussian processes}
\label{sec:gaussian_processes}

\begin{definition}[Gaussian process]\label{def:IsGaussianProcess}
  \uses{def:IsGaussian}
  \leanok
  \lean{ProbabilityTheory.IsGaussianProcess}
A process $X : T \to \Omega \to E$ is Gaussian if for every finite subset $t_1, \ldots, t_n \in T$, the random vector $(X_{t_1}, \ldots, X_{t_n})$ has a Gaussian distribution.
\end{definition}


\begin{lemma}\label{lem:map_eq_of_version}
Let $X, Y : T \to \Omega \to E$ be two stochastic processes that are versions of each other (that is, for all $t \in T$, $X_t =_{a.e.} Y_t$).
Then for all $t_1, \ldots, t_n \in T$, the random vector $(X_{t_1}, \ldots, X_{t_n})$ has the same distribution as the random vector $(Y_{t_1}, \ldots, Y_{t_n})$.
\end{lemma}

\begin{proof}
For all measurable sets $A \subseteq E^n$, we have
\begin{align*}
  \vert \mathbb{P}((X_{t_1}, \ldots, X_{t_n}) \in A) - \mathbb{P}((Y_{t_1}, \ldots, Y_{t_n}) \in A) \vert
  &\le \mathbb{P}(\exists i \in [n], X_{t_i} \ne Y_{t_i})
  \\
  &\le \sum_{i=1}^n \mathbb{P}(X_{t_i} \ne Y_{t_i})
  \\
  &= 0
  \: .
\end{align*}
\end{proof}


\begin{lemma}\label{lem:isGaussianProcess_of_version}
  \uses{def:IsGaussianProcess}
  \leanok
  \lean{ProbabilityTheory.IsGaussianProcess.version}
Let $X, Y : T \to \Omega \to E$ be two stochastic processes that are versions of each other (that is, for all $t \in T$, $X_t =_{a.e.} Y_t$).
If $X$ is a Gaussian process, then $Y$ is a Gaussian process as well.
\end{lemma}

\begin{proof}
  \uses{lem:map_eq_of_version}
Being a Gaussian process is defined in terms of the distribution of finite-dimensional random vectors.
By Lemma~\ref{lem:map_eq_of_version}, the random vector $(Y_{t_1}, \ldots, Y_{t_n})$ has the same distribution as the random vector $(X_{t_1}, \ldots, X_{t_n})$ for all $t_1, \ldots, t_n \in T$.
\end{proof}


\begin{lemma}\label{lem:indistinguishable_of_version_of_continuous}
Let $X, Y : T \to \Omega \to E$ be two stochastic processes that are versions of each other (that is, for all $t \in T$, $X_t =_{a.e.} Y_t$) and are almost surely continuous.
Then $X$ and $Y$ are indistinguishable. That is, almost surely, $X_t = Y_t$ for all $t \in T$ simultaneously.

TODO: hypotheses on $T, E$?
\end{lemma}

\begin{proof}

\end{proof}

\chapter{Projective family of the Brownian motion}
\label{chap:projective_family}


\section{Kolmogorov extension theorem}

This theorem has been formalized in the repository \href{https://github.com/RemyDegenne/kolmogorov_extension4}{kolmogorov\_extension4}.

\begin{definition}[Projective family]\label{def:IsProjectiveMeasureFamily}
  \mathlibok
  \lean{MeasureTheory.IsProjectiveMeasureFamily}
A family of measures $P$ indexed by finite sets of $T$ is projective if, for finite sets $J \subseteq I$, the projection from $E^I$ to $E^J$ maps $P_I$ to $P_J$.
\end{definition}


\begin{definition}[Projective limit]\label{def:IsProjectiveLimit}
  \uses{def:IsProjectiveMeasureFamily}
  \mathlibok
  \lean{MeasureTheory.IsProjectiveLimit}
A measure $\mu$ on $E^T$ is the projective limit of a projective family of measures $P$ indexed by finite sets of $T$ if, for every finite set $I \subseteq T$, the projection from $E^T$ to $E^I$ maps $\mu$ to $P_I$.
\end{definition}


\begin{theorem}[Kolmogorov extension theorem]\label{thm:kolmogorovExtension}
  \uses{def:IsProjectiveLimit, def:IsProjectiveMeasureFamily}
  \leanok
  \lean{MeasureTheory.projectiveLimit, MeasureTheory.IsProjectiveLimit.unique, MeasureTheory.isProjectiveLimit_projectiveLimit, MeasureTheory.isFiniteMeasure_projectiveLimit, MeasureTheory.isProbabilityMeasure_projectiveLimit}
Let $\mathcal{X}$ be a Polish space, equipped with the Borel $\sigma$-algebra, and let $T$ be an index set.
Let $P$ be a projective family of finite measures on $\mathcal{X}$.
Then the projective limit $\mu$ of $P$ exists, is unique, and is a finite measure on $\mathcal{X}^T$.
Moreover, if $P_I$ is a probability measure for every finite set $I \subseteq T$, then $\mu$ is a probability measure.
\end{theorem}

\begin{proof}\leanok

\end{proof}


\section{Projective family of Gaussian measures}

We build a projective family of Gaussian measures indexed by $\mathbb{R}_+$.
In order to do so, we need to define specific Gaussian measures on finite index sets $\{t_1, \ldots, t_n\}$.
We want to build a multivariate Gaussian measure on $\mathbb{R}^n$ with mean $0$ and covariance matrix $C_{ij} = \min(t_i, t_j)$ for $1 \leq i,j \leq n$.

% \paragraph{First method: Gaussian increments}

% In this method, we build the Gaussian measure by adding independent Gaussian increments.

% \begin{definition}(Gaussian increment)\label{def:gaussianIncrement}
% For $v \ge 0$, the map from $\mathbb{R}$ to the probability measures on $\mathbb{R}$ defined by $x \mapsto \mathcal{N}(x, v)$ is a Markov kernel.
% We call that kernel the \emph{Gaussian increment} with variance $v$ and denote it by $\kappa^G_v$.
% \end{definition}

% TODO: perhaps the equality $\mathcal{N}(x, v) = \delta_x \ast \mathcal{N}(0, v)$ is useful to show that it is a kernel?

% \begin{definition}\label{def:gaussianFromIncrements}
%   \uses{def:gaussianIncrement}
% Let $0 \le t_1 \le \ldots \le t_n$ be non-negative reals.
% Let $\mu_0$ be the real Gaussian distribution $\mathcal{N}(0, t_1)$.
% For $i \in \{1, \ldots, n-1\}$, let $\kappa_i$ be the Markov kernel from $\mathbb{R}$ to $\mathbb{R}$ defined by $\kappa_i(x) = \mathcal{N}(x, t_{i+1} - t_i)$ (the Gaussian increment $\kappa^G_{t_{i+1} - t_i}$).
% Let $P_{t_1, \ldots, t_n}$ be the measure on $\mathbb{R}^n$ defined by $\mu_0 \otimes \kappa_1 \otimes \ldots \otimes \kappa_{n-1}$.
% \end{definition}

% TODO: explain the notation $\otimes$ in the lemma above: $\kappa_{n-1}$ takes the value at $n-1$ only to produce the distribution at $n$.

% \begin{lemma}\label{lem:isGaussian_gaussianFromIncrements}
%   \uses{def:gaussianFromIncrements, def:IsGaussian}
% $P_{t_1, \ldots, t_n}$ is a Gaussian measure on $\mathbb{R}^n$ with mean $0$ and covariance matrix $C_{ij} = \min(t_i, t_j)$ for $1 \leq i,j \leq n$.
% \end{lemma}

% \begin{proof}

% \end{proof}


% \paragraph{Second method: covariance matrix}

We prove that the matrix $C_{ij} = \min(t_i, t_j)$ is positive semidefinite, which means that there exists a Gaussian distribution with mean 0 and covariance matrix $C$.

\begin{definition}[Gram matrix]\label{def:gramMatrix}
Let $v_1, \ldots, v_n$ be vectors in an inner product space $E$.
The Gram matrix of $v_1, \ldots, v_n$ is the matrix in $\mathbb{R}^{n \times n}$ with entries $G_{ij} = \langle v_i, v_j \rangle$ for $1 \leq i,j \leq n$.
\end{definition}


\begin{lemma}\label{lem:posSemidef_gramMatrix}
  \uses{def:gramMatrix}
A gram matrix is positive semidefinite.
\end{lemma}

\begin{proof}
Symmetry is obvious from the definition.
Let $x \in E$. Then
\begin{align*}
  \langle x, G x \rangle
  &= \sum_{i,j} x_i x_j \langle v_i, v_j \rangle
  \\
  &= \langle \sum_i x_i v_i, \sum_j x_j v_j \rangle
  \\
  &= \left\Vert \sum_i x_i v_i \right\Vert^2
  \\
  &\ge 0
  \: .
\end{align*}
\end{proof}


\begin{lemma}\label{lem:C_eq_gramMatrix}
  \uses{def:gramMatrix}
Let $I = \{t_1, \ldots, t_n\}$ be a finite subset of $\mathbb{R}_+$.
For $i \le n$, let $v_i = \mathbb{I}_{[0, t_i]}$ be the indicator function of the interval $[0, t_i]$, as an element of $L^2(\mathbb{R})$.
Then the Gram matrix of $v_1, \ldots, v_n$ is equal to the matrix $C_{ij} = \min(t_i, t_j)$ for $1 \leq i,j \leq n$.
\end{lemma}

\begin{proof}
By definition of the inner product in $L^2(\mathbb{R})$,
\begin{align*}
  \langle v_i, v_j \rangle
  &= \int_{\mathbb{R}} \mathbb{I}_{[0, t_i]}(x) \mathbb{I}_{[0, t_j]}(x) \: dx
  = \min(t_i, t_j)
  \: .
\end{align*}
\end{proof}


\begin{lemma}\label{lem:posSemidef_brownianCov}
For $I = \{t_1, \ldots, t_n\}$ a finite subset of $\mathbb{R}_+$, let $C \in \mathbb{R}^{n \times n}$ be the matrix $C_{ij} = \min(t_i, t_j)$ for $1 \leq i,j \leq n$.
Then $C$ is positive semidefinite.
\end{lemma}

\begin{proof}\uses{lem:C_eq_gramMatrix, lem:posSemidef_gramMatrix}
$C$ is a Gram matrix by Lemma~\ref{lem:C_eq_gramMatrix}.
By Lemma~\ref{lem:posSemidef_gramMatrix}, it is positive semidefinite.
\end{proof}


\paragraph{Definition of the projective family and extension}

\begin{definition}[Projective family of the Brownian motion]\label{def:gaussianProjectiveFamily}
  \uses{def:multivariateGaussian, lem:posSemidef_brownianCov}
For $I = \{t_1, \ldots, t_n\}$ a finite subset of $\mathbb{R}_+$, let $P^B_I$ be the multivariate Gaussian measure on $\mathbb{R}^n$ with mean $0$ and covariance matrix $C_{ij} = \min(t_i, t_j)$ for $1 \leq i,j \leq n$.
We call the family of measures $P^B_I$ the \emph{projective family of the Brownian motion}.
\end{definition}


\begin{lemma}\label{lem:isProjectiveMeasureFamily_gaussianProjectiveFamily}
  \uses{def:gaussianProjectiveFamily, def:IsProjectiveMeasureFamily}
The projective family of the Brownian motion is a projective family of measures.
\end{lemma}

\begin{proof}
  \uses{lem:isGaussian_map, lem:isGaussian_multivariateGaussian, lem:covMatrix_map}
Let $J \subseteq I$ be finite subsets of $\mathbb{R}_+$.
We need to show that the restriction from $\mathbb{R}^I$ to $\mathbb{R}^J$ (denote it by $\pi_{IJ}$) maps $P^B_I$ to $P^B_J$.

The restriction is a continuous linear map from $\mathbb{R}^I$ to $\mathbb{R}^J$.
The map of a Gaussian measure by a continuous linear map is Gaussian (Lemma~\ref{lem:isGaussian_map}).
It thus suffices to show that the mean and covariance matrix of the map are equal to the ones of $P^B_J$.

The mean of the map is $0$, since the mean of $P^B_I$ is $0$ and the map is linear.

For the covariance matrix and $i, j \in J$, by Lemma~\ref{lem:covMatrix_map} we have
\begin{align*}
  \langle e_i, \Sigma_{\pi_{IJ*}\mu} e_j\rangle
  &= \langle \pi_{IJ}^\dagger(e_i), \Sigma_\mu \pi_{IJ}^\dagger(e_j)\rangle
  \: .
\end{align*}
$\pi_{IJ}^\dagger(u)$ is the vector of $\mathbb{R}^I$ with coordinates $(\pi_{IJ}^\dagger(u))_i = u_i$ if $i \in J$ and $(\pi_{IJ}^\dagger(u))_i = 0$ otherwise.
This gives the same covariance matrix as the one of $P^B_J$.
\end{proof}


\begin{definition}\label{def:gaussianLimit}
  \uses{thm:kolmogorovExtension, lem:isProjectiveMeasureFamily_gaussianProjectiveFamily}
We denote by $P_B$ the projective limit of the projective family of the Brownian motion given by Theorem~\ref{thm:kolmogorovExtension}.
This is a probability measure on $\mathbb{R}^{\mathbb{R}_+}$.
\end{definition}

\chapter{Kolmogorov-Chentsov Theorem}
\label{chap:kolmogorov_chentsov}

We follow the proof of the Kolmogorov-Chentsov theorem from \cite{kratschmer2023kolmogorov}.
That proof notably uses the chaining technique developed by Talagrand \cite{talagrand2022upper}.

That theorem is about stochastic processes $X : T \to \Omega \to E$, where $\Omega$ is a measurable space with a probability measure $\mathbb{P}$, the index set $T$ is a metric space with distance $d_T$, and $E$ is also a metric space with distance $d_E$, on which we put the Borel $\sigma$-algebra.

The main result is Theorem~\ref{thm:countable_set_bound}.
Under an assumption on the covering number of $T$, for a process $X$ that satisfies the Kolmogorov condition $\mathbb{E}[d_E(X_s, X_t)^p] \le M d_T(s, t)^q$ (see Definition~\ref{def:IsKolmogorovProcess}),the theorem gives a finite bound on the expectation of the supremum of the ratio
\begin{align*}
  \mathbb{E}\left[ \sup_{s, t \in T'} \frac{d_E(X_s, X_t)^p}{d_T(s, t)^{\beta p}} \right]
  \: ,
\end{align*}
for $T'$ a countable subset of $T$.
As a corollary, we obtain that there exists a modification of $X$ with Hölder continuous paths.

In Lean, we will use the typeclass \texttt{PseudoEMetricSpace} for both $T$ and $E$ as long as possible, and then specialize to \texttt{EMetricSpace} (or perhaps even \texttt{MetricSpace}) when we need the stronger properties of a metric space.
For example, to prove the existence of a modification of a stochastic process, we will eventually use the fact that $d_E(x, y) = 0$ implies $x = y$, which does not hold in a pseudo-metric space.
All distances will be expressed with \texttt{edist}, which takes values in \texttt{ENNReal}, and the integrals refer to Lebesgue integrals.

\section{Covers and covering numbers}

Let $(E, d_E)$ be a pseudo-metric space.

\begin{definition}[$\varepsilon$-cover]\label{def:IsCover}
  \leanok
  \lean{IsCover}
  A set $C \subseteq E$ is an $\varepsilon$-cover of a set $A \subseteq E$ if for every $x \in A$, there exists $y \in C$ such that $d_E(x, y) \le \varepsilon$.
\end{definition}


\begin{definition}[External covering number]\label{def:externalCoveringNumber}
  \uses{def:IsCover}
  \leanok
  \lean{externalCoveringNumber}
  The external covering number of a set $A \subseteq E$ for $\varepsilon \ge 0$ is the smallest cardinality of an $\varepsilon$-cover of $A$.
  Denote it by $N^{ext}_\varepsilon(A)$.
\end{definition}


\begin{definition}[Internal covering number]\label{def:internalCoveringNumber}
  \uses{def:IsCover}
  \leanok
  \lean{internalCoveringNumber}
  The internal covering number of a set $A \subseteq E$ for $\varepsilon \ge 0$ is the smallest cardinality of an $\varepsilon$-cover of $A$ which is a subset of $A$.
  Denote it by $N^{int}_\varepsilon(A)$.
\end{definition}


\begin{lemma}\label{lem:externalCoveringNumber_le_internalCoveringNumber}
  \uses{def:externalCoveringNumber, def:internalCoveringNumber}
  \leanok
  \lean{externalCoveringNumber_le_internalCoveringNumber}
$N^{ext}_\varepsilon(A) \le N^{int}_\varepsilon(A)$.
\end{lemma}

\begin{proof}\leanok

\end{proof}


\begin{definition}[Bounded internal covering number]\label{def:HasBoundedInternalCoveringNumber}
  \uses{def:internalCoveringNumber}
  \leanok
  \lean{HasBoundedInternalCoveringNumber}
  Let $\mathrm{diam}(A)$ be the diameter of $A \subseteq E$, i.e. $\mathrm{diam}(A) = \sup_{x,y \in A} d_E(x, y)$.
  A set $A \subseteq E$ has bounded internal covering number with constant $c>0$ and exponent $t>0$ if for all $\varepsilon \in (0, \mathrm{diam}(A))$, $N^{int}_\varepsilon(A) \le c \varepsilon^{-t}$.
\end{definition}


\begin{lemma}\label{lem:hasBoundedInternalCoveringNumber_unitInterval}
  \uses{def:HasBoundedInternalCoveringNumber}
The unit interval $I = [0, 1] \subseteq \mathbb{R}$ has bounded internal covering number with constant $1$ and exponent $1$: for $\varepsilon \le 1$, $N^{int}_\varepsilon(I) \le 1/\varepsilon$.
\end{lemma}

\begin{proof}

\end{proof}


\section{Chaining}

\subsection{Chaining sequence}


\begin{definition}\label{def:nearestPt}
  \leanok
  \lean{nearestPt}
Let $S$ be a finite set of $E$ and $x \in E$.
We denote by $\pi(x, S)$ the point in $S$ which is closest to $x$, i.e. a point such that $d_E(x, S) = \min_{y \in S} d_E(x, y)$ (chosen arbitrarily among the minima if there are several).
\end{definition}


\begin{lemma}\label{lem:dist_nearestPt_le}
  \uses{def:nearestPt}
  \leanok
  \lean{edist_nearestPt_le}
Let $S$ be a finite set of $E$ and $x \in E$.
Then for all $y \in S$, $d_E(x, \pi(x, S)) \le d_E(x, y)$.
\end{lemma}

\begin{proof}\leanok
By definition.
\end{proof}


\begin{lemma}\label{lem:dist_nearestPt_of_isCover}
  \uses{def:nearestPt, def:IsCover}
  \leanok
  \lean{edist_nearestPt_of_isCover}
Let $C_\varepsilon$ be a finite $\varepsilon$-cover of $A \subseteq E$ (assuming such a finite cover exists).
Then for all $x \in A$, $d_E(x, \pi(x, C_\varepsilon)) \le \varepsilon$.
\end{lemma}

\begin{proof}\leanok

\end{proof}


\begin{definition}[Chaining sequence]\label{def:chainingSequence}
  \uses{def:nearestPt, def:IsCover}
  \leanok
  \lean{chainingSequence}
Let $(\varepsilon_n)_{n \in \mathbb{N}}$ be a sequence of positive numbers, $C_n$ a finite $\varepsilon_n$-cover of $A \subseteq E$ with $C_n \subseteq A$ and $x \in C_k$ for some $k \in \mathbb{N}$.
We define the chaining sequence of $x$, denoted $(\bar{x}_i)_{i \le k}$, recursively as follows: $\bar{x}_k = x$ and for $i < k$, $\bar{x}_i = \pi(\bar{x}_{i+1}, C_i)$.
\end{definition}


\begin{lemma}\label{lem:chainingSequence_mem}
  \uses{def:chainingSequence}
  \leanok
  \lean{chainingSequence_mem}
Let $(\varepsilon_n)_{n \in \mathbb{N}}$ be a sequence of positive numbers, $C_n$ a finite $\varepsilon_n$-cover of $A \subseteq E$ with $C_n \subseteq A$ and $x \in C_k$ for some $k \in \mathbb{N}$.
Then for all $i \le k$, $\bar{x}_i\in C_i$.
\end{lemma}

\begin{proof}\leanok
By definition.
\end{proof}


\begin{lemma}\label{lem:dist_chainingSequence_add_one}
  \uses{def:chainingSequence}
  \leanok
  \lean{edist_chainingSequence_add_one}
Let $(\varepsilon_n)_{n \in \mathbb{N}}$ be a sequence of positive numbers, $C_n$ a finite $\varepsilon_n$-cover of $A \subseteq E$ with $C_n \subseteq A$ and $x \in C_k$ for some $k \in \mathbb{N}$.
Then for all $i < k$, $d_E(\bar{x}_i, \bar{x}_{i+1}) \le \varepsilon_i$.
\end{lemma}

\begin{proof}\leanok
  \uses{lem:dist_nearestPt_of_isCover, lem:chainingSequence_mem}
Apply Lemma~\ref{lem:dist_nearestPt_of_isCover} with $S = C_i$ and $x = \bar{x}_{i+1}$.
\end{proof}


\begin{lemma}\label{lem:dist_chainingSequence_le_sum}
  \uses{def:chainingSequence}
  \leanok
  \lean{edist_chainingSequence_le_sum}
Let $(\varepsilon_n)_{n \in \mathbb{N}}$ be a sequence of positive numbers, $C_n$ a finite $\varepsilon_n$-cover of $A \subseteq E$ with $C_n \subseteq A$ and $x \in C_k$ for some $k \in \mathbb{N}$.
Then for $m \le k$, $d_E(\bar{x}_m, x) \le \sum_{i=m}^{k-1} \varepsilon_i$.
\end{lemma}

\begin{proof}\leanok
  \uses{lem:dist_chainingSequence_add_one}
By the triangle inequality and Lemma~\ref{lem:dist_chainingSequence_add_one},
\begin{align*}
  d_E(\bar{x}_m, x)
  \le \sum_{i=m}^{k-1} d_E(\bar{x}_i, \bar{x}_{i+1})
  \le \sum_{i=m}^{k-1} \varepsilon_i
  \: .
\end{align*}
\end{proof}


\begin{lemma}\label{lem:dist_chainingSequence_le}
  \uses{def:chainingSequence}
  \leanok
  \lean{edist_chainingSequence_le}
Let $(\varepsilon_n)_{n \in \mathbb{N}}$ be a sequence of positive numbers, $C_n$ a finite $\varepsilon_n$-cover of $A \subseteq E$ with $C_n \subseteq A$.
Let $m, k, \ell \in \mathbb{N}$ with $m \le k$ and $m \le \ell$ and let $x \in C_k$ and $y \in C_\ell$.
Then
\begin{align*}
  d_E(\bar{x}_m, \bar{y}_m)
  &\le d_E(x, y) + \sum_{i=m}^{k-1} \varepsilon_i + \sum_{j=m}^{\ell-1} \varepsilon_j
\end{align*}
\end{lemma}

\begin{proof}\leanok
  \uses{lem:dist_chainingSequence_le_sum}
Triangle inequality and Lemma~\ref{lem:dist_chainingSequence_le_sum}.
\end{proof}


\begin{corollary}\label{cor:dist_chainingSequence_pow_two_le}
  \uses{def:chainingSequence}
  \leanok
  \lean{edist_chainingSequence_pow_two_le}
For $\varepsilon_n = \varepsilon_0 2^{-n}$, with the hypothesis of Lemma~\ref{lem:dist_chainingSequence_le}, we have
\begin{align*}
  d_E(\bar{x}_m, \bar{y}_m)
  &\le d_E(x, y) + \varepsilon_0 2^{-m+2}
  \: .
\end{align*}
\end{corollary}

\begin{proof}\leanok
  \uses{lem:dist_chainingSequence_le}

\end{proof}


\subsection{A subset of pairs}

We will be interested in bounding expressions of the form $\sup_{s,t\in J, d_T(s,t) \le c} d_E(f(s), f(t))$ for a finite set $J$ and some function $f : T \to E$.
This is a supremum over pairs in $J$ and there could be $\vert J \vert^2$ such pairs.
We will build a subset $K$ of $J^2$ which is much smaller, of size linear in $\vert J \vert$, such that its points are not too far apart and
\begin{align*}
  \sup_{s,t\in J, d_T(s,t) \le c} d_E(f(s), f(t))
  & \le 2 \sup_{(s,t) \in K} d_E(f(s), f(t))
  \: .
\end{align*}
The pairs $(s, t) \in K$ will still be close together, in the sense that $d_T(s, t) \le c n$ for some $n$ that is logarithmic in the size of $J$.

For $t \in V \subseteq T$ and $u\ge 0$, we denote by $B_V(t, u)$ the closed ball with center $t$ and radius $u$ in $V$.
That is, $B_V(t, u) = \{s \in V \mid d_T(s, t) \le u\}$.

We want to cover $J$ with balls that have radius logarithmic in the number of points of the ball.

\begin{definition}\label{def:logSizeRadius}
  \leanok
  \lean{logSizeRadius}
Let $V$ be a finite subset of a metric space and let $t \in V$ and $a > 1$, $c > 0$.
Let the \emph{log-size radius} of $t$ in $V$, denoted by $r_{V,t}$, be the smallest positive integer $r$ such that $\vert B_V(t, r c) \vert \le a^{r}$.
\end{definition}


\begin{lemma}\label{lem:card_logSizeRadius_ge}
  \uses{def:logSizeRadius}
  \leanok
  \lean{card_le_logSizeRadius_ge}
$a^{r_{V,t}-1} \le \vert B_V(t, (r_{V,t}-1)c) \vert$~.
\end{lemma}

\begin{proof}\leanok

\end{proof}


\begin{lemma}\label{lem:card_logSizeRadius_le}
  \uses{def:logSizeRadius}
  \leanok
  \lean{card_le_logSizeRadius_le}
$\vert B_V(t, r_{V,t}c) \vert \le a^{r_{V,t}}$~.
\end{lemma}

\begin{proof}\leanok

\end{proof}


\begin{definition}[Log-size ball sequence]\label{def:logSizeBallSequence}
  \uses{def:logSizeRadius}
  \leanok
  \lean{logSizeBallSeq}
Let $(T,d_T)$ be a metric space and let $J \subseteq T$ be finite, $a,c \in \mathbb R_+$ with $a \ge 1$ and $n \in \{1, 2, ...\}$ such that $|J| \le a^n$.
An log-size ball sequence for $(J, a, c, n)$ is a sequence of $(V_i, t_i, r_i)_{i \in \mathbb{N}}$ such that
\begin{itemize}
  \item $V_0 = J$, $t_0$ is an arbitrary point in $J$,
  \item for all $i$, $r_i$ is the log-size radius of $t_i$ in $V_i$,
  \item $V_{i+1} = V_i \setminus B_{V_i}(t_i, (r_i - 1)c)$, $t_{i+1}$ is arbitrarily chosen in $V_{i+1}$.
\end{itemize}
\end{definition}

A log-size ball sequence gives a partition of $J$ into sets which are contained in balls of radius $(r_i - 1)c$ around $t_i$, and satisfy cardinality constraints.


\begin{lemma}\label{lem:logSizeRadius_logSizeBallSequence_le}
  \uses{def:logSizeBallSequence}
  \leanok
  \lean{radius_logSizeBallSeq_le}
The radius of a log-size ball sequence $(V_i, t_i, r_i)_{i \in \mathbb{N}}$ for $(J, a, c, n)$ satisfies $r_i \le n$ for all $i \in \mathbb{N}$.
\end{lemma}

\begin{proof}
Since $|J| \le a^n$, we have $\vert B_{V_i}(t_i, n c) \vert \le \vert J \vert \le a^{n}$.
\end{proof}


\begin{lemma}\label{lem:logSizeBallSequence_V_anti}
  \uses{def:logSizeBallSequence}
  \leanok
  \lean{finset_logSizeBallSeq_add_one_subset}
The sets $V_i$ of a log-size ball sequence $(V_i, t_i, r_i)_{i \in \mathbb{N}}$ are a decreasing sequence of sets. That is, $V_{i+1} \subseteq V_i$ for all $i \in \mathbb{N}$.
\end{lemma}

\begin{proof}
$V_{i+1} = V_i \setminus B_{V_i}(t_i, (r_i - 1)c)$ hence $V_{i+1} \subseteq V_i$.
\end{proof}


\begin{lemma}\label{lem:logSizeBallSequence_eq_zero}
  \uses{def:logSizeBallSequence}
  \leanok
  \lean{card_finset_logSizeBallSeq_card_eq_zero}
For any log-size ball sequence $(V_i, t_i, r_i)_{i \in \mathbb{N}}$ for $(J, a, c, n)$, for all $k \ge \vert J \vert$, $V_k = \emptyset$.
\end{lemma}

\begin{proof}
  \uses{lem:logSizeBallSequence_V_anti}
$V_{i+1} = V_i \setminus B_{V_i}(t_i, (r_i - 1)c)$ and since $t_i \in B_{V_i}(t_i, (r_i - 1)c)$, we have $\vert V_{i+1} \vert < \vert V_i \vert$ and the cardinal eventually reaches $0$, in at most $\vert J \vert$ steps.
\end{proof}


\begin{lemma}\label{lem:logSizeBallSequence_disjoint_B}
  \uses{def:logSizeBallSequence}
  \leanok
  \lean{disjoint_smallBall_logSizeBallSeq}
For $i \ne j$, the balls $B_{V_i}(t, (r_i-1)c)$ and $B_{V_j}(t_j, (r_j-1)c)$ of a log-size ball sequence $(V_i, t_i, r_i)_{i \in \mathbb{N}}$ are disjoint.
\end{lemma}

\begin{proof}
  \uses{lem:logSizeBallSequence_V_anti}
Assume w.l.o.g. that $i < j$.
Then $B_{V_j}(t_j, (r_j-1)c) \subseteq V_j \subseteq V_{i+1}$.
It suffices to show that $B_{V_i}(t_i, (r_i-1)c)$ and $V_{i+1}$ are disjoint.
This follows from the definition of $V_{i+1} = V_i \setminus B_{V_i}(t_i, (r_i-1)c)$.
\end{proof}


\begin{definition}\label{def:pairSet}
  \uses{def:logSizeBallSequence}
  \leanok
  \lean{pairSet}
Let $(V_i, t_i, r_i)_{i \in \mathbb{N}}$ be a log-size ball sequence for $(J, a, c, n)$.
For $i \in \mathbb{N}$, let $K_i = \{t_i\} \times B_{V_i}(t_i, r_i c)$ be the set of pairs $(t_i, s)$ for $s$ in the ball $B_{V_i}(t_i, r_i c)$.
We define $K = \bigcup_{i=0}^{\vert J \vert-1} K_i$, set of all pairs from the log-size ball sequence.
\end{definition}


\begin{lemma}\label{lem:card_pairSet_le}
  \uses{def:pairSet}
  \leanok
  \lean{card_pairSet_le}
The cardinal of the pair set $K$ of a log-size ball sequence for $(J, a, c, n)$ satisfies $|K| \le a |J|$.
\end{lemma}

\begin{proof}
  \uses{lem:card_logSizeRadius_ge, lem:card_logSizeRadius_le, lem:logSizeBallSequence_disjoint_B}
Using Lemma~\ref{lem:card_logSizeRadius_le}, the cardinal of $K$ is bounded by
\begin{align*}
  \vert K \vert
  &\le \sum_{i=0}^{m-1} \vert K_i \vert
  \le \sum_{i=0}^{m-1} a^{r_i}
  \: .
\end{align*}
Since the sets $B_{V_i}(t_i, (r_i-1)c)$ are disjoint by Lemma~\ref{lem:logSizeBallSequence_disjoint_B}, we can use Lemma~\ref{lem:card_logSizeRadius_ge} to get
\begin{align*}
  \sum_{i=0}^{m-1} a^{r_i - 1}
  \le \sum_{i=0}^{m-1} \vert B_{V_i}(t_i, (r_i-1)c) \vert
  = \left\vert \bigcup_{i=0}^{m-1} B_{V_i}(t_i, (r_i-1)c) \right\vert
  \le \vert J \vert
  \: .
\end{align*}
We obtained the inequality $\vert K \vert \le a \vert J \vert$
\end{proof}


\begin{lemma}\label{lem:dist_le_of_mem_pairSet}
  \uses{def:pairSet}
  \leanok
  \lean{edist_le_of_mem_pairSet}
Let $(s, t)$ be a pair in the pair set $K$ of a log-size ball sequence for $(J, a, c, n)$.
Then $d_T(s, t) \le c n$.
\end{lemma}

\begin{proof}
  \uses{lem:logSizeRadius_logSizeBallSequence_le}
A pair $(t, s) \in K$ is of the form $(t_i, s)$ for $s \in B_V(t_i, r_i c)$ and satisfies
\begin{align*}
  d_T(t_i, s) \le c r_i \le c n \: .
\end{align*}
The last inequality is from Lemma~\ref{lem:logSizeRadius_logSizeBallSequence_le}.
\end{proof}


\begin{lemma}\label{lem:sup_dist_le_two_mul_sup_dist_pairSet}
  \uses{def:pairSet}
  \leanok
  \lean{iSup_edist_pairSet}
Let $K$ be the pair set of a log-size ball sequence $(V_i, t_i, r_i)_{i \in \mathbb{N}}$ for $(J, a, c, n)$.
Then for any function $f : T \to E$ with $(E,d_E)$ a metric space,
\begin{align*}
  \sup_{s,t\in J, d_T(s,t) \le c} d_E(f(s), f(t))
  & \le 2 \sup_{(s,t) \in K} d_E(f(s), f(t))
  \: .
\end{align*}
\end{lemma}

\begin{proof}
Let $(s, t) \in J^2$ such that $d_T(s, t) \le c$.
Then there exists a largest $\ell \in \mathbb{N}$ such that $s, t \in V_\ell$.
Assume w.l.o.g. that $s \notin V_{\ell + 1}$. Then $s \in B_{V_\ell}(t_\ell, (r_\ell-1)c)$ (since $V_{\ell + 1} = V_\ell \setminus B_{V_\ell}(t_\ell, (r_\ell-1)c)$), which implies $d_T(s, t_\ell) \le (r_\ell - 1)c$.

Since $d_T(s, t) \le c$, $d_T(t, t_\ell) \le d_T(t, s) + d_T(s, t_\ell) \le r_\ell c$, hence $t \in B_{V_\ell}(t_\ell, r_\ell c)$ and we have that both $s$ and $t$ are in $B_{V_\ell}(t_\ell, r_\ell c)$.
Thus both $(t_\ell, s)$ and $(t_\ell, t)$ are in $K_\ell \subseteq K$.
Finally
\begin{align*}
  d_E(f(s), f(t))
  &\le d_E(f(s), f(t_\ell)) + d_E(f(t_\ell), f(t))
  \\
  &\le 2\sup_{(s',t') \in K} d_E(f(s'), f(t'))
  \: .
\end{align*}
\end{proof}


\begin{lemma}\label{lem:pair_reduction}
  \uses{def:pairSet}
  \leanok
  \lean{pair_reduction}
Let $(T,d_T)$ be a metric space.
Let $J \subseteq T$ be finite, $a > 1$, $c>0$ and $n \in \{1, 2, ...\}$ such that $|J| \le a^n$.
Then, there is $K \subseteq J^2$ such that for any function $f : T \to E$ with $(E,d_E)$ a metric space,
\begin{align}
  |K|
  & \le a |J|
  \:, \label{eq:chain1} \\
  \forall (s,t) \in K,
  &\:  d_T(s,t) \le c n
  \:, \label{eq:chain2} \\
  \sup_{s,t\in J, d_T(s,t) \le c} d_E(f(s), f(t))
  & \le 2 \sup_{(s,t) \in K} d_E(f(s), f(t))
  \: . \label{eq:chain3}
\end{align}
\end{lemma}

\begin{proof}\leanok
  \uses{lem:card_pairSet_le, lem:dist_le_of_mem_pairSet, lem:sup_dist_le_two_mul_sup_dist_pairSet}
Let $(V_i, t_i, r_i)_{i \in \mathbb{N}}$ be a log-size ball sequence for $(J, a, c, n)$. We show that its pair set satisfies the conditions of the lemma.

Equation~\eqref{eq:chain1} is given by Lemma~\ref{lem:card_pairSet_le}.
The second property~\eqref{eq:chain2} is Lemma~\ref{lem:dist_le_of_mem_pairSet}.
Equation~\eqref{eq:chain3} was proved in Lemma~\ref{lem:sup_dist_le_two_mul_sup_dist_pairSet}.
\end{proof}





\section{Chaining for stochastic processes under Kolmogorov conditions}

\subsection{Kolmogorov condition}

\begin{definition}[Kolmogorov condition]\label{def:IsKolmogorovProcess}
  \leanok
  \lean{ProbabilityTheory.IsKolmogorovProcess}
Let $X : T \to \Omega \to E$ be a stochastic process, where $(T, d_T)$ and $(E, d_E)$ are pseudo-metric spaces and $(\Omega, \mathbb{P})$ is a measure space.
Let $p, q > 0$.
We say that $X$ satisfies the Kolmogorov condition for exponents $(p,q)$ with constant $M$ if for all $s, t \in T$, $(X_s, X_t)$ is $\mathbb{P}$-a.e. measurable for the Borel $\sigma$-algebra on $E^2$ and
\begin{align*}
  \mathbb{E}[d_E(X_s, X_t)^p] \le M d_T(s, t)^q
  \: .
\end{align*}
\end{definition}

Remark: the measurability condition on the pair would be implied by the measurability of $X_t$ for all $t \in T$ if we assumed that $E$ is separable (\texttt{SecondCountableTopology} in Lean), which implies that the Borel $\sigma$-algebra on the product is equal to the product of the Borel $\sigma$-algebras.
We follow \cite{kratschmer2023kolmogorov} and do not require separability.

\paragraph{Measurability}

\begin{lemma}\label{lem:IsKolmogorovProcess.aemeasurable}
  \uses{def:IsKolmogorovProcess}
  \leanok
  \lean{ProbabilityTheory.IsKolmogorovProcess.aemeasurable}
If $X : T \to \Omega \to E$ is a function that satisfies the Kolmogorov condition, then for all $t \in T$, $X_t$ is $\mathbb{P}$-a.e. measurable.
\end{lemma}

\begin{proof}\leanok

\end{proof}


\begin{lemma}\label{lem:aemeasurable_pair_of_aemeasurable}
  \leanok
  \lean{ProbabilityTheory.aemeasurable_pair_of_aemeasurable}
If $E$ is separable and $X : T \to \Omega \to E$ is a process such that $X_t$ is $\mathbb{P}$-a.e. measurable for all $t \in T$, then for all $s, t \in T$, the pair $(X_s, X_t)$ is $\mathbb{P}$-a.e. measurable for the Borel $\sigma$-algebra on $E^2$.
\end{lemma}

\begin{proof}\leanok

\end{proof}


\begin{lemma}\label{lem:IsKolmogorovProcess.aemeasurable_edist}
  \uses{def:IsKolmogorovProcess}
  \leanok
  \lean{ProbabilityTheory.IsKolmogorovProcess.aemeasurable_edist}
If $X : T \to \Omega \to E$ is a process that satisfies the Kolmogorov condition, then for all $s,t \in T$ the function $\omega \mapsto d_E(X_s(\omega), X_t(\omega))$ is $\mathbb{P}$-a.e. measurable.
\end{lemma}

\begin{proof}\leanok

\end{proof}

\paragraph{Distance bounds}

\begin{lemma}\label{lem:integral_sup_rpow_dist_le_card_mul_rpow}
  \uses{def:IsKolmogorovProcess}
  \leanok
  \lean{ProbabilityTheory.lintegral_sup_rpow_edist_le_card_mul_rpow}
Let $X : T \to \Omega \to E$ be a process that satisfies the Kolmogorov condition for exponents $(p,q)$ with constant $M$.
Let $\varepsilon > 0$ and $C \subseteq T^2$ be a finite set such that for all $(s, t) \in C$, $d_T(s, t) \le \varepsilon$.
Then
\begin{align*}
  \mathbb{E}\left[\sup_{(s,t) \in C} d_E(X_s, X_t)^p \right]
  &\le \vert C \vert M \varepsilon^q
  \: .
\end{align*}
\end{lemma}

\begin{proof}
  \uses{def:IsKolmogorovProcess}
\begin{align*}
  \mathbb{E}\left[\sup_{(s,t) \in C} d_E(X_s, X_t)^p \right]
  &\le \mathbb{E}\left[\sum_{(s,t) \in C} d_E(X_s, X_t)^p \right]
  \\
  &\le M \sum_{(s,t) \in C} d_T(s, t)^q
  \\
  &\le \vert C \vert M \varepsilon^q
  \: .
\end{align*}
\end{proof}


\begin{lemma}\label{lem:integral_sup_rpow_dist_of_dist_le}
  \uses{def:IsKolmogorovProcess}
  \leanok
  \lean{ProbabilityTheory.lintegral_sup_rpow_edist_le_card_mul_rpow_of_dist_le}
Let $X : T \to \Omega \to E$ be a process that satisfies the Kolmogorov condition for exponents $(p,q)$ with constant $M$.
Let $J \subseteq T$ be finite, $a, c \in \mathbb R_+$ with $a \ge 1$ and $n \in \{1, 2, ...\}$ such that $|J| \le a^n$.
Then
\begin{align*}
  \mathbb{E} \left[ \sup_{s, t \in J; d_T(s, t) \le c} d_E(X_s, X_t)^p \right]
  &\le 2^p a |J| M (cn)^q
  \: .
\end{align*}
\end{lemma}

\begin{proof}
  \uses{lem:pair_reduction, lem:integral_sup_rpow_dist_le_card_mul_rpow}
By Lemma~\ref{lem:pair_reduction}, there exists $K \subseteq J^2$ such that
\begin{align*}
  |K|
  & \le a |J|
  \:, \\
  \forall (s,t) \in K,
  & \ d_T(s,t) \le c n
  \:, \\
  \sup_{s,t\in J, d_T(s,t) \le c} d_E(X_s, X_t)
  & \le 2 \sup_{(s,t) \in K} d_E(X_s, X_t)
  \: .
\end{align*}
Hence for such a set $K$,
\begin{align*}
  \mathbb{E} \left[ \sup_{s, t \in J; d_T(s, t) \le c} d_E(X_s, X_t)^p \right]
  &\le 2^p \mathbb{E} \left[ \sup_{(s, t) \in K} d_E(X_s, X_t)^p \right]
  \: .
\end{align*}
Then by Lemma~\ref{lem:integral_sup_rpow_dist_le_card_mul_rpow},
\begin{align*}
  \mathbb{E} \left[ \sup_{(s, t) \in K} d_E(X_s, X_t)^p \right]
  &\le |K| M (cn)^q
  \le a |J| M (cn)^q
  \: .
\end{align*}
\end{proof}


\subsection{Bound for a set of points that are close together}

For a finite index set $T$, we want to obtain a bound on
\begin{align*}
  \mathbb{E}\left[ \sup_{s, t \in T; d_T(s, t) \le \delta} d_E(X_s, X_t)^p \right] \: .
\end{align*}
Note the condition that the supremum is taken over pairs $(s, t)$ such that $d_T(s, t) \le \delta$.

We consider covers of $T$ at different scales. $C_n$ is a finite $\varepsilon_n$-cover of $T$ with $\varepsilon_n = \varepsilon_0 2^{-n}$.
$T$ is equal to $C_k$ for some $k$ large enough, so the supremum over $T$ is a supremum at that scale $k$.
We will change scale to some $m \le k$ that depends on the distance bound $\delta$ ($m$ is of order $\log_2 \delta$) and consider the supremum over $C_m$ (plus a term due to the scale change).
Then for the supremum over a set in $C_m$, we use the reduction in the number of pairs of Lemma~\ref{lem:pair_reduction}.

\begin{lemma}\label{lem:scale_change}
  \uses{def:chainingSequence}
  \leanok
  \lean{scale_change}
Let $X : T \to E$.
Let $(\varepsilon_n)_{n \in \mathbb{N}}$ be a sequence of positive numbers, $C_n$ a finite $\varepsilon_n$-cover of $J \subseteq T$ with $C_n \subseteq J$.
For $m \le k$,
\begin{align*}
  \sup_{s, t \in C_k; d_T(s, t) \le \delta} d_E(X_s, X_t)
  &\le \sup_{s, t \in C_k; d_T(s, t) \le \delta} d_E(X_{\bar{s}_m}, X_{\bar{t}_m})
    + 2 \sup_{s \in C_k} d_E(X_s, X_{\bar{s}_m})
  \: .
\end{align*}
\end{lemma}

\begin{proof}
By the triangle inequality,
\begin{align*}
  d_E(X_s, X_t)
  &\le d_E(X_s, X_{\bar{s}_m}) + d(X_{\bar{s}_m}, X_{\bar{t}_m}) + d_E(X_{\bar{t}_m}, X_t)
  \: .
\end{align*}
\end{proof}


\begin{corollary}\label{cor:scale_change_rpow}
  \uses{def:chainingSequence}
  \leanok
  \lean{scale_change_rpow}
Let $X : T \to E$.
Let $(\varepsilon_n)_{n \in \mathbb{N}}$ be a sequence of positive numbers, $C_n$ a finite $\varepsilon_n$-cover of $J \subseteq T$ with $C_n \subseteq J$.
For $m \le k$,
\begin{align*}
  \sup_{s, t \in C_k; d_T(s, t) \le \delta} d_E(X_s, X_t)^p
  &\le 2^p \sup_{s, t \in C_k; d_T(s, t) \le \delta} d_E(X_{\bar{s}_m}, X_{\bar{t}_m})^p
    + 4^p \sup_{s \in C_k} d_E(X_s, X_{\bar{s}_m})^p
  \: .
\end{align*}
\end{corollary}

\begin{proof}
  \uses{lem:scale_change}
This is Lemma~\ref{lem:scale_change}, together with the fact that for $a, b \ge 0$,
\begin{align*}
  (a + b)^p \le (2\max(a,b))^p = 2^p \max(a^p,b^p) \le 2^p (a^p + b^p)
  \: .
\end{align*}
\end{proof}



\subsubsection{First term}


\begin{lemma}\label{lem:integral_sup_rpow_dist_cover_of_dist_le}
  \uses{def:IsKolmogorovProcess}
  \leanok
  \lean{ProbabilityTheory.lintegral_sup_rpow_edist_cover_of_dist_le}
Let $X : T \to \Omega \to E$ be a process that satisfies the Kolmogorov condition for exponents $(p,q)$ with constant $M$.
Let $C$ be a finite $\varepsilon$-cover of $J \subseteq T$ with $C \subseteq J$, with minimal cardinal.
Then for $c \ge 0$,
\begin{align*}
  \mathbb{E} \left[ \sup_{s, t \in C; d_T(s, t) \le c} d_E(X_s, X_t)^p \right]
  &\le 2^{p+2} M \left(2 c \log_2 N^{int}_{\varepsilon}(J) \right)^q  N^{int}_{\varepsilon}(J)
  \: .
\end{align*}
Note the logarithm has base $2$.
\end{lemma}

\begin{proof}
  \uses{lem:integral_sup_rpow_dist_of_dist_le}
Let $\bar{r} = \min\{r \in \mathbb{N} \mid N^{int}_{\varepsilon}(J) \le 2^r\}$.
\begin{align*}
  \vert C \vert
  = N^{int}_{\varepsilon}(J)
  \le 2^{\bar{r}}
  \: .
\end{align*}
By Lemma~\ref{lem:integral_sup_rpow_dist_of_dist_le} with $J = C$, $a = 2$, $b = 1$, $c = c$, $n = \bar{r}$,
\begin{align*}
  \mathbb{E} \left[ \sup_{s, t \in C; d_T(s, t) \le c} d_E(X_s, X_t)^p \right]
  &\le 2^{p+1} |C_m| M (c \bar{r})^q
  \le 2^{p+1} M (c \bar{r})^q  2^{\bar{r}}
  \: .
\end{align*}

By definition, $2^{\bar{r}} \le 2 N^{int}_{\varepsilon}(J)$ and $\bar{r} \le 1 + \log_2 N^{int}_{\varepsilon}(J)$.
Suppose $N^{int}_{\varepsilon}(J) \ge 2$ (if it equals one the result is trivial).
Then $\bar{r} \le 2 \log_2 N^{int}_{\varepsilon}(J)$.
\begin{align*}
  \mathbb{E} \left[ \sup_{s, t \in C; d_T(s, t) \le c} d_E(X_s, X_t)^p \right]
  &\le 2^{p+2} M \left(2 c \log_2 N^{int}_{\varepsilon}(J) \right)^q  N^{int}_{\varepsilon}(J)
  \: .
\end{align*}
\end{proof}


\begin{lemma}\label{lem:integral_sup_rpow_dist_cover_rescale}
  \uses{def:IsKolmogorovProcess, def:chainingSequence}
  \leanok
  \lean{ProbabilityTheory.lintegral_sup_rpow_edist_cover_rescale}
Let $X : T \to \Omega \to E$ be a process that satisfies the Kolmogorov condition for exponents $(p,q)$ with constant $M$.
For all $n \in \mathbb{N}$, let $C_n$ a finite $\varepsilon_n$-cover of $J \subseteq T$ with $C_n \subseteq J$ for $\varepsilon_n = \varepsilon_0 2^{-n}$, with minimal cardinal.
Let $\delta > 0$ and let $n_2$ be such that $\varepsilon_0 2^{-n_2} < \delta/2 \le \varepsilon_0 2^{-n_2+1}$.
Let $k = \min \{j \in \mathbb{Z} \mid \varepsilon_0 2^{-j} < \inf_{s, t \in J; s \ne t}d_T(s, t)\}$.
For $m = \min(n_2, k)$,
\begin{align*}
  \mathbb{E} \left[ \sup_{s, t \in C_k; d_T(s, t) \le \delta} d_E(X_{\bar{s}_m}, X_{\bar{t}_m})^p \right]
  &\le 2^{p+2} M \left(16 \delta \log_2 N^{int}_{\delta/4}(J) \right)^q  N^{int}_{\delta/4}(J)
  \: .
\end{align*}
TODO: $C_k = J$
\end{lemma}

\begin{proof}
  \uses{lem:integral_sup_rpow_dist_cover_of_dist_le, cor:dist_chainingSequence_pow_two_le}
If $\delta \le \varepsilon_0 2^{-k}$, then $\{(s, t) \in C_k; d_T(s, t) \le \delta\} = \{(s, s) \mid s \in C_k\}$ and the inequality holds trivially.
We can thus assume $\delta > \varepsilon_0 2^{-k}$.

For $s, t \in C_k$ with $d_T(s, t) \le \delta$, $d_T(\bar{s}_m, \bar{t}_m) \le \delta + \varepsilon_0 2^{-m+2} \le \varepsilon_0 2^{-m+3}$ (Corollary~\ref{cor:dist_chainingSequence_pow_two_le}).
It thus suffices to get a bound on $\mathbb{E} \left[ \sup_{s, t \in C_m; d_T(s, t) \le \varepsilon_0 2^{-m+3}} d_E(X_s, X_t)^p \right]$.

We can apply Lemma~\ref{lem:integral_sup_rpow_dist_cover_of_dist_le} with $\varepsilon = \varepsilon_m$, $c = \varepsilon_0 2^{-m+3}$. We obtain
\begin{align*}
  \mathbb{E} \left[ \sup_{s, t \in C_m; d_T(s, t) \le \varepsilon_0 2^{-m+3}} d_E(X_s, X_t)^p \right]
  &\le 2^{p+2} M \left(16 \varepsilon_0 2^{-m} \log_2 N^{int}_{\varepsilon_m}(J) \right)^q  N^{int}_{\varepsilon_m}(J)
  \: .
\end{align*}
By definition of $m$ and $n_2$, $\varepsilon_m = \varepsilon_0 2^{-m} \ge \varepsilon_0 2^{-n_2} \ge \delta/4$,
hence $N^{int}_{\varepsilon_m}(J) \le N^{int}_{\delta / 4}(J)$.

If $m = n_2$ then $\varepsilon_0 2^{-m} = \varepsilon_0 2^{-n_2} < \delta/2$.
Otherwise, $m = k$ and $\varepsilon_0 2^{-m} = \varepsilon_0 2^{-k} < \delta$ as argued at the start of the proof.
We thus get $\varepsilon_0 2^{-m} \le \delta$.
\end{proof}



\subsubsection{Second term}


\begin{lemma}\label{lem:integral_sup_rpow_dist_succ}
  \uses{def:IsKolmogorovProcess}
  \leanok
  \lean{ProbabilityTheory.lintegral_sup_rpow_edist_succ}
Let $X : T \to \Omega \to E$ be a process that satisfies the Kolmogorov condition for exponents $(p,q)$ with constant $M$.
Let $(\varepsilon_n)_{n \in \mathbb{N}}$ be a sequence of positive numbers and $C_n$ a finite $\varepsilon_n$-cover of $T$ with $C_n \subseteq T$.
Then for $j < k$,
\begin{align*}
  \mathbb{E}\left[\sup_{t \in C_k} d_E(X_{\bar{t}_j}, X_{\bar{t}_{j+1}})^p \right]
  &\le \vert C_{j+1} \vert M \varepsilon_j^q
  \: .
\end{align*}
\end{lemma}

\begin{proof}
  \uses{lem:dist_chainingSequence_add_one, lem:integral_sup_rpow_dist_le_card_mul_rpow}
\begin{align*}
  \mathbb{E}\left[\sup_{t \in C_k} d_E(X_{\bar{t}_j}, X_{\bar{t}_{j+1}})^p \right]
  &\le \mathbb{E}\left[\sum_{u \in C_{j+1}} d_E(X_{\bar{u}_j}, X_{u})^p \right]
  \: .
\end{align*}
We then apply Lemma~\ref{lem:integral_sup_rpow_dist_le_card_mul_rpow} to the set $C = \{(\bar{u}_j, u) \mid u \in C_{j+1}\}$, which satisfies the condition $d_T(\bar{u}_j, u) \le \varepsilon_j$ and has cardinal $\vert C_{j+1} \vert$.
\end{proof}


\begin{lemma}\label{lem:integral_sup_dist_le_sum_rpow}
  \uses{def:chainingSequence}
  \leanok
  \lean{ProbabilityTheory.lintegral_sup_rpow_edist_le_sum_rpow}
Let $X : T \to \Omega \to E$ be a stochastic process.
Let $(\varepsilon_n)_{n \in \mathbb{N}}$ be a sequence of positive numbers and $C_n$ a finite $\varepsilon_n$-cover of $T$ with $C_n \subseteq T$.
For $m \le k$,
\begin{align*}
  \mathbb{E}\left[\sup_{t \in C_k} d_E(X_t, X_{\bar{t}_m})^p \right]
  &\le \left(\sum_{i=m}^{k-1} \left( \mathbb{E}\left[\sup_{t \in C_k} d_E(X_{\bar{t}_i}, X_{\bar{t}_{i+1}})^p\right] \right)^{1/p}\right)^p
  \: .
\end{align*}
\end{lemma}

\begin{proof}
\begin{align*}
  \sup_{t \in C_k} d_E(X_t, X_{\bar{t}_m})^p
  &\le \sup_{t \in C_k} \left( \sum_{i=m}^{k-1} d_E(X_{\bar{t}_i}, X_{\bar{t}_{i+1}}) \right)^p
  \\
  &\le \left( \sum_{i=m}^{k-1} \sup_{t \in C_k} d_E(X_{\bar{t}_i}, X_{\bar{t}_{i+1}}) \right)^p
  \: .
\end{align*}
And then, by Minkowski's inequality,
\begin{align*}
  \left(\mathbb{E} \left[\sup_{t \in C_k} d_E(X_t, X_{\bar{t}_m})^p \right]\right)^{1/p}
  &\le \sum_{i=m}^{k-1} \mathbb{E} \left[\sup_{t \in C_k} d_E(X_{\bar{t}_i}, X_{\bar{t}_{i+1}}) \right]
  \\
  &\le \sum_{i=m}^{k-1} \left( \mathbb{E}\left[\sup_{t \in C_k} d_E(X_{\bar{t}_i}, X_{\bar{t}_{i+1}})^p \right] \right)^{1/p}
  \: .
\end{align*}
\end{proof}


\begin{lemma}\label{lem:integral_sup_rpow_dist_le_sum}
  \uses{def:IsKolmogorovProcess}
  \leanok
  \lean{ProbabilityTheory.lintegral_sup_rpow_edist_le_sum}
Let $X : T \to \Omega \to E$ be a process that satisfies the Kolmogorov condition for exponents $(p,q)$ with constant $M$.
Let $(\varepsilon_n)_{n \in \mathbb{N}}$ be a sequence of positive numbers and $C_n$ a finite $\varepsilon_n$-cover of $T$ with $C_n \subseteq T$.
Then for $m \le k$,
\begin{align*}
  \mathbb{E} \left[\sup_{t \in C_k} d_E(X_t, X_{\bar{t}_m})^p \right]
  &\le M \left( \sum_{j=m}^{k-1} \vert C_{j+1} \vert^{1/p} \varepsilon_j^{q/p} \right)^p
  \: .
\end{align*}
\end{lemma}

\begin{proof}
  \uses{lem:integral_sup_rpow_dist_succ, lem:integral_sup_dist_le_sum_rpow}
Put together Lemma~\ref{lem:integral_sup_rpow_dist_succ} and Lemma~\ref{lem:integral_sup_dist_le_sum_rpow}.
\end{proof}


\begin{lemma}\label{lem:integral_sup_rpow_dist_le_of_minimal_cover}
  \uses{def:IsKolmogorovProcess, def:HasBoundedInternalCoveringNumber}
  \leanok
  \lean{ProbabilityTheory.lintegral_sup_rpow_edist_le_of_minimal_cover}
Let $X : T \to \Omega \to E$ be a process that satisfies the Kolmogorov condition for exponents $(p,q)$ with constant $M$.
Let $(\varepsilon_n)_{n \in \mathbb{N}}$ be a sequence of positive numbers in $(0, \mathrm{diam}(T))$ and $C_n$ a finite $\varepsilon_n$-cover of $T$ with $C_n \subseteq T$, and with minimal cardinality.
Suppose that $T$ has bounded internal covering number with constant $c_1>0$ and exponent $d > 0$.
Then for $m \le k$,
\begin{align*}
  \mathbb{E} \left[\sup_{t \in C_k} d_E(X_t, X_{\bar{t}_m})^p \right]
  &\le M c_1 \left( \sum_{j=m}^{k-1} \varepsilon_{j+1}^{-d/p} \varepsilon_j^{q/p} \right)^p
  \: .
\end{align*}
\end{lemma}

\begin{proof}
  \uses{lem:integral_sup_rpow_dist_le_sum, def:HasBoundedInternalCoveringNumber}
By Lemma~\ref{lem:integral_sup_rpow_dist_le_sum}, we have
\begin{align*}
  \mathbb{E} \left[\sup_{t \in C_k} d_E(X_t, X_{\bar{t}_m})^p \right]
  &\le M \left( \sum_{j=m}^{k-1} \vert C_{j+1} \vert^{1/p} \varepsilon_j^{q/p} \right)^p
  \: .
\end{align*}
Then by the minimality of the cardinality of $C_n$ and the bounded internal covering number hypothesis, we have
\begin{align*}
  \vert C_{j+1} \vert
  &\le N^{int}_{\varepsilon_{j+1}}(T)
  \le c_1 \varepsilon_{j+1}^{-d}
  \: .
\end{align*}
\end{proof}


\begin{corollary}\label{cor:integral_sup_rpow_dist_le_of_minimal_cover_two}
  \uses{def:IsKolmogorovProcess, def:HasBoundedInternalCoveringNumber}
  \leanok
  \lean{ProbabilityTheory.lintegral_sup_rpow_edist_le_of_minimal_cover_two}
Under the assumptions of Lemma~\ref{lem:integral_sup_rpow_dist_le_of_minimal_cover}, for $\varepsilon_n = \varepsilon_0 2^{-n}$, then for $m \le k$,
\begin{align*}
  \mathbb{E} \left[\sup_{t \in C_k} d_E(X_t, X_{\bar{t}_m})^p \right]
  &\le M c_1 \varepsilon_0^{q - d} 2^{d - m(q-d)}\left( \frac{1 - 2^{- (k - m) (q -d)/p}}{1 - 2^{- (q -d)/p}}\right)^p
  \\
  &\le 2^d M c_1 (\varepsilon_0 2^{-m + 1})^{q - d} \frac{1}{\left( 2^{(q -d)/p} - 1\right)^p}
  \: .
\end{align*}
\end{corollary}

\begin{proof}
  \uses{lem:integral_sup_rpow_dist_le_of_minimal_cover}

\end{proof}



\subsubsection{Putting it all together}

\begin{theorem}\label{thm:finite_set_bound_of_dist_le}
  \uses{def:IsKolmogorovProcess, def:HasBoundedInternalCoveringNumber}
  \leanok
  \lean{ProbabilityTheory.finite_set_bound_of_edist_le}
Suppose that $T$ is a finite set with bounded internal covering number with constant $c_1>0$ and exponent $d > 0$.
Let $X : T \to \Omega \to E$ be a process that satisfies the Kolmogorov condition for exponents $(p,q)$ with constant $M$, with $q > d$ and $p > 0$.
For all $\delta > 0$,
\begin{align*}
  &\mathbb{E}\left[ \sup_{s, t \in T; d_T(s, t) \le \delta} d_E(X_s, X_t)^p \right]
  \\
  &\le 4^{p+2q+1} M \delta^{q-d} \left(\delta^d \left(\log_2 N^{int}_{\delta/4}(T) \right)^q  N^{int}_{\delta/4}(T)
    + \frac{c_1}{\left( 2^{(q -d)/p} - 1\right)^p}\right)
  \: .
\end{align*}
\end{theorem}

\begin{proof}
  \uses{cor:integral_sup_rpow_dist_le_of_minimal_cover_two, lem:integral_sup_rpow_dist_cover_rescale, cor:scale_change_rpow}
Let $\varepsilon_0 \in (0, \mathrm{diam}(T))$.
For all $n \in \mathbb{N}$, let $C_n$ a finite $\varepsilon_n$-cover of $T$ with $C_n \subseteq T$ for $\varepsilon_n = \varepsilon_0 2^{-n}$, with minimal cardinal.
Let $k = \min \{j \in \mathbb{Z} \mid \varepsilon_0 2^{-j} < \inf_{s, t \in J; s \ne t}d_T(s, t)\}$.
Note that $C_k = T$.
By Corollary~\ref{cor:scale_change_rpow},
\begin{align*}
  &\mathbb{E}\left[ \sup_{s, t \in C_k; d_T(s, t) \le \delta} d_E(X_s, X_t)^p \right]
  \\
  &\le 2^p \mathbb{E}\left[ \sup_{s, t \in C_k; d_T(s, t) \le \delta} d_E(X_{\bar{s}_m}, X_{\bar{t}_m})^p \right]
    + 4^p \mathbb{E}\left[ \sup_{s \in C_k} d_E(X_s, X_{\bar{s}_m})^p \right]
  \: .
\end{align*}
By Lemma~\ref{lem:integral_sup_rpow_dist_cover_rescale} and Corollary~\ref{cor:integral_sup_rpow_dist_le_of_minimal_cover_two},
\begin{align*}
  \mathbb{E} \left[ \sup_{s, t \in C_k; d_T(s, t) \le \delta} d_E(X_{\bar{s}_m}, X_{\bar{t}_m})^p \right]
  &\le 2^{p+2} M \left(16 \delta \log_2 N^{int}_{\delta/4}(T) \right)^q  N^{int}_{\delta/4}(J)
  \: . \\
  \mathbb{E} \left[\sup_{t \in C_k} d_E(X_t, X_{\bar{t}_m})^p \right]
  &\le2^d M c_1 (\varepsilon_0 2^{-m+1})^{q - d} \frac{1}{\left( 2^{(q -d)/p} - 1\right)^p}
  \: .
\end{align*}
Putting these two inequalities together, we obtain
\begin{align*}
  &\mathbb{E}\left[ \sup_{s, t \in C_k; d_T(s, t) \le \delta} d_E(X_s, X_t)^p \right]
  \\
  &\le 4^p M \left(4\left(16 \delta \log_2 N^{int}_{\delta/4}(T) \right)^q  N^{int}_{\delta/4}(T)
    + 2^d c_1 (\varepsilon_0 2^{-m+1})^{q - d} \frac{1}{\left(2^{(q -d)/p} - 1\right)^p}\right)
  \: .
\end{align*}
TODO: it remains to argue that w.l.o.g. $\varepsilon_0 2^{-m} \le \delta$ as in another lemma above (todo: refactor that).
We obtain
\begin{align*}
  &\mathbb{E}\left[ \sup_{s, t \in T; d_T(s, t) \le \delta} d_E(X_s, X_t)^p \right]
  \\
  &\le 4^p M \left(4^{2q+1}\delta^q \left(\log_2 N^{int}_{\delta/4}(T) \right)^q  N^{int}_{\delta/4}(T)
    + c_1 2^q \delta^{q - d} \frac{1}{\left( 2^{(q -d)/p} - 1\right)^p}\right)
  \\
  &\le 4^{p+2q+1} M \delta^{q-d} \left(\delta^d \left(\log_2 N^{int}_{\delta/4}(T) \right)^q  N^{int}_{\delta/4}(T)
    + \frac{c_1}{\left( 2^{(q -d)/p} - 1\right)^p}\right)
  \: .
\end{align*}
\end{proof}


\begin{corollary}\label{cor:finite_set_bound_of_dist_le_of_le_diam}
  \uses{def:IsKolmogorovProcess, def:HasBoundedInternalCoveringNumber}
  \leanok
  \lean{ProbabilityTheory.finite_set_bound_of_edist_le_of_le_diam}
With the same assumptions and notations as in Theorem~\ref{thm:finite_set_bound_of_dist_le}, for all $\delta \in (0, 4\mathrm{diam}(T))$,
\begin{align*}
  \mathbb{E}\left[ \sup_{s, t \in T; d_T(s, t) \le \delta} d_E(X_s, X_t)^p \right]
  &\le 4^{p+2q+1} M c_1 \delta^{q-d} \left(4^d \left(\log_2 \left(c_1 \delta^{-d} 4^d \right) \right)^q
    + \frac{1}{\left( 2^{(q -d)/p} - 1\right)^p}\right)
  \: .
\end{align*}
\end{corollary}

\begin{proof}
  \uses{thm:finite_set_bound_of_dist_le}
We apply Theorem~\ref{thm:finite_set_bound_of_dist_le} and then remark that for $\delta \le 4\mathrm{diam}(T)$, we can use the bounded internal covering number hypothesis to bound $N^{int}_{\delta/4}(T)$~:
\begin{align*}
  N^{int}_{\delta/4}(T) \le c_1 \left(\frac{\delta}{4}\right)^{-d} \: .
\end{align*}
\end{proof}





\section{Kolmogorov-Chentsov Theorem}


\subsection{Sets with bounded internal covering number}

\begin{lemma}\label{lem:integral_div_dist_le_sum_integral_dist_le}
  \leanok
  \lean{ProbabilityTheory.lintegral_div_edist_le_sum_integral_edist_le}
Let $J \subseteq T$ be a finite set and suppose that $T$ has finite diameter.
For $k \in \mathbb{N}$, let $\eta_k = 2^{-k}(\mathrm{diam}(T) + 1)$.
For $X : T \to \Omega \to E$ a stochastic process and $\beta \in(0, (q - d)/p)$,
\begin{align*}
  \mathbb{E}\left[ \sup_{s, t \in J;\: s \ne t} \frac{d_E(X_s, X_t)^p}{d_T(s, t)^{\beta p}} \right]
  &\le \sum_{k=0}^\infty 2^{k \beta p} \mathbb{E}\left[ \sup_{s, t \in J;\: s \ne t, \: d_T(s, t) \le 2 \eta_k} d_E(X_s, X_t)^p \right]
  \: .
\end{align*}
\end{lemma}

\begin{proof}
We introduce for each $k \in \mathbb{N}$ the set of pairs $(s, t)$ such that $\eta_k < d_T(s, t) \le 2 \eta_k$.
Note that $\eta_k \ge 2^{-k}$.
\begin{align*}
  \mathbb{E}\left[ \sup_{s, t \in J;\: s \ne t} \frac{d_E(X_s, X_t)^p}{d_T(s, t)^{\beta p}} \right]
  &\le \sum_{k=0}^\infty \mathbb{E}\left[ \sup_{s, t \in J;\: s \ne t, \: \eta_k < d_T(s, t) \le 2 \eta_k} \frac{d_E(X_s, X_t)^p}{d_T(s, t)^{\beta p}} \right]
  \\
  &\le \sum_{k=0}^\infty \eta_k^{-\beta p} \mathbb{E}\left[ \sup_{s, t \in J;\: s \ne t, \: d_T(s, t) \le 2 \eta_k} d_E(X_s, X_t)^p \right]
  \\
  &\le \sum_{k=0}^\infty 2^{k \beta p} \mathbb{E}\left[ \sup_{s, t \in J;\: s \ne t, \: d_T(s, t) \le 2 \eta_k} d_E(X_s, X_t)^p \right]
  \: .
\end{align*}
\end{proof}


\begin{lemma}\label{lem:finite_set_bound}
  \uses{def:IsKolmogorovProcess, def:HasBoundedInternalCoveringNumber}
  \leanok
  \lean{ProbabilityTheory.finite_kolmogorov_chentsov, ProbabilityTheory.constL_lt_top}
Suppose that $J \subseteq T$ is a finite set and that $T$ has bounded internal covering number with constant $c_1>0$ and exponent $d > 0$.
Let $X : T \to \Omega \to E$ be a process that satisfies the Kolmogorov condition for exponents $(p,q)$ with constant $M$, with $q > d$ and $p > 0$.
Let $\beta \in(0, (q - d)/p)$.
Then there exists a constant $L(T, c_1, d, p, q, \beta) < \infty$ (that depends on $T$ bot not $J$) such that
\begin{align*}
  \mathbb{E}\left[ \sup_{s, t \in J;\: s \ne t} \frac{d_E(X_s, X_t)^p}{d_T(s, t)^{\beta p}} \right]
  \le M L(T, c_1, d, p, q, \beta)
  \: .
\end{align*}
\end{lemma}

\begin{proof}
  \uses{thm:finite_set_bound_of_dist_le, cor:finite_set_bound_of_dist_le_of_le_diam, lem:integral_div_dist_le_sum_integral_dist_le}
Let $\eta_k = 2^{-k}(\mathrm{diam}(T) + 1)$ for $k \in \mathbb{N}$.
By Lemma~\ref{lem:integral_div_dist_le_sum_integral_dist_le}, we have
\begin{align*}
  \mathbb{E}\left[ \sup_{s, t \in J;\: s \ne t} \frac{d_E(X_s, X_t)^p}{d_T(s, t)^{\beta p}} \right]
  &\le \sum_{k=0}^\infty 2^{k \beta p} \mathbb{E}\left[ \sup_{s, t \in J;\: s \ne t, \: d_T(s, t) \le 2 \eta_k} d_E(X_s, X_t)^p \right]
  \: .
\end{align*}
We want to show that the sum is finite.

Let $k_0 = \min \{k \in \mathbb{N} \mid \eta_{k+1} \le \mathrm{diam}(T)\}$.
We deal separately with the parts of the sum for $k < k_0$ and $k \ge k_0$.

\emph{Large $k$}.

For $k \ge k_0$, $2\eta_k / 4 = \eta_{k+1} \le \mathrm{diam}(T)$, which means that we can apply Corollary~\ref{cor:finite_set_bound_of_dist_le_of_le_diam} to bound each expectation in the sum.
\begin{align*}
  &\mathbb{E}\left[ \sup_{s, t \in J;\: s \ne t, \: d_T(s, t) \le 2 \eta_k} d_E(X_s, X_t)^p \right]
  \\
  &\le 4^{p+2q+1} M c_1 (2 \eta_k)^{q-d} \left(4^d \left(\log_2 \left(c_1 (2 \eta_k)^{-d} 4^d \right) \right)^q
    + \frac{1}{\left( 2^{(q -d)/p} - 1\right)^p}\right)
  \\
  &\le 4^{p+2q+1} M c_1 (\mathrm{diam}(T)+1)^{q-d} 2^{(-k + 1)(q-d)} \left(4^d \left(\log_2 \left(c_1 2^{(k + 1)d} \right) \right)^q
    + \frac{1}{\left( 2^{(q -d)/p} - 1\right)^p}\right)
  \\
  &= 4^{p+2q+1} M c_1 (\mathrm{diam}(T)+1)^{q-d} 2^{(-k + 1)(q-d)} \left(4^d \left(\log_2(c_1) + (k + 1)d \right)^q
    + \frac{1}{\left( 2^{(q -d)/p} - 1\right)^p}\right)
  \: .
\end{align*}
Let $a_k = 2^{k \beta p} 4^{p+2q+1} M c_1 (\mathrm{diam}(T)+1)^{q-d} 2^{(-k + 1)(q-d)} \left(4^d \left(\log_2(c_1) + (k + 1)d \right)^q
    + \frac{1}{\left( 2^{(q -d)/p} - 1\right)^p}\right)$.
To show that the sum $\sum_{k=k_0}^\infty a_k$ is finite, we can use the ratio test.
\begin{align*}
  \frac{\vert a_{k+1} \vert}{\vert a_k \vert}
  &= 2^{\beta p - (q - d)}
    \frac{\left(4^d \left(\log_2(c_1) + (k + 2)d \right)^q + \frac{1}{\left( 2^{(q -d)/p} - 1\right)^p}\right)}
    {\left(4^d \left(\log_2(c_1) + (k + 1)d \right)^q + \frac{1}{\left( 2^{(q -d)/p} - 1\right)^p}\right)}
\end{align*}
The limit at infinity of that ratio is $2^{\beta p - (q - d)} < 1$, hence the series $\sum_{k=k_0}^\infty a_k$ converges.

\emph{Small $k$}.

For $k < k_0$, $N^{int}_{\eta_{k+1}}(T) = 1$. Applying Theorem~\ref{thm:finite_set_bound_of_dist_le} with $\delta = 2 \eta_k$, we obtain
\begin{align*}
  &\mathbb{E}\left[ \sup_{s, t \in T; d_T(s, t) \le 2 \eta_k} d_E(X_s, X_t)^p \right]
  \\
  &\le 4^{p+2q+1} M (2 \eta_k)^{q-d} \frac{c_1}{\left( 2^{(q -d)/p} - 1\right)^p}
  \\
  &\le 4^{p+2q+1} M (\mathrm{diam}(T) + 1)^{q-d} 2^{(-k+1)q-d} \frac{c_1}{\left( 2^{(q -d)/p} - 1\right)^p}
  \: .
\end{align*}
This is an expression of the form $K 2^{-k(q-d)}$ for some constant $K < \infty$.
Coming back to the sum, we have
\begin{align*}
  \sum_{k=1}^{k_0 - 1} 2^{k \beta p} \mathbb{E}\left[ \sup_{s, t \in J;\: s \ne t, \: d_T(s, t) \le 2 \eta_k} d_E(X_s, X_t)^p \right]
  &\le K \sum_{k=1}^{k_0 - 1} 2^{k (\beta p - (q-d))}
  \\
  &\le K \frac{1}{1 - 2^{(\beta p - (q-d))}}
  \: .
\end{align*}
The sum is finite since $\beta p < (q - d)$ by assumption.
\end{proof}


\begin{theorem}\label{thm:countable_set_bound}
  \uses{def:IsKolmogorovProcess, def:HasBoundedInternalCoveringNumber}
  \leanok
  \lean{ProbabilityTheory.countable_kolmogorov_chentsov}
Suppose that $T$ has bounded internal covering number with constant $c_1>0$ and exponent $d > 0$.
Let $X : T \to \Omega \to E$ be a process that satisfies the Kolmogorov condition for exponents $(p,q)$ with constant $M$, with $q > d$ and $p > 0$.
Let $\beta \in(0, (q - d)/p)$.
Then there exists a constant $L(T, c_1, d, p, q, \beta) < \infty$ such that for every countable subset $T' \subseteq T$ with positive diameter,
\begin{align*}
  \mathbb{E}\left[ \sup_{s, t \in T';\: s \ne t} \frac{d_E(X_s, X_t)^p}{d_T(s, t)^{\beta p}} \right]
  \le M L(T, c_1, d, p, q, \beta)
  \: .
\end{align*}
\end{theorem}

\begin{proof}
  \uses{lem:finite_set_bound}
Build a monotone sequence of finite sets $T_n \subseteq T'$, use Lemma~\ref{lem:finite_set_bound} to obtain a bound for each $T_n$ that does not depend on $T_n$, and then use monotone convergence.
\end{proof}


\begin{corollary}\label{cor:countable_set_bound_of_le}
Under the same assumptions as in Theorem~\ref{thm:countable_set_bound}, for every countable subset $T' \subseteq T$ with positive diameter, for $L(T, c_1, d, p, q, \beta) < \infty$ the same constant,
\begin{align*}
  \mathbb{E}\left[ \sup_{s, t \in T';\: d_T(s, t) \le \delta} d_E(X_s, X_t)^p \right]
  \le M L(T, c_1, d, p, q, \beta) \delta^{\beta p}
  \: .
\end{align*}
\end{corollary}

\begin{proof}
  \uses{thm:countable_set_bound}
Immediately follows from Theorem~\ref{thm:countable_set_bound}.
\end{proof}


\begin{lemma}\label{lem:holder_modification_single}
  \uses{def:IsKolmogorovProcess, def:HasBoundedInternalCoveringNumber}
Under the assumptions of Theorem~\ref{thm:countable_set_bound}, for $E$ a complete space and $\beta \in (0, (q - d)/p)$, there exists a modification $Y$ of $X$ (i.e., a process $Y$ with $\mathbb{P}(Y_t \ne X_t) = 0$ for all $t$) such that the paths of $Y$ are Hölder continuous of order $\beta$.
\end{lemma}

\begin{proof}
  \uses{thm:countable_set_bound}
Let $T'$ be a countable dense subset of $T$.
Let $A$ be the event
\begin{align*}
  \left\{\sup_{s, t \in T';\: s \ne t} \frac{d_E(X_s, X_t)^p}{d_T(s, t)^{\beta p}} < \infty \right\}
  \: .
\end{align*}
As a consequence of Theorem~\ref{thm:countable_set_bound}, we have $\mathbb{P}(A) = 1$.

On the event $A$, $(X_t)_{t \in T'}$ has Hölder continuous paths of order $\beta$.
Let $x_0 \in E$ be arbitrary and let $Y: T \to \Omega \to E$ be the process defined by
\begin{align*}
  Y_t(\omega)
  &= \begin{cases}
    \lim_{s \to t, s \in T'} X_s(\omega) & \text{if } \omega \in A \\
    x_0 & \text{otherwise}
  \end{cases}
  \: .
\end{align*}
Then $Y$ has Hölder continuous paths of order $\beta$ almost surely.

We can show that $(Y_s, Y_t)$ is $\mathbb{P}$-a.e. measurable for all $s, t \in T$.

It remains to show that $Y$ is a modification of $X$.
Let then $t \in T$ and let $(t_n)_{n \in \mathbb{N}}$ be a sequence in $T'$ that converges to $t$.
We want to show that $\mathbb{P}(Y_t \ne X_t) = 0$.
It suffices to show that $\mathbb{P}(d_E(Y_t, X_t) > 0) = 0$, which itself would follow from $\mathbb{P}(d_E(Y_t, X_t) > \varepsilon) = 0$ for all $\varepsilon > 0$.

\begin{align*}
  \mathbb{P}(d_E(Y_t, X_t) > \varepsilon)
  &\le \mathbb{P}(d_E(Y_t, X_{t_n}) + d_E(X_{t_n}, X_t) > \varepsilon)
  \\
  &\le \mathbb{P}(d_E(Y_t, X_{t_n}) > \varepsilon/2) + \mathbb{P}(d_E(X_{t_n}, X_t) > \varepsilon/2)
  \: .
\end{align*}

TODO
\end{proof}


\begin{theorem}\label{thm:holder_modification}
  \uses{def:IsKolmogorovProcess, def:HasBoundedInternalCoveringNumber}
Under the assumptions of Theorem~\ref{thm:countable_set_bound}, for $E$ a complete space, there exists a modification $Y$ of $X$ (i.e., a process $Y$ with $\mathbb{P}(Y_t \ne X_t) = 0$ for all $t$) such that the paths of $Y$ are Hölder continuous of all orders $\gamma \in (0, (q - d)/p)$.
\end{theorem}

\begin{proof}
  \uses{lem:holder_modification_single, lem:indistinguishable_of_modification_of_continuous}
Let $(\beta_n)$ be an increasing sequence of numbers in $(0, (q - d)/p)$ such that $\beta_n \to (q - d)/p$.
For each $n$, let $Y^n$ be the modification of $X$ given by Lemma~\ref{lem:holder_modification_single} for $\beta = \beta_n$.
Then by Lemma~\ref{lem:indistinguishable_of_modification_of_continuous}, the processes $Y^0$ and $Y^n$ are indistinguishable for all $n$.
That is, there exists an event $A_n$ such that $\mathbb{P}(A_n) = 1$ and such that for all $\omega \in A_n$, $Y^0_t(\omega) = Y^n_t(\omega)$ for all $t \in T$.

Let $A = \bigcap_{n \in \mathbb{N}} A_n$ and let $x_0 \in E$ be arbitrary.
Then $\mathbb{P}(A) = 1$ and the process $Y(\omega) = Y^0(\omega)$ for $\omega \in A$ and $Y(\omega) = x_0$ for $\omega \notin A$ has paths that are Hölder continuous of all orders $\gamma \in (0, (q - d)/p)$.
\end{proof}



\subsection{Localized Kolmogorov-Chentsov theorem}

TODO: this section will be reworked. Don't formalize it yet.

\begin{definition}[Cover with bounded covering numbers]\label{def:HasBoundedCoveringNumberCover}
  \uses{def:HasBoundedInternalCoveringNumber}
A set $T$ is said to have a cover with bounded covering numbers if there exists a monotone sequence of totally bounded open subsets $(T_n)_{n \in \mathbb{N}}$ of $T$ such that for all $n$, $T_n$ has bounded internal covering number with constant $c_n>0$ and exponent $d_n > 0$, and such that $\bigcup_{n \in \mathbb{N}} T_n = T$.
\end{definition}


\begin{lemma}\label{lem:hasBoundedCoveringNumberCover_nnreal}
  \uses{def:HasBoundedCoveringNumberCover}
$\mathbb{R}_+$ has a cover with bounded covering numbers for the sets $T_n = [0,n)$, constants $c_n = n$ and exponents $d_n = 1$.
\end{lemma}

\begin{proof}

\end{proof}


\begin{definition}[Localized Kolmogorov conditions]\label{def:IsLocalizedKolmogorovProcess}
  \uses{def:IsKolmogorovProcess, def:HasBoundedCoveringNumberCover}
  \notready
Let $T$ be a metric space with a cover $(T_n)$ with bounded covering numbers with constants $c_n$ and exponents $d_n$.
A stochastic process $X : T \to \Omega \to E$ satisfies the localized Kolmogorov conditions if $(X_s, X_t)$ is $\mathbb{P}$-a.e. measurable for all $s, t \in T$ and if for all $n \in \mathbb{N}$, there are exponents $(p_n, q_n)$ with $p_n > 0$ and $q_n > d_n$ and a constant $M_n > 0$ as well as $\rho_n > 0$ such that for all $s, t \in T_n$ with $d_T(s, t) \le \rho_n$,
\begin{align*}
  \mathbb{E}\left[ d_E(X_s, X_t)^{p_n} \right]
  &\le M_n d_T(s, t)^{q_n}
  \: .
\end{align*}
\end{definition}


\begin{lemma}\label{lem:IsKolmogorovProcess.IsLocalizedKolmogorovProcess}
  \uses{def:IsKolmogorovProcess, def:HasBoundedCoveringNumberCover, def:IsLocalizedKolmogorovProcess}
  \notready
Let $T$ be a metric space with a cover $(T_n)$ with bounded covering numbers with constants $c_n$ and exponents $d_n$.
Let $X : T \to \Omega \to E$ be a process that satisfies the Kolmogorov condition for exponents $(p,q)$ with constant $M$, with $q > \sup_n d_n$.
Then $X$ satisfies the localized Kolmogorov conditions with exponents $(p_n, q_n) = (p, q)$, constants $M_n = M$ and any $\rho_n > 0$.
\end{lemma}

\begin{proof}

\end{proof}


\begin{theorem}\label{thm:localized_holder_modification}
  \uses{def:IsLocalizedKolmogorovProcess, def:HasBoundedCoveringNumberCover}
  \notready
Let $T$ be a metric space with a cover $(T_n)$ with bounded covering numbers with constants $c_n$ and exponents $d_n$.
Let $X : T \to \Omega \to E$ be a process that satisfies the localized Kolmogorov conditions with exponents $(p_n, q_n)$ with $q_n > d_n$.
Then $X$ has a modification $Y$ such that almost surely the paths of $Y$ are Hölder continuous of all orders $\gamma \in \bigcap_n [0, (q_n - d_n)/p_n)$, in which ``Hölder continuous of order 0'' means ``uniformly continuous''.
Moreover, for $n \in \mathbb{N}$, $t \in T_n$, there is an open neighborhood $V_t$ of $t$ in $T$ such that for all $\beta \in [0, (q_n - d_n)/p_n)$,
\begin{align*}
  \mathbb{E}\left[ \sup_{s, t \in V_t} \frac{d_E(Y_s, Y_t)^{p_n}}{d_T(s, t)^{\beta p_n}} \right]
  &< \infty
  \: .
\end{align*}
\end{theorem}

\begin{proof}
  \uses{thm:holder_modification}

\end{proof}

\chapter{Brownian motion}
\label{chap:brownian}


\section{Stochastic process with continuous paths}

\begin{definition}[pre-Brownian process]\label{def:preBrownian}
  \uses{def:gaussianLimit}
  \leanok
  \lean{ProbabilityTheory.preBrownian}
Let $\Omega = \mathbb{R}^{\mathbb{R}_+}$ and consider the probability space $(\Omega, P_B)$ (where $P_B$ is the measure defined in Definition~\ref{def:gaussianLimit}).
The identity on that space is a function $\Omega \to \mathbb{R}_+ \to \mathbb{R}$.
We reorder the arguments to define a stochastic process $X : \mathbb{R}_+ \to \Omega \to \mathbb{R}$, which we call the pre-Brownian process.
\end{definition}


\begin{lemma}\label{lem:isGaussianProcess_preBrownian}
  \uses{def:preBrownian, def:IsGaussianProcess}
  \leanok
  \lean{ProbabilityTheory.isGaussianProcess_preBrownian}
  The pre-Brownian process $X$ of Definition~\ref{def:preBrownian} is a Gaussian process.
\end{lemma}

\begin{proof}\leanok
  \uses{lem:isGaussian_multivariateGaussian}

\end{proof}


\begin{lemma}\label{lem:map_sub_preBrownian}
  \uses{def:preBrownian}
  \leanok
  \lean{ProbabilityTheory.map_sub_preBrownian}
Let $X$ be the pre-Brownian process of Definition~\ref{def:preBrownian}.
Then, for all $s, t \in \mathbb{R}_+$, the random variable $X_t - X_s$ is a Gaussian random variable with mean $0$ and variance $|t - s|$.
\end{lemma}

\begin{proof}

\end{proof}


\begin{lemma}\label{lem:isKolmogorovProcess_preBrownian}
  \uses{def:preBrownian}
  \leanok
  \lean{ProbabilityTheory.isKolmogorovProcess_preBrownian}
The pre-Brownian process $X$ of Definition~\ref{def:preBrownian} satisfies the Kolmogorov condition for exponents $(2n,n)$ with constant $(2n - 1)!!$ for all $n \in \mathbb{N}$.
That is, for all $s, t \in \mathbb{R}_+$, we have
\begin{align*}
  \mathbb{E} \left[ |X_t - X_s|^{2n} \right] \le (2n - 1)!! |t - s|^n
  \: .
\end{align*}
\end{lemma}

\begin{proof}
  \uses{lem:centralMoment_two_mul_gaussianReal, lem:map_sub_preBrownian}
$X_t - X_s$ is a Gaussian random variable with mean $0$ and variance $|t - s|$ (Lemma~\ref{lem:map_sub_preBrownian}).
Thus, by Lemma~\ref{lem:centralMoment_two_mul_gaussianReal}, we have
\begin{align*}
  \mathbb{E} \left[ |X_t - X_s|^{2n} \right]
  = (2n - 1)!! |t - s|^n
  \: .
\end{align*}
\end{proof}


\begin{definition}[Brownian motion]\label{def:brownian}
  \uses{thm:localized_holder_modification_sup, def:preBrownian, lem:isKolmogorovProcess_preBrownian, lem:hasBoundedCoveringNumberCover_nnreal}
  \leanok
By Theorem~\ref{thm:localized_holder_modification_sup}, there exists a modification $B$ of the pre-Brownian process such that all the paths of $B$ are Hölder continuous of all orders $\gamma \in (0, 1/2)$.
We call $B$ the \emph{Brownian motion} on $\mathbb{R}_+$.
\end{definition}


\begin{lemma}\label{lem:isGaussianProcess_brownian}
  \uses{def:brownian, def:IsGaussianProcess}
  \leanok
  \lean{ProbabilityTheory.isGaussianProcess_brownian}
The Brownian motion is a Gaussian process.
\end{lemma}

\begin{proof}\leanok
  \uses{lem:isGaussianProcess_of_modification, lem:isGaussianProcess_preBrownian}
The pre-Brownian process is a Gaussian process by Lemma~\ref{lem:isGaussianProcess_preBrownian}.
The Brownian motion is a modification of the pre-Brownian process by Definition~\ref{def:brownian}.
Thus, the Brownian motion is a Gaussian process as well by Lemma~\ref{lem:isGaussianProcess_of_modification}.
\end{proof}


\begin{lemma}\label{lem:isHolderWith_brownian}
  \uses{def:brownian}
  \leanok
  \lean{ProbabilityTheory.isHolderWith_brownian}
The paths of the Brownian motion are Hölder continuous of all orders $\gamma \in (0, 1/2)$.
\end{lemma}

\begin{proof}

\end{proof}


\begin{lemma}\label{lem:continuous_brownian}
  \uses{def:brownian}
  \leanok
  \lean{ProbabilityTheory.continuous_brownian}
The paths of the Brownian motion are continuous.
\end{lemma}

\begin{proof}
  \uses{lem:isHolderWith_brownian}

\end{proof}


\begin{lemma}\label{lem:law_brownian_sub}
  \uses{def:brownian}
For $s, t \in \mathbb{R}_+$, the law of $B_t - B_s$ is the real Gaussian measure $\mathcal{N}(0,\vert t - s \vert)$.
\end{lemma}

\begin{proof}

\end{proof}


\begin{lemma}\label{lem:law_brownian_apply}
  \uses{def:brownian}
For $t \in \mathbb{R}_+$, the law of $B_t$ (the Brownian motion at time $t$) is the real Gaussian measure $\mathcal{N}(0,t)$.
\end{lemma}

\begin{proof}

\end{proof}

\section{Wiener measure on the continuous functions}

We want to turn the Brownian motion into a measure on the continuous functions $C(\mathbb{R}_+, \mathbb{R})$ with the Borel sigma-algebra generated by the compact-open topology.


\begin{definition}[Auxiliary Wiener measure]\label{def:wienerMeasureAux}
  \uses{def:brownian, def:gaussianLimit, lem:continuous_brownian}
  \leanok
  \lean{ProbabilityTheory.wienerMeasureAux}
The pushforward of the measure $P_B$ of Definition~\ref{def:gaussianLimit} by the Brownian motion $B$ is a measure on the continuous functions on $\mathbb{R}^{\mathbb{R}_+}$, with the sigma-algebra induced by the product sigma-algebra on $\mathbb{R}^{\mathbb{R}_+}$.
\end{definition}

\textbf{Lean remark}: the auxiliary Wiener measure is a measure on the subtype \texttt{\{f  // Continuous f\}}. This is not the same type as $C(\mathbb{R}_+, \mathbb{R})$.


\begin{theorem}\label{thm:ContinuousMap.borel_eq_iSup_comap_eval}
  \leanok
  \lean{ProbabilityTheory.ContinuousMap.borel_eq_iSup_comap_eval}
The borel sigma-algebra on $C(\mathbb{R}_+, \mathbb{R})$ coming from the compact-open topology is equal to the smallest sigma-algebra for which the evaluation maps $f \mapsto f(t)$ are measurable for every $t \in \mathbb{R}_+$.
\end{theorem}

\begin{proof}
Possible ref: \href{https://math.stackexchange.com/questions/4789531/when-does-the-borel-sigma-algebra-of-compact-convergence-coincide-with-the-pr}{stackexchange question}.
\end{proof}


\begin{definition}\label{def:MeasurableEquiv.continuousMap}
  \uses{thm:ContinuousMap.borel_eq_iSup_comap_eval}
  \leanok
  \lean{ProbabilityTheory.MeasurableEquiv.continuousMap}
The identity is a measurable equivalence between the continuous functions of $\mathbb{R}^{\mathbb{R}_+}$ with the subset sigma-algebra obtained from the product sigma-algebra, and $C(\mathbb{R}_+, \mathbb{R})$ with the Borel sigma-algebra coming from the compact-open topology.

Mathematically this says nothing more than the equality of sigma-algebras of Theorem~\ref{thm:ContinuousMap.borel_eq_iSup_comap_eval} but in Lean we have two different types so we need an equivalence.
\end{definition}


\begin{definition}[Wiener measure]\label{def:wienerMeasure}
  \uses{def:MeasurableEquiv.continuousMap, def:wienerMeasureAux}
  \leanok
  \lean{ProbabilityTheory.wienerMeasure}
The Wiener measure on $C(\mathbb{R}_+, \mathbb{R})$ with the Borel sigma-algebra is the map of the auxiliary Wiener measure by the measurable equivalence of definition~\ref{def:MeasurableEquiv.continuousMap}.
\end{definition}


TODO: add the main properties of the Brownian motion and the Wiener measure.
We need to be able to tell that we have built the correct objects.


\putbib

\part{Stochastic integral}

\textbf{Overview}

We describe the construction of a stochastic integral.

\textbf{Status} The formalization is ongoing.

\textbf{Formalization authors} Anyone is welcome to contribute!

\chapter{Doob-Meyer Theorem}
\label{chap:doob_meyer}


This chapter follows \cite{Beiglböck_Schachermayer_Veliyev_2012}
which gives an elementary and short proof of the result.



\section{Cadlag modifications of (local) martingales}



\begin{definition}\label{def:Doob_Meyer_class}
  \uses{def:IsStoppingTime}
$D$ is the class of all adapted processes $(S_t)_{0\leq t\leq T}$ such that the set $\{S_\tau \mid \tau \text{ is a stopping time}\}$ is uniformly integrable.
\end{definition}


\begin{definition}[Dyadics]\label{def:dyadics}
For $T>0$, let $\mathcal{D}_n^T = \left\lbrace \frac{k}{2^n}T \mid k=0,\cdots 2^n\right\rbrace$ be the set of dyadics at scale $n$ and let $\mathcal{D}^T=\bigcup_{n\in\mathbb{N}}\mathcal{D}_n^T$ be the set of all dyadics of $[0,T]$.
\end{definition}


\begin{lemma}\label{lem:martingale_exists_dyadic_limit_left}
  \uses{def:dyadics, def:Martingale}
  Let $X=(X_t)_{t\in\mathcal{D}}$ be a martingale indexed by the dyadics. Then almost surely, for every $t\geq 0$ the limit
  $$
  \lim_{\stackrel{s\rightarrow t^-}{s\in\mathcal{D}}}X_s(\omega)
  $$
  exists and is finite.
\end{lemma}

\begin{proof}
  See 8.2.1 of Pascucci.
\end{proof}


\begin{lemma}\label{lem:martingale_exists_dyadic_limit_right}
  \uses{def:dyadics, def:Martingale}
  Let $X=(X_t)_{t\in\mathcal{D}}$ be a martingale indexed by the dyadics. Then almost surely, for every $t\geq 0$ the limit
  $$
  \lim_{\stackrel{s\rightarrow t^+}{s\in\mathcal{D}}}X_s(\omega)
  $$
  exists and is finite.
\end{lemma}

\begin{proof}
  See 8.2.1 of Pascucci.
\end{proof}


\begin{lemma}\label{lem:mg_is_cadlag}
  \uses{def:usualConditions, def:Martingale}
  Let the filtered probability space satisfy the usual conditions.
  Then every martingale $X$ admits a modification that is still a martingale with cadlag trajectories.
\end{lemma}

\begin{proof}
  \uses{lem:martingale_exists_dyadic_limit_right,lem:martingale_exists_dyadic_limit_left}
  See 8.2.3 of Pascucci.
\end{proof}


\begin{lemma}\label{lem:exists_cadlag_mod_of_nonneg_submg}
  \uses{def:usualConditions, def:Submartingale}
  Let the filtered probability space satisfy the usual conditions.
  Then every nonnegative submartingale $X$ admits a modification that is still a nonnegative submartingale with cadlag trajectories.
\end{lemma}

\begin{proof}
  \uses{lem:martingale_exists_dyadic_limit_right,lem:martingale_exists_dyadic_limit_left}
  See 8.2.3 of Pascucci.
\end{proof}


\begin{lemma}\label{lem:exists_cadlag_mod_of_local_mg}
  \uses{def:usualConditions}
  Let the filtered probability space satisfy the usual conditions.
  Then every local martingale $X$ admits a modification that is still a local martingale with cadlag trajectories.
\end{lemma}

\begin{proof}
  \uses{lem:mg_is_cadlag}
\end{proof}



\section{Komlòs Lemma}



Firstly we will need Komlos' Lemma.
%technically a more general version with Cesaro sums exists, but it is not needed for this case
%(see "J. Komlòs, A generalization of a problem of Steinhaus, Acta Math. Acad. Sci. Hungar. 18 (1967) 217–229").


\begin{lemma}\label{lem:komlos_aux}
  Let $H$ be a Hilbert space and $(f_n)_{n\in\mathbb{N}}$ a bounded sequence in $H$. Then there exist functions $g_n\in convex(f_n,f_{n+1},\cdots)$ such that $(g_n)_{n\in\mathbb{N}}$ converges in $H$.
\end{lemma}

\begin{proof}
  Let $r_n = \inf(\|g\|_2:g\in convex(f_n, f_{n+1},\ldots))$.
  Let $A=\sup_{n\geq1} r_n$. $A$ is finite by boundedness of $(f_n)_{n\in\mathbb{N}}$ and
  for each $n$ we  may pick some $g_n\in convex(f_n, f_{n+1},\ldots)$ such that $ \|g_n\|_2\leq A+1/n$ by $\inf$ and $\sup$ definitions.
  Let $\epsilon>0$.
  By construction $(r_n)_{n\in\mathbb{N}}$ is increasing. By properties of $\sup$ there exists $\bar{n}$ such that $r_{\bar{n}}\geq A-\epsilon$ and such that $\frac{1}{\bar{n}}\leq\epsilon$.
  Let $m\geq k\geq \bar{n}$. $(g_k+g_m)/2 \in convex(f_k,f_{k+1},\ldots)$. It follows since $(r_n)_{n\in\mathbb{N}}$ is increasing that
  $\|(g_k+g_m)/2\|_2\geq A-\epsilon$.
  Hence due to the ordering of $m,k,\bar{n}$
  $$ \|g_k-g_m\|_2^2=2 \|g_k\|_2^2+2\|g_m\|_2^2- \|g_k+g_m\|_2^2
  \leq 4(A+\frac{1}{\bar{n}})^2-4(A-\epsilon)^2\leq 16A\epsilon.$$ By completeness, $(g_n)_{n\geq1}$  converges in $\|.\|_2$.
\end{proof}


\begin{lemma}\label{lem:convex_of_converg_seq_is_converg}
  Let $X$ be a normed vector space (over $\mathbb{R}$).
  %For topological spaces we need that a convex combinations of elements of neighborhoods are still in the neighborhood (just like balls). Also for metric space we want that $d(ax,ay)\leq a d(x,y)$ (in lean dist_pair_smul).
  Let $(x_n)_{n\in\mathbb{N}}$ be a sequence in $X$ converging to $x$ w.r.t. the topology of $X$.
  Let $(N_n)_{n\in\mathbb{N}}$ be a sequence in $\mathbb{N}$ such that $n\leq N_n$ for every $n\in\mathbb{N}$ (maybe here we could have $N_n$ increasing WLOG).
  Let $(a_{n,m})_{n\in\mathbb{N},m\in\left\lbrace n,\cdots,N_n\right\rbrace}$ be a triangular array in $\mathbb{R}$ such that $0\leq a_{n,m}\leq 1$ and $\sum_{m=n}^{N_n}a_{n,m}=1$.
  Then $(\sum_{m=n}^{N_n}a_{n,m}x_m)_{n\in\mathbb{N}}$ converges to $x$ uniformly w.r.t. the triangular array.
\end{lemma}

\begin{proof}
  Let $\epsilon>0$.
  By convergence of $x_n$ we have $\exists \bar{n}$ such that $\forall n\geq\bar{n}$ $|x_n-x|\leq \epsilon$.
  By triangular inequality it follows that
  $$
  |\sum_{m=n}^{N_n}a_{n,m}x_m - x|\leq \sum_{m=n}^{N_n}a_{n,m}|x_m-x|\leq\epsilon.
  $$
\end{proof}


\begin{lemma}\label{lem:komlos_convex_aux}
  For $i,n\in\mathbb{N}$ set $f_{n}^{(i)}:=f_n \mathbb{1}_{(|f_n|\leq i)}$ such that $f_{n}^{(i)}\in L^2$.
  There exists the sequence of convex weights $\lambda_n^{n}, \ldots, \lambda_{N_n}^{n}$ such that the functions
  $ (\lambda_n^{n} f_n^{(i)} + \ldots+\lambda_{N_n}^{n} f_{N_n}^{(i)})_{n\in\mathbb{N}}$
  converge in $L^2$ for every $i\in\mathbb{N}$ uniformly.
\end{lemma}

\begin{proof}
  \uses{lem:komlos_aux, lem:convex_of_converg_seq_is_converg}
  Firstly by lemma \ref{lem:komlos_aux} over $(f_n^{(1)})_{n\in\mathbb{N}}$ there exist convex weights $\prescript{1}{}{\lambda}^n_n,\cdots,\prescript{1}{}{\lambda}^n_{N^1_n}$ such that
  $g^1_n=\sum_{m=n}^{N^1_n}\prescript{1}{}{\lambda}^n_mf_m^{(1)}$ converges to some $g^1$.
  Secondly apply the lemma to $(\tilde{g}^2_n=\sum_{m=n}^{N^1_n}\prescript{1}{}{\lambda}^n_mf^{(2)}_m)_{n\in\mathbb{N}}$, there exists convex weights $\tilde{\lambda}^n_n,\cdots,\tilde{\lambda}^n_{\tilde{N}_n}$ such that
  $g^2_n=\sum_{m=n}^{\tilde{N}_n}\tilde{\lambda}^n_m\tilde{g}_m^{(2)}=\sum_{m=n}^{N^2_n}\prescript{2}{}{\lambda}^n_mf_m^{(2)}$ converges to some $g^2$.
  Notice that $\sum_{m=n}^{N^2_n}\prescript{2}{}{\lambda}^n_mf_m^{(1)}=\sum_{m=n}^{\tilde{N}_n}\tilde{\lambda}^n_m\tilde{g}_m^{(1)}$ and thus this sequence by lemma \ref{lem:convex_of_converg_seq_is_converg} converges still to $g^1$.
  By iteration we may define $\prescript{i}{}{\lambda}^n_n,\cdots,\prescript{i}{}{\lambda}^n_{N^i_n}$ convex weights such that if used on $(f^j_n)_{n\in\mathbb{N}}$ they make the sequence convergent if $1\leq j\leq i$.
  At this point consider $\lambda^n_m=\prescript{n}{}{\lambda}^n_m$.
  Since $\forall m\geq i$ $\sum_{j=n}^{N^m_n}\prescript{m}{}{\lambda}^n_j f^{(i)}_j\rightarrow g^i$ and even better
  $\forall\epsilon>0$ $\exists\bar{n}$, $\forall n\geq\bar{n}$, $\forall m\geq i$ $|\sum_{j=n}^{N^m_n}\prescript{m}{}{\lambda}^n_j f^{(i)}_j - g^i|\leq\epsilon$
  (this works by lemma \ref{lem:convex_of_converg_seq_is_converg} uniformity of convergence w.r.t. triangular array) this concludes.
\end{proof}


\begin{lemma}[Komlòs Lemma]\label{lem:komlos}
  Let $( f_n)_{n\in\mathbb{N}}$ be a uniformly integrable sequence of functions on a probability space $(\Omega , \mathcal{F} , P)$.
  Then there exist functions $g_n \in convex( f_n, f_{n+1}, \cdots)$ such that $(g_n)_{n\in\mathbb{N}}$ converges in  $L^1 (\Omega )$.
\end{lemma}

\begin{proof}
  \uses{lem:komlos_convex_aux}
  For $i,n\in\mathbb{N}$ set $f_{n}^{(i)}:=f_n \mathbb{1}_{(|f_n|\leq i)}$ such that $f_{n}^{(i)}\in L^2$.
  Using \ref{lem:komlos_convex_aux} there exist for every $n$ convex weights $\lambda_n^{n}, \ldots, \lambda_{N_n}^{n}$ such that the functions
  $ \lambda_n^{n} f_n^{(i)} + \ldots+\lambda_{N_n}^{n} f_{N_n}^{(i)}$ converge in $L^2$ for every $i\in\mathbb{N}$.
  By uniform integrability, $\lim_{i\to \infty}\| f^{(i)}_n- f_n\|_1=0$, uniformly with respect to $n$.
  Hence, once again, uniformly with respect to $n$,
  $$ \textstyle\lim_{i\to\infty}\|  (\lambda_n^{n} f_n^{(i)} + \ldots+\lambda_{N_n}^{n} f_{N_n}^{(i)})-(\lambda_n^{n} f_n + \ldots+\lambda_{N_n}^{n} f_{N_n})\|_1= 0.$$
  Thus $(\lambda_n^{n} f_n + \ldots+\lambda_{N_n}^{n} f_{N_n})_{n\geq 1}$  is a Cauchy sequence in $L^1$.
\end{proof}



\section{Doob-Meyer decomposition}



For uniqueness of Doob-Meyer Decomposition we will need theorem \ref{thm:IsLocalMartingale.eq_zero_of_finiteVariation}.

We now start the construction for the existence part.
Let $T>0$ and recall that $\mathcal{D}_n^T=\left\lbrace \frac{k}{2^n}T \mid k=0,\cdots 2^n\right\rbrace$.

TODO: everywhere below, $S$ is a cadlag submartingale of class D on $[0,T]$?

\begin{definition}[A]\label{def:A}
  \uses{def:dyadics}
Define $A_0=0$ and for $t\in\mathcal{D}_n^T$ positive,
\begin{align*}
A^n_t
&=A^n_{t-T2^{-n}} + \mathbb{E}\left[ S_t-S_{t-T2^{-n}}|\mathcal{F}_{t-T2^{-n}}\right]
\: .
\end{align*}
\end{definition}


\begin{definition}[M]\label{def:M}
  \uses{def:A}
For $t\in\mathcal{D}_n^T$, define $M^n_t = S_t-A^n_t$~.
\end{definition}


\begin{lemma}\label{lem:Doob_Meyer_Finite_Predictable}
  \uses{def:A, def:predictable}
  $(A^n_t)_{t\in\mathcal{D}_n^T}$ is a predictable process.
\end{lemma}

\begin{proof}
  Trivial
\end{proof}


\begin{lemma}\label{lem:Doob_Meyer_Finite_Martingale}
  \uses{def:M, def:Martingale}
  $(M^n_t)_{t\in\mathcal{D}_n^T}$ is a martingale.
\end{lemma}

\begin{proof}
  Trivial
\end{proof}


\begin{lemma}\label{lem:Predict_Part_Increasing}
  \uses{def:A}
  $(A^n_t)_{t\in\mathcal{D}_n^T}$ is an increasing process.
\end{lemma}

\begin{proof}
  \uses{def:Submartingale}
  $S$ is a submartingale.
\end{proof}


\begin{definition}\label{def:hittingAGT}
  \uses{def:A}
Let $c>0$. Define the hitting time on $\mathcal{D}^T_n$
\begin{align*}
  \tau_n(c)
  &= \inf\{t \in \mathcal{D}^T_n \mid A^n_{t + 2^{-n}T} > c\} \wedge T
  \: .
\end{align*}
\end{definition}


\begin{lemma}\label{lem:IsStoppingTime_hittingAGT}
  \uses{def:IsStoppingTime,def:hittingAGT}
  $\tau_n(c)$ is a stopping time.
\end{lemma}

\begin{proof}
Since $A^n_{t}$ is predictable, $A^n_{t + 2^{-n}T}$ is adapted.
The stopping time of an adapted process is a stopping time.
\end{proof}


\begin{lemma}\label{lem:A_uniform_integrable}
  \uses{def:A}
  The sequence $(A^n_T)_{n\in\mathbb{N}}$ is uniformly integrable (bounded in $L^1$ norm).
\end{lemma}

\begin{proof}
  \uses{lem:Doob_Meyer_Finite_Predictable,lem:Predict_Part_Increasing,lem:Doob_Meyer_Finite_Martingale,lem:IsStoppingTime_hittingAGT}
  WLOG $S_T=0$ and $S_t\leq 0$ (else consider $S_t-\mathbb{E}\left[S_T\vert\mathcal{F}_{t}\right]$).

  We have that $0=S_T=M^n_T+A^n_T$. Thus
  \begin{equation}\label{equation_DM_e1}
  M^n_T=-A^n_T.
  \end{equation}
  Since $M^n$ is a martingale it follows by optional sampling that for any $(\mathcal{F}_t)_{t\in\mathcal{D}_n}$ stopping time $\tau$
  \begin{equation}\label{equation_DM_e2}
  S_\tau=M^n_\tau+A^n_\tau = \mathbb{E}[M^n_T\vert\mathcal{F}_\tau]+A^n_\tau\stackrel{\eqref{equation_DM_e1}}{=} -\mathbb{E}[A^n_T\vert\mathcal{F}_\tau]+A^n_\tau.
  \end{equation}
  Let $c>0$. By Lemma~\ref{lem:IsStoppingTime_hittingAGT}, $\tau_n(c)$ (Definition~\ref{def:hittingAGT}) is a stopping time.
  % Define the last time when $A^n$ has always been inside $[0,c]$, by the Début Theorem \ref{thm:hitting_is_stopping_time} and the fact that $A^n$ is predictable the following is a stopping time
  % $$
  % \tau_n(c)=\inf\left(\frac{j-1}{2^n}T\vert\, A^n_{jT2^{-n}}>c\right)\wedge T.
  % $$
  By construction $A^n_{\tau_n(c)}\leq c$. It follows that
  \begin{equation}\label{equation_DM_e3}
  S_{\tau_n(c)}\stackrel{\eqref{equation_DM_e2}}{=}-\mathbb{E}[A^n_T\vert\mathcal{F}_{\tau_n(c)}]+A^n_{\tau_n(c)}\leq -\mathbb{E}[A^n_T\vert\mathcal{F}_{\tau_n(c)}]+c.
  \end{equation}
  Since $(A^n_T>c)=(\tau_n(c)<T)$ we have
  \begin{align}\nonumber
  \int_{(A^n_T>c)}A^n_TdP&=\int_{(\tau_n(c)<T)}A^n_TdP\stackrel{\mathrm{Tower}}{=}\int_{(\tau_n(c)<T)}\mathbb{E}[A^n_T\vert\mathcal{F}_{\tau_n(c)}]dP\\
  &\stackrel{\eqref{equation_DM_e3}}{\leq} cP(\tau_n(c)<T)-\int_{\tau_n(c)<T}S_{\tau_n(c)}dP.\label{equation_DM_e4}
  \end{align}
  Now we notice that $(\tau_n(c)<T)\subseteq (\tau_n(c/2)<T)$, thus
  \begin{align}\nonumber
  \int_{\tau_n(c/2)<T}-S_{\tau_n(c/2)}dP&\stackrel{\eqref{equation_DM_e2}}{=}\int_{(\tau_n(c/2))<T}\mathbb{E}[A^n_T\vert\mathcal{F}_{\tau_n(c/2)}]-A^n_{\tau_n(c/2)}dP\\
  &\stackrel{\mathrm{Tower}}{=}\int_{(\tau_n(c/2)<T)}A^n_t-A^n_{\tau_n(c/2)}dP\geq \int_{(\tau_n(c)<T)}A^n_t-A^n_{\tau_n(c/2)}dP\nonumber\\
  \intertext{(over the event $(\tau_n(c)<T)$ $A^n_T\geq c$ and $A^n_{\tau_n(c/2)}\leq c/2$, thus $A^n_T-A^n_{\tau_n(c/2)}\geq c/2$)}
  &\geq \frac{c}{2}P(\tau_n(c)<T).\label{equation_DM_e5}
  \end{align}
  It follows
  $$
  \int_{(A^n_T>c)}A^n_TdP\stackrel{\eqref{equation_DM_e4}}{\leq}cP(\tau_n(c)<T)-\int_{\tau_n(c)<T}S_{\tau_n(c)}dP\stackrel{\eqref{equation_DM_e5}}{\leq}-2\int_{\tau_n(c/2)<T}S_{\tau_n(c/2)}dP-\int_{\tau_n(c)<T}S_{\tau_n(c)}dP.
  $$
  We may notice that
  $$
  P(\tau_n(c)<T)=P(A^n_T>c)\stackrel{Markov}{\leq}\frac{\mathbb{E}[A^n_T]}{c}=-\frac{\mathbb{E}[M^n_T]}{c}\stackrel{mg}{=}-\frac{\mathbb{E}[S_0]}{c}
  $$
  which goes to $0$ uniformly in $n$ as $c$ goes to infinity.
  This implies that $\int_{(A^n_T>c)}A^n_TdP$ is uniformly bounded in $n$ due to the fact that $S$ is of class $D$. And so also the $L^1$ norm is uniformly bounded.
\end{proof}


\begin{lemma}\label{lem:M_uniform_integrable}
  The sequence $(M^n_T)_{n\in\mathbb{N}}$ is uniformly integrable (bounded in $L^1$ norm).
\end{lemma}

\begin{proof}
  \uses{lem:A_uniform_integrable}
  $M^n_T=S_T-A^n_T$, also $S$ is of class $D$ and $A^n_T$ is uniformly integrable.
\end{proof}


\begin{lemma}\label{lem:incr_fun_lim_right_cont_limsup_ineq}
  If $f_n, f : [0, 1] \rightarrow \mathbb{R}$ are increasing functions such that $f$ is right continuous and
  $\lim_n f_n(t) = f (t)$ for $t \in\mathcal{D}^T$, then  $\limsup_n  f_n(t) \leq f (t)$ for all $t \in [0, T]$.
\end{lemma}

\begin{proof}
  Let $t\in[0,T]$ and $s\in\mathcal{D}^T$ such that $t<s$. We have
  $$
  \limsup_n f_n(t)\leq \limsup_n f_n(s)=f(s).
  $$
  Since the above is true uniformly in $s$ in particular since $f$ is right-continuous
  $$
  \limsup_n f_n(t)\leq\lim_{\stackrel{s\rightarrow t^+}{s\in\mathcal{D}^T}}f(s)=f(t).
  $$
\end{proof}


\begin{lemma}\label{lem:incr_fun_lim_right_cont_lim_eq}
  If $f_n, f : [0, 1] \rightarrow \mathbb{R}$ are increasing functions such that $f$ is right continuous and
  $\lim_n f_n(t) = f (t)$ for $t \in\mathcal{D^T}$, if $f$ is continuous in $t\in[0,T]$ then $\lim_n  f_n(t) = f (t)$.
\end{lemma}

\begin{proof}
  \uses{lem:incr_fun_lim_right_cont_limsup_ineq}
  By lemma \ref{lem:incr_fun_lim_right_cont_limsup_ineq} it is enough to show that $\liminf_n f_n(t)\geq f(t)$.
  Let $s\in\mathcal{D}^T$ such that $t>s$. We have
  $$
  \liminf_n f_n(t)\geq \liminf_n f_n(s)=f(s).
  $$
  Since the above is true uniformly in $s$ in particular since $f$ is continuous in $t$
  $$
  \liminf_n f_n(t)\geq\lim_{\stackrel{s\rightarrow t^-}{s\in\mathcal{D}^T}}f(s)=f(t).
$$
\end{proof}

Define $M^n_t$ on $[0,T]$ using $M^n_t=\mathbb{E}[M^n_T\vert\mathcal{F}_t]$.

\begin{lemma}\label{lem:M_n_cadlag_mg}
  $M^n_t$ admits a modification which is a cadlag martingale.
\end{lemma}

\begin{proof}
  \uses{lem:mg_is_cadlag}
  By theorem \ref{lem:mg_is_cadlag}
\end{proof}

From this point onwards $M^n_t$ will be redefined as the modification from lemma \ref{lem:M_n_cadlag_mg}.

\begin{lemma}\label{lem:M_cal_converges_L1}
  There are convex weights $\lambda^n_n,\cdots,\lambda^n_{N_n}$ such that
  $\mathcal{M}^n_T\stackrel{L^1}{\rightarrow}M$, where $\mathcal{M}^n:=\lambda^n_nM^n+\cdots +\lambda^n_{N_n}M^{N_n}.$
\end{lemma}
\begin{proof}
  \uses{lem:M_uniform_integrable,lem:komlos}
  By lemma \ref{lem:M_uniform_integrable} $(M^n_T)_{n\in\mathbb{N}}$ is uniformly bounded in $L^1$, thus by lemma \ref{lem:komlos} there are convex weights $\lambda^n_n,\cdots,\lambda^n_{N_n}$ such that
  $\mathcal{M}^n_T\stackrel{L^1}{\rightarrow}M$, where $\mathcal{M}^n:=\lambda^n_nM^n+\cdots +\lambda^n_{N_n}M^{N_n}.$
\end{proof}

\begin{lemma}\label{lem:M_cal_cadlag}
  $\mathcal{M}^n$ is cadlag.
\end{lemma}
\begin{proof}
  \uses{lem:M_n_cadlag_mg,lem:M_cal_converges_L1}
  By construction and \ref{lem:M_n_cadlag_mg}
\end{proof}

Let \begin{equation}\label{equation_DM_e6} M_t = \mathbb{E}[M\vert\mathcal{F}_t].\end{equation}

\begin{lemma}\label{lem:M_cadlag_mg}
  $M_t$ admits a martingale cadlag modification.
\end{lemma}
\begin{proof}
  \uses{lem:M_cal_converges_L1, lem:mg_is_cadlag}
  By construction $M_t$ is a martingale and thus by theorem \ref{lem:mg_is_cadlag} admits a cadlag martingale modification
  ($M_t$ is a version of $\mathbb{E}[M\vert\mathcal{F}_t]$ and thus passing to modification does not pose any problem).
\end{proof}

From this point onwards $M^n_t$ will be redefined as the modification from lemma \ref{lem:M_cadlag_mg}.
Define
\begin{itemize}
  \item Extend now $A^n$ as a left continuous process $A^n_s:=\sum_{t\in\mathcal{D}^T_n}A^n_t\mathbb{1}_{]t-2^{-n},t]}(s)$
  \item $\mathcal{A}^n=\lambda^n_nA^n+\cdots +\lambda^n_{N_n}A^{N_n}$
  \item $A_t=S_t-M_t$
\end{itemize}

\begin{lemma}\label{lem:M1_komlos}
  For every $t\in[0,T]$ we have $\mathcal{M}^n_t\stackrel{L^1}{\rightarrow}M_t$.
\end{lemma}
\begin{proof}
  \uses{lem:M_cal_converges_L1}
  We may notice that by Jensen's inequality, the tower lemma and lemma \ref{lem:M_cal_converges_L1}
  \begin{gather}\nonumber
    \mathbb{E}[|\mathcal{M}^n_t-M_t|]=\mathbb{E}[|\mathbb{E}[\mathcal{M}^n_T-M\vert\mathcal{F}_t]|]\leq \mathbb{E}[|\mathcal{M}^n_T-M|]\rightarrow0,\\
    \Rightarrow\mathcal{M}^n_t\stackrel{L^1}{\rightarrow} M_t,\quad \forall t\in[0,T].\label{equation_DM_e7}
  \end{gather}
\end{proof}

\begin{lemma}\label{lem:A_cal_conv_A_on_D_T}
  There exists a set $E\subseteq\Omega$, $P(E)=0$ and a subsequence $k_n$ such that $\lim_n\mathcal{A}^{k_n}_t(\omega)=A_t(\omega)$ for every $t\in\mathcal{D}^T,\omega\in\Omega\setminus E$.
\end{lemma}
\begin{proof}
  \uses{lem:M1_komlos}
  By Lemma \ref{lem:M1_komlos}
  $$
  \mathcal{A}^n_t=S_t-\mathcal{M}^n_t\stackrel{L^1}{\rightarrow}S_t-M_t=A_t,\quad\forall t\in\mathcal{D}^T.
  $$
  $\mathcal{D}^T$ is countable we can arrange the elements as $(t_n)_{n\in\mathbb{N}}$.
  Given $t_0\in\mathcal{D}^T$ there exists a subsequence $k^{0}_n$ for which $\mathcal{A}^{k^{0}_n}_{t_0}$ converges to $A_{t_0}$ over the set $\Omega\setminus E_{0}$ where $P(E_{0})=0$.
  Suppose we have a sequence $k^m_n$ for which $\mathcal{A}^{k^j_n}_{t_j}$ converges to $A_{t_j}$ over the set $\Omega\setminus E_{m}$ where $P(E_{m})=0$ for each $j=0,\cdots,m$.
  From this subsequence we may extract a new subsequence $k^{m+1}_n$ for which $\mathcal{A}^{k^{m+1}_n}_{t_{m+1}}$ converges to $A_{t_{m+1}}$ over the set $\Omega\setminus E_{m+1}$ where $P(E_{m+1})=0$.
  By construction over this subsequence the convergence for $t_0,\cdots,t_m$ still applies.
  With a diagonal argument we obtain the final result with $E=\bigcup_n E_n$.
\end{proof}

\begin{lemma}\label{lem:A_increasing}
  $(A_t)_{t\in[0,T]}$ is an increasing process.
\end{lemma}
\begin{proof}
  \uses{lem:A_cal_conv_A_on_D_T, lem:Predict_Part_Increasing, lem:M_cadlag_mg}
  Since $\mathcal{A}^n_t$ is increasing on $\mathcal{D}^T$ by lemma \ref{lem:A_cal_conv_A_on_D_T} also $A$ is almost surely increasing on $\mathcal{D}^T$.
  Since $S,M$ are cadlag also $A$ is cadlag (thus right-continuous). It follows that $A$ must be increasing on $[0,T]$.
\end{proof}

\begin{lemma}\label{lem:lim_Exp_A_n_tau_is_Exp_A_tau}
  Let $\tau$ be an $(\mathcal{F}_t)_{t\in[0,T]}$ stopping time. We have $\lim_n\mathbb{E}[A^n_\tau]=\mathbb{E}[A_\tau]$.
\end{lemma}
\begin{proof}
  \uses{lem:M_cadlag_mg, lem:M_n_cadlag_mg}
  Let $\sigma_n:=\inf\left(t\in\mathcal{D}^T_n\vert t>\tau\right)$. By construction of $A^n$ we have $A^n_\tau=A^n_{\sigma_n}$.
  Also $\sigma_n\searrow\tau$. Since $S$ is of class $D$ and cadlag we have
  \begin{align*}
    \mathbb{E}[A^n_\tau]&=\mathbb{E}[A^n_{\sigma_n}]=\mathbb{E}[S_{\sigma_n}]-\mathbb{E}[M^n_{\sigma_n}]=\mathbb{E}[S_{\sigma_n}]-\mathbb{E}[M^n_0]=\\
    &=\mathbb{E}[S_{\sigma_n}]-\mathbb{E}[S_0]\rightarrow \mathbb{E}[S_\tau]-\mathbb{E}[M_0]=\mathbb{E}[S_\tau]-\mathbb{E}[M_\tau]=\mathbb{E}[A_\tau].
  \end{align*}
\end{proof}

\begin{lemma}\label{lem:limsup_A_n_tau_is_A_tau_ae}
  Let $\tau$ be an $(\mathcal{F}_t)_{t\in[0,T]}$ stopping time. We have $\limsup_n \mathcal{A}_\tau^n = A_\tau$.
\end{lemma}
\begin{proof}
  \uses{lem:lim_Exp_A_n_tau_is_Exp_A_tau, lem:incr_fun_lim_right_cont_limsup_ineq, lem:Predict_Part_Increasing, lem:A_cal_conv_A_on_D_T}
  Firstly we notice that $\liminf_n \mathbb{E}[A_\tau^n]  \leq \limsup_n  \mathbb{E}  [\mathcal{A}_\tau^n  ]  \leq \mathbb{E}[\limsup_n  \mathcal{A}_\tau^n  ]  \leq \mathbb{E}[ A_\tau ]$,
  where the first inequality is justified by the definition of limsup and liminf and the fact that
  $$
  \sup_{k\geq n}\mathbb{E}[\mathcal{A}^k_\tau]\geq \sum_{m=k}^{N_k}\lambda^k_m\mathbb{E}[A^m_\tau]\geq \sum_{m=k}^{N_k}\lambda^k_m\inf_{j\geq n}\mathbb{E}[A^j_\tau]=\inf_{k\geq n}\mathbb{E}[A^k_\tau]
  $$
  the third inequality by \ref{lem:incr_fun_lim_right_cont_limsup_ineq}.
  Let's prove the second inequality: observe that
  $$
  \mathcal{A}^n_\tau= A_1+\mathcal{A}^n_\tau-A_1\leq A_1+(\mathcal{A}^n_\tau-A_1)_+,
  $$
  thus it follows that $\mathcal{A}^n_\tau - (\mathcal{A}^n_\tau-A_1)_+\leq A_1$; since $A_1$ is an integrable guardian the inverse Fatou Lemma may be applied to show together with limsup properties that
  \begin{align*}
    \limsup_n\mathbb{E}[\mathcal{A}^n_\tau]+0 &= \limsup_n\mathbb{E}[\mathcal{A}^n_\tau]+\liminf_n-\mathbb{E}[(\mathcal{A}^n_\tau-A_1)_+] \leq \limsup_n\mathbb{E}[\mathcal{A}^n_\tau-(\mathcal{A}^n_\tau-A_1)_+]\leq\\
    &\leq \mathbb{E}[\limsup_n\mathcal{A}^n_\tau-(\mathcal{A}^n_\tau-A_1)_+]\leq \mathbb{E}[\limsup_n\mathcal{A}^n_\tau]-\mathbb{E}[\liminf_n(\mathcal{A}^n_\tau-A_1)_+]\leq\mathbb{E}[\limsup_n\mathcal{A}^n_\tau],
    \end{align*}
  where the first equality is justified by the fact that $\mathcal{A}^n_\tau\leq\mathcal{A}^n_1\rightarrow A_1$ almost surely.
  Due to lemma \ref{lem:lim_Exp_A_n_tau_is_Exp_A_tau} and \ref{lem:incr_fun_lim_right_cont_limsup_ineq} the first sequence of inequalities is a sequence of equalities, thus
  we know that $A_\tau- \limsup_n \mathcal{A}_\tau^n $ is an a.s. nonnegative function with null expected value, and thus it must be almost everywhere null.
\end{proof}

\begin{theorem}\label{thm:Doob_Meyer}
  Let $S = (S_t )_{0\leq t\leq T}$ be a cadlag submartingale of class $D$.
  Then, $S$ can be written in a unique way in the form  $S = M + A$ where $M$ is a cadlag martingale and $A$ is a predictable increasing process starting at $0$.
\end{theorem}
\begin{proof}
  \uses{lem:A_increasing, lem:M_cadlag_mg, lem:limsup_A_n_tau_is_A_tau_ae, lem:incr_fun_lim_right_cont_lim_eq}
  By construction $M$ is a cadlag martingale and $A_0=0$ and by lemma \ref{lem:A_increasing} $A$ is increasing. It suffices to show that $A$ is predictable.
  $A^n,\mathcal{A}^n$ are left continuous and adapted, and thus they are predictable (measurable wrt the predictable sigma algebra (the one generated by left-cont adapted processes)).
  It is enough to show that $\omega-a.e.$, $\forall t\in[0,T]$, $\limsup_n\mathcal{A}^n_t(\omega)=A_t(\omega)$.

  By lemma \ref{lem:incr_fun_lim_right_cont_lim_eq} that is true for any continuity point of $A$. Since $A$ is increasing it can only have a finite amount of jumps larger than $1/k$ for any $k\in\mathbb{N}$.
  Consider now $\tau_{q,k}$ the family of stopping times equal to the $q$-th time that the process $A_t$ has a jump higher than $1/k$. This is a countable family.
  Given a time $t$ and a trajectory $\omega$ there are only two possibilities: either $A$ is continuous or not at time $t$ along $\omega$.
  If $A$ is continuous at time $t$ we have $\limsup_n\mathcal{A}^n_t(\omega)=A_t(\omega)$, if it jumps there exists $q(\omega),k(\omega)$ such that $t=\tau_{q(\omega),k(\omega)}(\omega)$.
  Due to lemma \ref{lem:limsup_A_n_tau_is_A_tau_ae} we know that $\limsup_n A^n_{\tau_{q,k}} = A_{\tau_{q,k}}$ for each $q,k$ almost surely. Thus, since it is an intersection of a countable amount of almost sure
  events $\forall\omega\in\Omega'$ with $P(\Omega')=1$, for each $q,k$ $\limsup_n A^n_{\tau_{q,k}}(\omega) = A_{\tau_{q,k}}(\omega)$ ($\omega$ does not depend upon $q,k$).
  Consequently, $\forall\omega\in\Omega'$ we have $\limsup_n\mathcal{A}^n_t(\omega)=\limsup_n\mathcal{A}^n_{\tau_{q(\omega),k(\omega)}}(\omega)=A_{\tau_{q(\omega),k(\omega)}}(\omega)=A_t(\omega)$
\end{proof}

\chapter{Stochastic integral}

The lecture notes at \href{https://dec41.user.srcf.net/h/III_L/stochastic_calculus_and_applications/}{this link} as well as chapter 18 of \cite{kallenberg2021} are good references for this chapter.

\section{Total variation and Lebesgue-Stieltjes integral}

TODO: in Mathlib, we can integrate with respect to the measure given by a right-continuous monotone function (\texttt{StieltjesFunction.measure}). This will be useful to integrate against the quadratic variation of a local martingale.
However, we will also want to integrate with respect to a signed measure given by a càdlàg function with finite variation.
We need to investigate what's already in Mathlib. See \texttt{Mathlib.Topology.EMetricSpace.BoundedVariation}.



\section{Square integrable martingales}

In this section, $E$ denotes a complete normed space.

First, recall the definitions of a martingale, a stopping time and a stopped process, which are already in Mathlib.


\begin{definition}[Martingale]\label{def:Martingale}
  \mathlibok
  \lean{MeasureTheory.Martingale}
Let $\mathcal{F}$ be a filtration on a measurable space $\Omega$ with measure $P$ indexed by $T$.
A family of functions $M : T \to \Omega \to E$ is a martingale with respect to a filtration $\mathcal{F}$ if $M$ is adapted with respect to $\mathcal{F}$ and for all $i \le j$, $P[M_j \mid \mathcal{F}_i] = M_i$ almost surely.
\end{definition}


\begin{definition}[Stopping time]\label{def:IsStoppingTime}
  \mathlibok
  \lean{MeasureTheory.IsStoppingTime}
A stopping time with respect to some filtration $\mathcal{F}$ indexed by $T$ is a function $\tau : \Omega \to T$ such that for all $i$, the preimage of $\{j \mid j \le i\}$ along $\tau$ is measurable with respect to $\mathcal{F}_i$.
\end{definition}


\begin{definition}[Stopped process]\label{def:stoppedProcess}
  \mathlibok
  \lean{MeasureTheory.stoppedProcess}
Let $X : T \to \Omega \to E$ be a stochastic process and let $\tau : \Omega \to T$.
The stopped process with respect to $\tau$ is defined by
\begin{align*}
  (X^{\tau})_t = \begin{cases}
    X_t & \text{if } t \le \tau \\
    X_{\tau} & \text{otherwise}
  \end{cases}
\end{align*}
\end{definition}


\begin{theorem}[Doob's Lp inequality]\label{thm:doob_lp}
  \uses{def:Martingale}
Let $M : T \to \Omega \to \mathbb{R}$ be a martingale indexed by a countable index set $T$ and let $p, q > 1$ such that $\frac{1}{p} + \frac{1}{q} = 1$.
Then for all $t \in T$,
\begin{align*}
  \Vert \sup_{s \le t} \vert M_s \vert \Vert_p
  \le q \Vert M_t \Vert_p
  \: .
\end{align*}
\end{theorem}

\begin{proof}

\end{proof}

TODO: corollary: if $M$ is continuous we have the same inequality on $\mathbb{R}_+$.


\begin{definition}[Square integrable martingales]\label{def:squareIntegrableMartingales}
  \uses{def:Martingale}
Let $\mathcal{M}^2$ be the set of square integrable continuous martingales with respect to a filtration $\mathcal{F}$ indexed by $\mathbb{R}_+$,
\begin{align*}
  \mathcal{M}^2
  = \{ M : \mathbb{R}_+ \to \Omega \to \mathbb{R} \mid M \text{ continuous martingale with } \sup_{t}\mathbb{E}[M_t^2] < \infty \}
  \: .
\end{align*}
\end{definition}


\begin{theorem}\label{thm:hilbertSpace_squareIntegrableMartingales}
  \uses{def:squareIntegrableMartingales}
The space $\mathcal{M}^2$ is a Hilbert space with the inner product defined by
\begin{align*}
  \langle M, N \rangle = \mathbb{E}[M_\infty N_\infty]
  \: .
\end{align*}
\end{theorem}

\begin{proof}
  \uses{thm:doob_lp}

\end{proof}


\section{Local martingales}

TODO: filtrations should be assumed right-continuous and complete whenever needed.

\begin{definition}[Local martingale]\label{def:IsLocalMartingale}
  \uses{def:Martingale, def:IsStoppingTime, def:stoppedProcess}
Let $\mathcal{F} = (\mathcal{F}_t)_{t \in \mathbb{R}_+}$ be a filtration on a measurable space $\Omega$.
A local martingale with respect to $\mathcal{F}$ is a stochastic process $M : \mathbb{R}_+ \to \Omega \to E$ adapted to $\mathcal{F}$ such that there exists a localizing sequence $(\tau_n)_{n \in \mathbb{N}}$ such that the following conditions hold:
\begin{itemize}
  \item $\tau_n$ is a stopping time for every $n \in \mathbb{N}$,
  \item $\tau_n$ is non-decreasing and $\tau_n \to \infty$ as $n \to \infty$ (a.s.),
  \item for all $n \in \mathbb{N}$, the stopped and centered process $M^{\tau_n} - M_0$ is a martingale with respect to $\mathcal{F}$.
\end{itemize}
\end{definition}


\begin{lemma}\label{lem:Martingale.IsLocalMartingale}
  \uses{def:IsLocalMartingale}
Every martingale is a local martingale.
\end{lemma}

\begin{proof}

\end{proof}


\begin{theorem}\label{thm:IsLocalMartingale.eq_zero_of_finiteVariation}
  \uses{def:IsLocalMartingale}
Let $M$ be a continuous local martingale with $M_0 = 0$. If $M$ is also a finite variation process, then $M_t = 0$ for all $t$.
\end{theorem}

\begin{proof}

\end{proof}


\begin{definition}[Quadratic variation]\label{def:quadraticVariation}
  \uses{def:IsLocalMartingale}
For any continuous local martingale $M$, there exists a continuous process $[M]$ with $[M]_0 = 0$ such that $M^2 - [M]$ is a local martingale. That process is a.s. unique and is called the \emph{quadratic variation} of $M$.
\end{definition}


\begin{definition}[Covariation]\label{def:covariation}
  \uses{def:IsLocalMartingale}
For any continuous local martingales $M$ and $N$, there exists a continuous process $[M,N]$ with $[M,N]_0 = 0$ such that $MN - [M,N]$ is a local martingale. That process is a.s. unique and is called the \emph{covariation} of $M$ and $N$.

It can be defined by $[M, N]_t = \frac{1}{4}\left([M+N]_t - [M-N]_t \right)$~.
\end{definition}


\begin{lemma}\label{lem:covariation_eq_inner}
  \uses{def:covariation, def:squareIntegrableMartingales}
Let $M$ and $N$ be continuous square integrable martingales. Then
\begin{align*}
  [M,N]_\infty = \langle M, N \rangle_{\mathcal{M}^2}
  \: .
\end{align*}
\end{lemma}

\begin{proof}

\end{proof}


\begin{lemma}\label{lem:quadraticVariation_brownian}
  \uses{def:brownian, def:quadraticVariation}
Let $B$ be a standard Brownian motion. Then the quadratic variation of $B$ is given by $[B]_t = t$~.
\end{lemma}

\begin{proof}

\end{proof}


\begin{definition}[Continuous semi-martingale]\label{def:continuousSemiMartingale}
  \uses{def:IsLocalMartingale}
A continuous semi-martingale is a process that can be decomposed into a local martingale and a finite variation process.
More formally, a process $X : \mathbb{R}_+ \to \Omega \to E$ is a continuous semi-martingale if there exists a continuous local martingale $M$ and a continuous adapted process $A$ with locally finite variation and $A_0 = 0$ such that
\begin{align*}
  X_t = M_t + A_t
\end{align*}
for all $t \ge 0$.
The decomposition is a.s. unique.
\end{definition}


\section{Stochastic integral}


\begin{definition}[Simple process]\label{def:simpleProcess}
Let $0 \le t_0 \le t_1 \le \ldots \le t_n$ in $\mathbb{R}_+$.
Let $(\eta_k)_{0 \le k \le n-1}$ be $\mathcal{F}_{t_k}$-measurable random variables.
Then the simple process for that sequence is the process $V : \mathbb{R}_+ \to \Omega \to E$ defined by
\begin{align*}
  V_t = \sum_{k=0}^{n-1} \eta_k \mathbb{1}_{(t_k, t_{k+1}]}(t)
  \: .
\end{align*}
Let $\mathcal{E}$ be the set of simple processes.
\end{definition}


\begin{definition}[Elementary stochastic integral]\label{def:elemStochIntegral}
  \uses{def:simpleProcess}
Let $V \in \mathcal{E}$ be a simple process and let $X$ be a stochastic process.
The \emph{elementary stochastic integral} process $V \cdot X : \mathbb{R}_+ \to \Omega \to E$ is defined by
\begin{align*}
  (V \cdot X)_t
  &= \sum_{k=0}^{n-1} \eta_k (X^t_{t_{k+1}} - X^t_{t_k})
  \: .
\end{align*}
\end{definition}


\begin{lemma}\label{lem:sq_norm_elemStochIntegral}
  \uses{def:elemStochIntegral}
For $V \in \mathcal{E}$ and $M \in \mathcal{M}^2$, then $V \cdot M \in \mathcal{M}^2$ and
\begin{align*}
  \Vert V \cdot M \Vert_{\mathcal{M}^2}^2
  &= \mathbb{E}\left[ \int_0^{\infty} V_t^2 \: d[M]_t \right]
  \: .
\end{align*}
\end{lemma}

\begin{proof}

\end{proof}


\subsection{Itô isometry}

\begin{definition}\label{def:L2M}
  \uses{def:squareIntegrableMartingales}
Let $M \in \mathcal{M}^2$ be a continuous square integrable martingale. We define
\begin{align*}
  L^2(M) = L^2(\Omega \times \mathbb{R}_+, \mathcal{P}, \mathbb{P} \times d[M])
\end{align*}
in which $\mathcal{P}$ is the predictable $\sigma$-algebra and $d[M]$ is the measure induced by the quadratic variation of $M$.
The norm on that Hilbert space is $\Vert X \Vert^2 = \mathbb{E}\left[ \int_0^{\infty} X_t^2 \: d[M]_t \right]$~.
\end{definition}

TODO the sources don't use the same assumptions: predictable vs progressive (\texttt{MeasureTheory.ProgMeasurable}). Progressive would be more general.


\begin{lemma}\label{lem:dense_simpleProcess}
  \uses{def:L2M, def:simpleProcess}
Let $M \in \mathcal{M}^2$. Then the set of simple processes is dense in $L^2(M)$.
\end{lemma}

\begin{proof}

\end{proof}


\begin{definition}[Itô isometry]\label{def:itoIsometry}
  \uses{lem:dense_simpleProcess, lem:sq_norm_elemStochIntegral, thm:hilbertSpace_squareIntegrableMartingales}
Let $M \in \mathcal{M}^2$. Then the elementary stochastic integral map $\mathcal{E} \to \mathcal{M}^2$ defined by $V \mapsto V \cdot M$ extends to an isometry $L^2(M) \to \mathcal{M}^2$.
\end{definition}


\begin{lemma}\label{lem:inner_itoIsometry}
  \uses{def:itoIsometry}
$\langle X \cdot M, Y \cdot M \rangle_{\mathcal{M}^2} = (XY) \cdot \langle M, N \rangle_{\mathcal{M}^2}$.
\end{lemma}

\begin{proof}

\end{proof}


\subsection{Local martingales}

\begin{definition}[$L^2_{loc}(M)$]\label{def:L2locM}
  \uses{def:L2M}
Let $M$ be a continuous local martingale.
We define $L^2_{loc}(M)$ as the space of predictable processes $X$ such that for all $t \ge 0$, $\mathbb{E}\left[ \int_0^t X_s^2 \: d[M]_s \right] < \infty$.
\end{definition}


\begin{definition}[Stochastic integral for continuous local martingales]\label{def:locStochIntegral}
  \uses{def:L2locM, def:itoIsometry}
Let $M$ be a continuous local martingale and let $X \in L^2_{loc}(M)$.
We define the local stochastic integral $X \cdot M$ as the unique continuous local martingale with $(X \cdot M)_0 = 0$ such that for any continuous local martingale $N$, almost surely,
\begin{align*}
  [X \cdot M, N] = X \cdot [M, N]
  \: .
\end{align*}
\end{definition}


\subsection{Semi-martingales}

\begin{definition}\label{def:stochIntegral}
  \uses{def:continuousSemiMartingale, def:locStochIntegral}
For a continuous semi-martingale $X = M + A$ and $V \in L^2_{semi}(X)$ (to be defined) we define the stochastic integral as
\begin{align*}
  V \cdot X = V \cdot M + V \cdot A
  \: ,
\end{align*}
in which $V \cdot M$ is the local stochastic integral defined in \ref{def:locStochIntegral} and $V \cdot A$ is the Lebesgue-Stieltjes integral with respect to the locally finite variation process $A$.
\end{definition}


For $X = M + A$ and $Y = N + B$, we define the covariation as
\begin{align*}
  [X, Y] = [M, N]
  \: .
\end{align*}

\section{Itô formula}


\begin{theorem}[Integration by parts]\label{thm:integration_by_parts}
  \uses{def:continuousSemiMartingale, def:stochIntegral}
Let $X$ and $Y$ be two continuous semi-martingales. Then we have almost surely
\begin{align*}
  X_t Y_t - X_0 Y_0
  = (X \cdot Y)_t + (Y \cdot X)_t + [X,Y]_t
  \: .
\end{align*}
\end{theorem}

\begin{proof}

\end{proof}


\begin{theorem}[Itô's formula]\label{thm:Ito_formula}
  \uses{def:continuousSemiMartingale}
Let $X^1, \ldots, X^d$ be continuous semi-martingales and let $f : \mathbb{R}^d \to \mathbb{R}$ be a twice continuously differentiable function.
Then, writing $X = (X^1, \ldots, X^d)$, the process $f(X)$ is a semi-martingale and we have
\begin{align*}
  f(X_t)
  &= f(X_0)
  + \sum_{i=1}^d \int_0^t \frac{\partial f}{\partial x_i}(X_s) \: dX^i_s
  + \frac{1}{2} \sum_{i,j=1}^d \int_0^t \frac{\partial^2 f}{\partial x_i \partial x_j}(X_s) \: d[X^i, X^j]_s
  \: .
\end{align*}
\end{theorem}

\begin{proof}
  \uses{thm:integration_by_parts}

\end{proof}


\putbib
