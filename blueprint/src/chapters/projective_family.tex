\chapter{Projective family of the Brownian motion}
\label{chap:projective_family}


\begin{definition}[Projective family]\label{def:IsProjectiveMeasureFamily}
  \mathlibok
  \lean{MeasureTheory.IsProjectiveMeasureFamily}
A family of measures $P$ indexed by finite sets of $T$ is projective if, for finite sets $J \subseteq I$, the projection from $E^I$ to $E^J$ maps $P_I$ to $P_J$.
\end{definition}


\begin{definition}[Projective limit]\label{def:IsProjectiveLimit}
  \uses{def:IsProjectiveMeasureFamily}
  \mathlibok
  \lean{MeasureTheory.IsProjectiveLimit}
A measure $\mu$ on $E^T$ is the projective limit of a projective family of measures $P$ indexed by finite sets of $T$ if, for every finite set $I \subseteq T$, the projection from $E^T$ to $E^I$ maps $\mu$ to $P_I$.
\end{definition}


\begin{theorem}[Kolmogorov extension theorem]\label{thm:kolmogorovExtension}
  \uses{def:IsProjectiveLimit, def:IsProjectiveMeasureFamily}
  \leanok
  \lean{MeasureTheory.projectiveLimit, MeasureTheory.IsProjectiveLimit.unique, MeasureTheory.isProjectiveLimit_projectiveLimit, MeasureTheory.isFiniteMeasure_projectiveLimit, MeasureTheory.isProbabilityMeasure_projectiveLimit}
Let $\mathcal{X}$ be a Polish space, equipped with the Borel $\sigma$-algebra, and let $T$ be an index set.
Let $P$ be a projective family of finite measures on $\mathcal{X}$.
Then the projective limit $\mu$ of $P$ exists, is unique, and is a finite measure on $\mathcal{X}^T$.
Moreover, if $P_I$ is a probability measure for every finite set $I \subseteq T$, then $\mu$ is a probability measure.
\end{theorem}

\begin{proof}\leanok

\end{proof}


\begin{lemma}\label{lem:posSemidef_brownianCov}
For $I = \{t_1, \ldots, t_n\}$ a finite subset of $\mathbb{R}_+$, let $C \in \mathbb{R}^{n \times n}$ be the matrix $C_{ij} = \min(t_i, t_j)$ for $1 \leq i,j \leq n$.
Then $C$ is positive semidefinite.
\end{lemma}

\begin{proof}

\end{proof}


\begin{definition}[Projective family of the Brownian motion]\label{def:gaussianProjectiveFamily}
  \uses{def:multivariateGaussian, lem:posSemidef_brownianCov}
For $I = \{t_1, \ldots, t_n\}$ a finite subset of $\mathbb{R}_+$, let $P^B_I$ be the multivariate Gaussian measure on $\mathbb{R}^n$ with mean $0$ and covariance matrix $C_{ij} = \min(t_i, t_j)$ for $1 \leq i,j \leq n$.
We call the family of measures $P^B_I$ the \emph{projective family of the Brownian motion}.
\end{definition}


\begin{lemma}\label{lem:isProjectiveMeasureFamily_gaussianProjectiveFamily}
  \uses{def:gaussianProjectiveFamily, def:IsProjectiveMeasureFamily}
The projective family of the Brownian motion is a projective family of measures.
\end{lemma}

\begin{proof}

\end{proof}


\begin{definition}\label{def:gaussianLimit}
  \uses{thm:kolmogorovExtension, lem:isProjectiveMeasureFamily_gaussianProjectiveFamily}
We denote by TODO the projective limit of the projective family of the Brownian motion given by Theorem~\ref{thm:kolmogorovExtension}.
This is a probability measure on $\mathbb{R}^{\mathbb{R}_+}$.
\end{definition}


% \begin{definition}\label{def:}
% Let $\Omega = \mathbb{R}^{\mathbb{R}_+}$ and consider the probability space $(\Omega, TODO)$.
% The identity on that space is a function $\Omega \to \mathbb{R}_+ \to \mathbb{R}$.
% We can reorder the arguments to define a process $X : \mathbb{R}_+ \to \Omega \to \mathbb{R}$.
% That process is a Gaussian process with covariance function $C(t,s) = \min(t,s)$.
% \end{definition}
