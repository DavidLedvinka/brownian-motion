\chapter{Kolmogorov-Chentsov Theorem}
\label{chap:kolmogorov_chentsov}

\section{Covers}

Let $(E, d_E)$ be a pseudo-metric space.

\begin{definition}[$\varepsilon$-cover]\label{def:IsCover}
  \leanok
  \lean{IsCover}
  A set $C \subseteq E$ is an $\varepsilon$-cover of a set $A \subseteq E$ if for every $x \in A$, there exists $y \in C$ such that $d_E(x, y) < \varepsilon$.
\end{definition}


\begin{definition}[External covering number]\label{def:externalCoveringNumber}
  \uses{def:IsCover}
  \leanok
  \lean{externalCoveringNumber}
  The external covering number of a set $A \subseteq E$ for $\varepsilon \ge 0$ is the smallest cardinality of an $\varepsilon$-cover of $A$.
  Denote it by $N^{ext}_\varepsilon(A)$.
\end{definition}


\begin{definition}[Internal covering number]\label{def:internalCoveringNumber}
  \uses{def:IsCover}
  \leanok
  \lean{internalCoveringNumber}
  The internal covering number of a set $A \subseteq E$ for $\varepsilon \ge 0$ is the smallest cardinality of an $\varepsilon$-cover of $A$ which is a subset of $A$.
  Denote it by $N^{int}_\varepsilon(A)$.
\end{definition}


\begin{lemma}\label{lem:externalCoveringNumber_le_internalCoveringNumber}
  \uses{def:externalCoveringNumber, def:internalCoveringNumber}
  \leanok
  \lean{externalCoveringNumber_le_internalCoveringNumber}
$N^{ext}_\varepsilon(A) \le N^{int}_\varepsilon(A)$.
\end{lemma}

\begin{proof}\leanok

\end{proof}


\begin{definition}[Bounded internal covering number]\label{def:HasBoundedInternalCoveringNumber}
  \uses{def:internalCoveringNumber}
  Let $\mathrm{diam}(A)$ be the diameter of $A \subseteq E$, i.e. $\mathrm{diam}(A) = \sup_{x,y \in A} d_E(x, y)$.
  A set $A \subseteq E$ has bounded internal covering number with constant $c>0$ and exponent $t>0$ if for all $\varepsilon \in (0, \mathrm{diam}(A))$, $N^{int}_\varepsilon(A) \le c \varepsilon^{-t}$.
\end{definition}


\begin{lemma}\label{lem:hasBoundedInternalCoveringNumber_unitInterval}
  \uses{def:HasBoundedInternalCoveringNumber}
The unit interval $I = [0, 1] \subseteq \mathbb{R}$ has bounded internal covering number with constant $1$ and exponent $1$: for $\varepsilon \le 1$, $N^{int}_\varepsilon(I) \le 1/\varepsilon$.
\end{lemma}

\begin{proof}

\end{proof}


\section{Chaining}

\subsection{Chaining sequence}


\begin{definition}\label{def:nearestPt}
Let $S$ be a finite set of $E$ and $x \in E$.
We denote by $\pi(x, S)$ the point in $S$ which is closest to $x$, i.e. a point such that $d_E(x, S) = \min_{y \in S} d_E(x, y)$ (chosen arbitrarily among the minima if there are several).
\end{definition}


\begin{lemma}\label{lem:dist_nearestPt_le}
  \uses{def:nearestPt}
Let $S$ be a finite set of $E$ and $x \in E$.
Then for all $y \in S$, $d_E(x, \pi(x, S)) \le d_E(x, y)$.
\end{lemma}

\begin{proof}
By definition.
\end{proof}


\begin{lemma}\label{lem:dist_nearestPt_of_isCover}
  \uses{def:nearestPt, def:IsCover}
Let $C_\varepsilon$ be a finite $\varepsilon$-cover of $A \subseteq E$ (assuming such a finite cover exists).
Then for all $x \in A$, $d_E(x, \pi(x, C_\varepsilon)) \le \varepsilon$.
\end{lemma}

\begin{proof}

\end{proof}


\begin{definition}[Chaining sequence]\label{def:chainingSequence}
  \uses{def:nearestPt, def:IsCover}
Let $(\varepsilon_n)_{n \in \mathbb{N}}$ be a sequence of positive numbers, $C_n$ a finite $\varepsilon_n$-cover of $A \subseteq E$ with $C_n \subseteq A$ and $x \in C_k$ for some $k \in \mathbb{N}$.
We define the chaining sequence of $x$, denoted $(\bar{x}_i)_{i \le k}$, recursively as follows: $\bar{x}_k = x$ and for $i < k$, $\bar{x}_i = \pi(\bar{x}_{i+1}, C_i)$.
\end{definition}


\begin{lemma}\label{lem:dist_chainingSequence_add_one}
  \uses{def:chainingSequence}
Let $(\varepsilon_n)_{n \in \mathbb{N}}$ be a sequence of positive numbers, $C_n$ a finite $\varepsilon_n$-cover of $A \subseteq E$ with $C_n \subseteq A$ and $x \in C_k$ for some $k \in \mathbb{N}$.
Then for all $i < k$, $d_E(\bar{x}_i, \bar{x}_{i+1}) \le \varepsilon_i$.
\end{lemma}

\begin{proof}
  \uses{lem:dist_nearestPt_of_isCover}
Apply Lemma~\ref{lem:dist_nearestPt_of_isCover} with $S = C_i$ and $x = \bar{x}_{i+1}$.
\end{proof}


\begin{lemma}\label{lem:dist_chainingSequence_le_sum}
  \uses{def:chainingSequence}
Let $(\varepsilon_n)_{n \in \mathbb{N}}$ be a sequence of positive numbers, $C_n$ a finite $\varepsilon_n$-cover of $A \subseteq E$ with $C_n \subseteq A$ and $x \in C_k$ for some $k \in \mathbb{N}$.
Then for $m \le k$, $d_E(\bar{x}_m, x) \le \sum_{i=m}^{k-1} \varepsilon_i$.
\end{lemma}

\begin{proof}
  \uses{lem:dist_chainingSequence_add_one}
By the triangle inequality and Lemma~\ref{lem:dist_chainingSequence_add_one},
\begin{align*}
  d_E(\bar{x}_m, x)
  \le \sum_{i=m}^{k-1} d_E(\bar{x}_i, \bar{x}_{i+1})
  \le \sum_{i=m}^{k-1} \varepsilon_i
  \: .
\end{align*}
\end{proof}


\begin{lemma}\label{lem:dist_chainingSequence_le}
  \uses{def:chainingSequence}
Let $(\varepsilon_n)_{n \in \mathbb{N}}$ be a sequence of positive numbers, $C_n$ a finite $\varepsilon_n$-cover of $A \subseteq E$ with $C_n \subseteq A$.
Let $m, k, \ell \in \mathbb{N}$ with $m \le k$ and $m \le \ell$ and let $x \in C_k$ and $y \in C_\ell$.
Then
\begin{align*}
  d_E(\bar{x}_m, \bar{y}_m)
  &\le d_E(x, y) + \sum_{i=m}^{k-1} \varepsilon_i + \sum_{j=m}^{\ell-j} \varepsilon_j
\end{align*}
\end{lemma}

\begin{proof}
  \uses{lem:dist_chainingSequence_le_sum}
Triangle inequality and Lemma~\ref{lem:dist_chainingSequence_le_sum}.
\end{proof}


\begin{corollary}\label{cor:dist_chainingSequence_pow_two_le}
  \uses{def:chainingSequence}
For $\varepsilon_n = \varepsilon_0 2^{-n}$, with the hypothesis of Lemma~\ref{lem:dist_chainingSequence_le}, we have
\begin{align*}
  d_E(\bar{x}_m, \bar{y}_m)
  &\le d_E(x, y) + \varepsilon_0 2^{-m+2}
  \: .
\end{align*}
\end{corollary}

\begin{proof}

\end{proof}


\subsection{A subset of pairs}

We will be interested in bounding expressions of the form $\sup_{s,t\in J, d_T(s,t) \le c} d_E(f(s), f(t))$ for a finite set $J$ and some function $f : T \to E$.
This is a supremum over pairs in $J$ and there could be $\vert J \vert^2$ such pairs.
We will build a subset $K$ of $J^2$ which is much smaller, of size linear in $\vert J \vert$, such that its points are not too far apart and
\begin{align*}
  \sup_{s,t\in J, d_T(s,t) \le c} d_E(f(s), f(t))
  & \le 2 \sup_{(s,t) \in K} d_E(f(s), f(t))
  \: .
\end{align*}

For $t \in V \subseteq T$ and $u\ge 0$, we denote by $B_V(t, u)$ the closed ball with center $t$ and radius $u$ in $V$.
That is, $B_V(t, u) = V \cap B(t, u) = \{s \in V \mid d_T(s, t) \le u\}$.

\begin{definition}\label{def:rtCBKV}
Let $(T,d_T)$ be a metric space and let $J \subseteq T$ be finite, $a,b,c \in \mathbb R_+$ with $a \ge 1$ and $n \in \{1, 2, ...\}$ such that $|J| \le b a^n$.
An rtCBKV structure for $(J, a, b, c, n)$ is a tuple $(r, t, C, B, K, V)$ where $r \in \mathbb{N}$, $t \in T$, $C, B, V$ are subsets of $J$, $K$ is a subset of $J^2$ with the properties that
\begin{itemize}
  \item $t \in V$
  \item $r \in \{1, 2, ...\}$ is the smallest number such that $\vert B_V(t, rc)\vert \le b a^r$
  \item $C = B_V(t, rc)$
  \item $B = B_V(t, (r-1)c) \subseteq C$
  \item $K = \{t\} \times C$
\end{itemize}
Note that given $(V, t)$, the other elements of the rtCBKV structure are uniquely determined.
\end{definition}


\begin{lemma}\label{lem:card_B_ge}
  \uses{def:rtCBKV}
In an rtCBKV structure for $(J, a, b, c, n)$, $b a^{r-1} \le \vert B \vert$~.
\end{lemma}

\begin{proof}

\end{proof}


\begin{lemma}\label{lem:card_K_le}
  \uses{def:rtCBKV}
In an rtCBKV structure for $(J, a, b, c, n)$, $\vert K \vert \le b a^r$~.
\end{lemma}

\begin{proof}

\end{proof}


\begin{definition}\label{def:rtCBKVSequence}
  \uses{def:rtCBKV}
Let $(T,d_T)$ be a metric space and let $J \subseteq T$ be finite, $a,b,c \in \mathbb R_+$ with $a \ge 1$ and $n \in \{1, 2, ...\}$ such that $|J| \le b a^n$.
An rtCBKV sequence for $(J, a, b, c, n)$ is a sequence of rtCBKV structures $(r_i, t_i, C_i, B_i, K_i, V_i)_{i \in \mathbb{N}}$ such that
\begin{itemize}
  \item $V_0 = J$, $t_0$ is an arbitrary point in $J$ (and the remainder of the rtCBKV structure is determined by those),
  \item $V_{i+1} = V_i \setminus B_i$, $t_{i+1}$ is arbitrarily chosen in $V_{i+1}$.
\end{itemize}
\end{definition}

An rtCBKV sequence gives a partition of $J$ into sets $B_i$, which are contained in balls of radius $(r_i - 1)c$ around $t_i$, and satisfy cardinality constraints.

\begin{lemma}\label{lem:rtCBKVSequence_eq_zero}
  \uses{def:rtCBKVSequence}
There exists $m \in \mathbb{N}$ such that the rtCBKV sequence $(r_i, t_i, C_i, B_i, K_i, V_i)_{i \in \mathbb{N}}$ for $(J, a, b, c, n)$ satisfies $V_k = \emptyset$ for all $k \ge m$.
\end{lemma}

\begin{proof}

\end{proof}


\begin{lemma}\label{lem:rtCBKVSequence_disjoint_B}
  \uses{def:rtCBKVSequence}
For $i \ne j$, the sets $B_i$ and $B_j$ of an rtCBKV sequence $(r_i, t_i, C_i, B_i, K_i, V_i)_{i \in \mathbb{N}}$ are disjoint.
\end{lemma}

\begin{proof}

\end{proof}


\begin{lemma}\label{lem:chain}
Let $(T,d_T)$ be a metric space.
Let $J \subseteq T$ be finite, $a,b,c \in \mathbb R_+$ with $a \ge 1$ and $n \in \{1, 2, ...\}$ such that $|J| \le b a^n$.
Then, there is $K \subseteq J^2$ such that for any function $f : T \to E$ with $(E,d_E)$ a metric space,
\begin{align}
  |K|
  & \le a |J|
  \:, \label{eq:chain1} \\
  (s,t) \in K
  & \implies d_T(s,t) \le c n
  \:, \label{eq:chain2} \\
  \sup_{s,t\in J, d_T(s,t) \le c} d_E(f(s), f(t))
  & \le 2 \sup_{(s,t) \in K} d_E(f(s), f(t))
  \: . \label{eq:chain3}
\end{align}
\end{lemma}

\begin{proof}
  \uses{lem:rtCBKVSequence_eq_zero, lem:card_B_ge, lem:card_K_le}
Let $(r_i, t_i, C_i, B_i, K_i, V_i)_{i \in \mathbb{N}}$ be an rtCBKV sequence for $(J, a, b, c, n)$.
Let $K = \bigcup_{i=0}^{m-1} K_i$ for $m$ as in Lemma~\ref{lem:rtCBKVSequence_eq_zero}.

Using Lemma~\ref{lem:card_K_le}, the cardinal of $K$ is bounded by
\begin{align*}
  \vert K \vert
  &\le \sum_{i=0}^{m-1} \vert K_i \vert
  \le \sum_{i=0}^{m-1} b a^{r_i}
  \: .
\end{align*}
Since the sets $B_i$ are disjoint by Lemma~\ref{lem:rtCBKVSequence_disjoint_B}, we can use Lemma~\ref{lem:card_B_ge} to get
\begin{align*}
  \sum_{i=0}^{m-1} b a^{r_i - 1}
  \le \sum_{i=0}^{m-1} \vert B_i \vert
  = \left\vert \bigcup_{i=0}^{m-1} B_i \right\vert
  \le \vert J \vert
  \: .
\end{align*}
We obtained the inequality $\vert K \vert \le a \vert J \vert$~\eqref{eq:chain1}.

A pair $(t, s) \in K$ is of the form $(t_i, s)$ for $s \in C_i$ and satisfies
\begin{align*}
  d_T(t_i, s) \le c r_i \le c n \: .
\end{align*}
This proves the second property~\eqref{eq:chain2}.

It remains to prove~\eqref{eq:chain3}.
Let $(s, t) \in J^2$ such that $d_T(s, t) \le c$.
Then there exists a largest $\ell \in \mathbb{N}$ such that $s, t \in V_\ell$.
Assume w.l.o.g. that $s \notin V_{\ell + 1}$. Then $s \in B_\ell$ (since $V_{\ell + 1} = V_\ell \setminus B_\ell$), which implies $d_T(s, t_\ell) \le (r_\ell - 1)c$.

Since $d_T(s, t) \le c$, $d_T(t, t_\ell) \le d_T(t, s) + d_T(s, t_\ell) \le r_\ell c$, hence $t \in C_\ell$ and we have that both $s$ and $t$ are in $C_\ell$.
Thus both $(t_\ell, s)$ and $(t_\ell, t)$ are in $K_\ell \subseteq K$.
Finally
\begin{align*}
  d_E(f(s), f(t))
  &\le d_E(f(s), f(t_\ell)) + d_E(f(t_\ell), f(t))
  \\
  &\le 2\sup_{(s',t') \in K} d_E(f(s'), f(t'))
  \: .
\end{align*}
\end{proof}


\subsection{Todo processes}

\begin{definition}[Todo process]\label{def:IsTodoProcess}
Let $X : T \to \Omega \to E$ be a stochastic process, where $(T, d_T)$ and $(E, d_E)$ are pseudo-metric spaces and $(\Omega, \mu)$ is a measure space.
Let $p, q > 0$.
We say that $X$ is a $(p,q)$-\emph{todo process} with constant $M$ if for all $s, t \in T$,
\begin{align*}
  \mu[d_E(X_s, X_t)^p] \le M d_T(s, t)^q
  \: .
\end{align*}
\end{definition}


\begin{lemma}\label{lem:integral_sup_rpow_dist_succ'}
  \uses{def:IsTodoProcess}
Let $X : T \to \Omega \to E$ be a $(p, q)$-todo process with constant $M$.
Let $\varepsilon > 0$ and $C \subseteq T^2$ be a finite set such that for all $(s, t) \in C$, $d_T(s, t) \le \varepsilon$.
Then
\begin{align*}
  \mu\left[\sup_{(s,t) \in C} d_E(X_s, X_t)^p \right]
  &\le \vert C \vert M \varepsilon^q
  \: .
\end{align*}
\end{lemma}

\begin{proof}
  \uses{def:IsTodoProcess}
\begin{align*}
  \mu\left[\sup_{(s,t) \in C} d_E(X_s, X_t)^p \right]
  &\le \mu\left[\sum_{(s,t) \in C} d_E(X_s, X_t)^p \right]
  \\
  &\le M \sum_{(s,t) \in C} d_T(s, t)^q
  \\
  &\le \vert C \vert M \varepsilon^q
  \: .
\end{align*}
\end{proof}


\begin{lemma}\label{lem:integral_sup_rpow_dist_of_dist_le}
Let $X : T \to \Omega \to E$ be a $(p, q)$-todo process with constant $M$.
Let $J \subseteq T$ be finite, $a,b,c \in \mathbb R_+$ with $a \ge 1$ and $n \in \{1, 2, ...\}$ such that $|J| \le b a^n$.
Then
\begin{align*}
  \mu \left[ \sup_{s, t \in J; d_T(s, t) \le c} d_E(X_s, X_t)^p \right]
  &\le 2^p a |J| M (cn)^q
  \: .
\end{align*}

\end{lemma}

\begin{proof}
  \uses{lem:chain, lem:integral_sup_rpow_dist_succ'}
By Lemma~\ref{lem:chain}, there exists $K \subseteq J^2$ such that
\begin{align*}
  |K|
  & \le a |J|
  \:, \\
  \forall (s,t) \in K,
  & \ d_T(s,t) \le c n
  \:, \\
  \sup_{s,t\in J, d_T(s,t) \le c} d_E(X_s, X_t)
  & \le 2 \sup_{(s,t) \in K} d_E(X_s, X_t)
  \: .
\end{align*}
Hence for such a set $K$,
\begin{align*}
  \mu \left[ \sup_{s, t \in J; d_T(s, t) \le c} d_E(X_s, X_t)^p \right]
  &\le 2^p \mu \left[ \sup_{(s, t) \in K} d_E(X_s, X_t)^p \right]
  \: .
\end{align*}
Then by Lemma~\ref{lem:integral_sup_rpow_dist_succ'},
\begin{align*}
  \mu \left[ \sup_{(s, t) \in K} d_E(X_s, X_t)^p \right]
  &\le |K| M (cn)^q
  \le a |J| M (cn)^q
  \: .
\end{align*}
\end{proof}


\subsection{Split}

TODO: we change scale.
From a set of points in $C_k$ which are not far apart, we build a set of points in $C_m$ for $m \le k$ which are not far apart.
Then for the set in $C_m$, we can use the reduction in the number of pairs.

\begin{lemma}\label{lem:scale_change}
  \uses{def:chainingSequence}
Let $X : T \to E$.
Let $(\varepsilon_n)_{n \in \mathbb{N}}$ be a sequence of positive numbers, $C_n$ a finite $\varepsilon_n$-cover of $J \subseteq T$ with $C_n \subseteq J$.
For $m \le k$,
\begin{align*}
  \sup_{s, t \in C_k; d_T(s, t) \le \delta} d_E(X_s, X_t)
  &\le \sup_{s, t \in C_k; d_T(s, t) \le \delta} d_E(X_{\bar{s}_m}, X_{\bar{t}_m})
    + 2 \sup_{s \in C_k} d_E(X_s, X_{\bar{s}_m})
  \: .
\end{align*}
\end{lemma}

\begin{proof}
  \uses{lem:dist_chainingSequence_le}
By the triangle inequality,
\begin{align*}
  d_E(X_s, X_t)
  &\le d_E(s, X_{\bar{s}_m}) + d(X_{\bar{s}_m}, X_{\bar{t}_m}) + d_E(X_{\bar{t}_m}, t)
  \: .
\end{align*}
\end{proof}


\begin{corollary}\label{cor:scale_change_rpow}
  \uses{def:chainingSequence}
Let $X : T \to E$.
Let $(\varepsilon_n)_{n \in \mathbb{N}}$ be a sequence of positive numbers, $C_n$ a finite $\varepsilon_n$-cover of $J \subseteq T$ with $C_n \subseteq J$.
For $m \le k$,
\begin{align*}
  \sup_{s, t \in C_k; d_T(s, t) \le \delta} d_E(X_s, X_t)^p
  &\le 2^p \sup_{s, t \in C_k; d_T(s, t) \le \delta} d_E(X_{\bar{s}_m}, X_{\bar{t}_m})^p
    + 4^p \sup_{s \in C_k} d_E(X_s, X_{\bar{s}_m})^p
  \: .
\end{align*}
\end{corollary}

\begin{proof}
  \uses{lem:scale_change}
This is Lemma~\ref{lem:scale_change}, together with the fact that for $a, b \ge 0$,
\begin{align*}
  (a + b)^p \le (2\max(a,b))^p = 2^p \max(a^p,b^p) \le 2^p (a^p + b^p)
  \: .
\end{align*}
\end{proof}


\subsection{First term}


\begin{lemma}\label{lem:integral_sup_rpow_dist_cover_of_dist_le}
  \uses{def:IsTodoProcess, def:chainingSequence}
Let $X : T \to \Omega \to E$ be a $(p, q)$-todo process with constant $M$.
Let $C$ be a finite $\varepsilon$-cover of $J \subseteq T$ with $C \subseteq J$, with minimal cardinal.
Then for $c \ge 0$,
\begin{align*}
  \mu \left[ \sup_{s, t \in C; d_T(s, t) \le c} d_E(X_s, X_t)^p \right]
  &\le 2^{p+2} M \left(\frac{2}{\log 2} c \log N^{int}_{\varepsilon}(J) \right)^q  N^{int}_{\varepsilon}(J)
  \: .
\end{align*}
\end{lemma}

\begin{proof}
  \uses{lem:integral_sup_rpow_dist_of_dist_le}
Let $\bar{r} = \min\{r \in \mathbb{N} \mid N^{int}_{\varepsilon}(J) \le 2^r\}$.
\begin{align*}
  \vert C \vert
  = N^{int}_{\varepsilon}(J)
  \le 2^{\bar{r}}
  \: .
\end{align*}
By Lemma~\ref{lem:integral_sup_rpow_dist_of_dist_le} with $J = C$, $a = 2$, $b = 1$, $c = c$, $n = \bar{r}$,
\begin{align*}
  \mu \left[ \sup_{s, t \in C; d_T(s, t) \le c} d_E(X_s, X_t)^p \right]
  &\le 2^{p+1} |C_m| M (c \bar{r})^q
  \le 2^{p+1} M (c \bar{r})^q  2^{\bar{r}}
  \: .
\end{align*}

By definition, $2^{\bar{r}} \le 2 N^{int}_{\varepsilon}(J)$ and $\bar{r} \le 1 + \frac{\log N^{int}_{\varepsilon}(J)}{\log 2}$.
Suppose $N^{int}_{\varepsilon}(J) \ge 2$ (TODO: other case).
Then $\bar{r} \le \frac{2 \log N^{int}_{\varepsilon}(J)}{\log 2}$.
\begin{align*}
  \mu \left[ \sup_{s, t \in C; d_T(s, t) \le c} d_E(X_s, X_t)^p \right]
  &\le 2^{p+2} M \left(\frac{2}{\log 2} c \log N^{int}_{\varepsilon}(J) \right)^q  N^{int}_{\varepsilon}(J)
  \: .
\end{align*}
\end{proof}


\begin{lemma}\label{lem:integral_sup_rpow_dist_cover_rescale}
  \uses{def:IsTodoProcess, def:chainingSequence}
Let $X : T \to \Omega \to E$ be a $(p, q)$-todo process with constant $M$.
For all $n \in \mathbb{N}$, let $C_n$ a finite $\varepsilon_n$-cover of $J \subseteq T$ with $C_n \subseteq J$ for $\varepsilon_n = \varepsilon_0 2^{-n}$, with minimal cardinal.
Let $\delta > 0$ and let $n_2$ be such that $\varepsilon_0 2^{-n_2} < \delta/2 \le \varepsilon_0 2^{-n_2+1}$.
Let $k = \min \{j \in \mathbb{Z} \mid \varepsilon_0 2^{-j} < \inf_{s, t \in J; s \ne t}d_T(s, t)\}$.
For $m = \min(n_2, k)$,
\begin{align*}
  \mu \left[ \sup_{s, t \in C_k; d_T(s, t) \le \delta} d_E(X_{\bar{s}_m}, X_{\bar{t}_m})^p \right]
  &\le 2^{p+2} M \left(\frac{16}{\log 2} \delta \log N^{int}_{\delta/4}(J) \right)^q  N^{int}_{\delta/4}(J)
  \: .
\end{align*}
TODO: $C_k = J$
\end{lemma}

\begin{proof}
  \uses{lem:integral_sup_rpow_dist_cover_of_dist_le, cor:dist_chainingSequence_pow_two_le}
If $\delta \le \varepsilon_0 2^{-k}$, then $\{(s, t) \in C_k; d_T(s, t) \le \delta\} = \{(s, s) \mid s \in C_k\}$ and the inequality holds trivially.
We can thus assume $\delta > \varepsilon_0 2^{-k}$.

For $s, t \in C_k$ with $d_T(s, t) \le \delta$, $d_T(\bar{s}_m, \bar{t}_m) \le \delta + \varepsilon_0 2^{-m+2} \le \varepsilon_0 2^{-m+3}$ (Corollary~\ref{cor:dist_chainingSequence_pow_two_le}).
It thus suffices to get a bound on $\mu \left[ \sup_{s, t \in C_m; d_T(s, t) \le \varepsilon_0 2^{-m+3}} d_E(X_s, X_t)^p \right]$.

We can apply Lemma~\ref{lem:integral_sup_rpow_dist_cover_of_dist_le} with $\varepsilon = \varepsilon_m$, $c = \varepsilon_0 2^{-m+3}$. We obtain
\begin{align*}
  \mu \left[ \sup_{s, t \in C_m; d_T(s, t) \le \varepsilon_0 2^{-m+3}} d_E(X_s, X_t)^p \right]
  &\le 2^{p+2} M \left(\frac{16}{\log 2} \varepsilon_0 2^{-m} \log N^{int}_{\varepsilon_m}(J) \right)^q  N^{int}_{\varepsilon_m}(J)
  \: .
\end{align*}
By definition of $m$ and $n_2$, $\varepsilon_m = \varepsilon_0 2^{-m} \ge \varepsilon_0 2^{-n_2} \ge \delta/4$,
hence $N^{int}_{\varepsilon_m}(J) \le N^{int}_{\delta / 4}(J)$.

If $m = n_2$ then $\varepsilon_0 2^{-m} = \varepsilon_0 2^{-n_2} < \delta/2$.
Otherwise, $m = k$ and $\varepsilon_0 2^{-m} = \varepsilon_0 2^{-k} < \delta$ as argued at the start of the proof.
We thus get $\varepsilon_0 2^{-m} \le \delta$.
\end{proof}


\subsection{Second term}


\begin{lemma}\label{lem:integral_sup_rpow_dist_succ}
  \uses{def:IsTodoProcess}
Let $X : T \to \Omega \to E$ be a $(p, q)$-todo process with constant $M$.
Let $(\varepsilon_n)_{n \in \mathbb{N}}$ be a sequence of positive numbers and $C_n$ a finite $\varepsilon_n$-cover of $T$ with $C_n \subseteq T$.
Then for $j < k$,
\begin{align*}
  \mu\left[\sup_{t \in C_k} d_E(X_{\bar{t}_j}, X_{\bar{t}_{j+1}})^p \right]
  &\le \vert C_{j+1} \vert M \varepsilon_j^q
  \: .
\end{align*}
\end{lemma}

\begin{proof}
  \uses{lem:dist_chainingSequence_add_one, lem:integral_sup_rpow_dist_succ'}
\begin{align*}
  \mu\left[\sup_{t \in C_k} d_E(X_{\bar{t}_j}, X_{\bar{t}_{j+1}})^p \right]
  &\le \mu\left[\sum_{u \in C_{j+1}} d_E(X_{\bar{u}_j}, X_{u})^p \right]
  \: .
\end{align*}
We then apply Lemma~\ref{lem:integral_sup_rpow_dist_succ'} to the set $C = \{(\bar{u}_j, u) \mid u \in C_{j+1}\}$, which satisfies the condition $d_T(\bar{u}_j, u) \le \varepsilon_j$ and has cardinal $\vert C_{j+1} \vert$.
\end{proof}


\begin{lemma}\label{lem:integral_sup_dist_le_sum_rpow}
  \uses{def:IsCover}
Let $X : T \to \Omega \to E$ be a stochastic process.
Let $(\varepsilon_n)_{n \in \mathbb{N}}$ be a sequence of positive numbers and $C_n$ a finite $\varepsilon_n$-cover of $T$ with $C_n \subseteq T$.
For $m \le k$,
\begin{align*}
  \mu \left[\sup_{t \in C_k} d_E(X_t, X_{\bar{t}_m})^p \right]
  &\le \left(\sum_{i=m}^{k-1} \left( \mu\left[\sup_{t \in C_k} d_E(X_{\bar{t}_i}, X_{\bar{t}_{i+1}})^p\right] \right)^{1/p}\right)^p
  \: .
\end{align*}
\end{lemma}

\begin{proof}
\begin{align*}
  \sup_{t \in C_k} d_E(X_t, X_{\bar{t}_m})^p
  &\le \sup_{t \in C_k} \left( \sum_{i=m}^{k-1} d_E(X_{\bar{t}_i}, X_{\bar{t}_{i+1}}) \right)^p
  \\
  &\le \left( \sum_{i=m}^{k-1} \sup_{t \in C_k} d_E(X_{\bar{t}_i}, X_{\bar{t}_{i+1}}) \right)^p
  \: .
\end{align*}
And then, by Minkowski's inequality,
\begin{align*}
  \left(\mu \left[\sup_{t \in C_k} d_E(X_t, X_{\bar{t}_m})^p \right]\right)^{1/p}
  &\le \sum_{i=m}^{k-1} \mu \left[\sup_{t \in C_k} d_E(X_{\bar{t}_i}, X_{\bar{t}_{i+1}}) \right]
  \\
  &\le \sum_{i=m}^{k-1} \left( \mu\left[\sup_{t \in C_k} d_E(X_{\bar{t}_i}, X_{\bar{t}_{i+1}})^p \right] \right)^{1/p}
  \: .
\end{align*}
\end{proof}


\begin{lemma}\label{lem:integral_sup_rpow_dist_le_sum}
  \uses{def:IsTodoProcess}
Let $X : T \to \Omega \to E$ be a $(p, q)$-todo process with constant $M$.
Let $(\varepsilon_n)_{n \in \mathbb{N}}$ be a sequence of positive numbers and $C_n$ a finite $\varepsilon_n$-cover of $T$ with $C_n \subseteq T$.
Then for $m \le k$,
\begin{align*}
  \mu \left[\sup_{t \in C_k} d_E(X_t, X_{\bar{t}_m})^p \right]
  &\le M \left( \sum_{j=m}^{k-1} \vert C_{j+1} \vert^{1/p} \varepsilon_j^{q/p} \right)^p
  \: .
\end{align*}
\end{lemma}

\begin{proof}
  \uses{lem:integral_sup_rpow_dist_succ, lem:integral_sup_dist_le_sum_rpow}
Put together Lemma~\ref{lem:integral_sup_rpow_dist_succ} and Lemma~\ref{lem:integral_sup_dist_le_sum_rpow}.
\end{proof}


\begin{lemma}\label{lem:integral_sup_rpow_dist_le_of_minimal_cover}
  \uses{def:IsTodoProcess, def:HasBoundedInternalCoveringNumber}
Let $X : T \to \Omega \to E$ be a $(p, q)$-todo process with constant $M$.
Let $(\varepsilon_n)_{n \in \mathbb{N}}$ be a sequence of positive numbers in $(0, \mathrm{diam}(T))$ and $C_n$ a finite $\varepsilon_n$-cover of $T$ with $C_n \subseteq T$, and with minimal cardinality.
Suppose that $T$ has bounded internal covering number with constant $c_1>0$ and exponent $d > 0$.
Then for $m \le k$,
\begin{align*}
  \mu \left[\sup_{t \in C_k} d_E(X_t, X_{\bar{t}_m})^p \right]
  &\le M c_1 \left( \sum_{j=m}^{k-1} \varepsilon_{j+1}^{-d/p} \varepsilon_j^{q/p} \right)^p
  \: .
\end{align*}
\end{lemma}

\begin{proof}
  \uses{lem:integral_sup_rpow_dist_le_sum, def:HasBoundedInternalCoveringNumber}
By Lemma~\ref{lem:integral_sup_rpow_dist_le_sum}, we have
\begin{align*}
  \mu \left[\sup_{t \in C_k} d_E(X_t, X_{\bar{t}_m})^p \right]
  &\le M \left( \sum_{j=m}^{k-1} \vert C_{j+1} \vert^{1/p} \varepsilon_j^{q/p} \right)^p
  \: .
\end{align*}
Then by the minimality of the cardinality of $C_n$ and the bounded internal covering number hypothesis, we have
\begin{align*}
  \vert C_{j+1} \vert
  &\le N^{int}_{\varepsilon_{j+1}}(T)
  \le c_1 \varepsilon_{j+1}^{-d}
  \: .
\end{align*}
\end{proof}


\begin{corollary}\label{cor:integral_sup_rpow_dist_le_of_minimal_cover_two}
  \uses{def:IsTodoProcess, def:HasBoundedInternalCoveringNumber}
Under the assumptions of Lemma~\ref{lem:integral_sup_rpow_dist_le_of_minimal_cover}, for $\varepsilon_n = \varepsilon_0 2^{-n}$, then for $m \le k$,
\begin{align*}
  \mu \left[\sup_{t \in C_k} d_E(X_t, X_{\bar{t}_m})^p \right]
  &\le M c_1 \varepsilon_0^{q - d} 2^{d - m(q-d)}\left( \frac{1 - 2^{- (k - m) (q -d)/p}}{1 - 2^{- (q -d)/p}}\right)^p
  \: .
\end{align*}
\end{corollary}

\begin{proof}
  \uses{lem:integral_sup_rpow_dist_le_of_minimal_cover}

\end{proof}


\subsection{Bound on finite sets}

\begin{theorem}\label{thm:finite_set_bound}
  \uses{def:IsTodoProcess, def:chainingSequence}
Let $X : T \to \Omega \to E$ be a $(p, q)$-todo process with constant $M$.
For all $n \in \mathbb{N}$, let $C_n$ a finite $\varepsilon_n$-cover of $J \subseteq T$ with $C_n \subseteq J$ for $\varepsilon_n = \varepsilon_0 2^{-n}$, with minimal cardinal.
Let $\delta > 0$ and let $n_2$ be such that $\varepsilon_0 2^{-n_2} < \delta/2 \le \varepsilon_0 2^{-n_2+1}$.
Let $k = \min \{j \in \mathbb{Z} \mid \varepsilon_0 2^{-j} < \inf_{s, t \in J; s \ne t}d_T(s, t)\}$.
For $m = \min(n_2, k)$,
\begin{align*}
  &\mu\left[ \sup_{s, t \in C_k; d_T(s, t) \le \delta} d_E(X_s, X_t)^p \right]
  \\
  &\le 4^p M \left(4^{2q+1}\delta^q \left(\log_2 N^{int}_{\delta/4}(J) \right)^q  N^{int}_{\delta/4}(J)
    + c_1 2^d \delta^{q - d} \left( \frac{1 - 2^{- (k - m) (q -d)/p}}{1 - 2^{- (q -d)/p}}\right)^p\right)
  \: .
\end{align*}

TODO: $C_k = J$.
Note the $\log_2$ (and not $\log$) in the expression above.
\end{theorem}

\begin{proof}
  \uses{cor:integral_sup_rpow_dist_le_of_minimal_cover_two, lem:integral_sup_rpow_dist_cover_rescale, lem:scale_change}
By Corollary~\ref{cor:scale_change_rpow},
\begin{align*}
  &\mu\left[ \sup_{s, t \in C_k; d_T(s, t) \le \delta} d_E(X_s, X_t)^p \right]
  \\
  &\le 2^p \mu\left[ \sup_{s, t \in C_k; d_T(s, t) \le \delta} d_E(X_{\bar{s}_m}, X_{\bar{t}_m})^p \right]
    + 4^p \mu\left[ \sup_{s \in C_k} d_E(X_s, X_{\bar{s}_m})^p \right]
  \: .
\end{align*}
By Lemma~\ref{lem:integral_sup_rpow_dist_cover_rescale} and Corollary~\ref{cor:integral_sup_rpow_dist_le_of_minimal_cover_two},
\begin{align*}
  \mu \left[ \sup_{s, t \in C_k; d_T(s, t) \le \delta} d_E(X_{\bar{s}_m}, X_{\bar{t}_m})^p \right]
  &\le 2^{p+2} M \left(\frac{16}{\log 2} \delta \log N^{int}_{\delta/4}(J) \right)^q  N^{int}_{\delta/4}(J)
  \: . \\
  \mu \left[\sup_{t \in C_k} d_E(X_t, X_{\bar{t}_m})^p \right]
  &\le M c_1 \varepsilon_0^{q - d} 2^{d - m(q-d)}\left( \frac{1 - 2^{- (k - m) (q -d)/p}}{1 - 2^{- (q -d)/p}}\right)^p
  \: .
\end{align*}
Putting these two inequalities together, we obtain
\begin{align*}
  &\mu\left[ \sup_{s, t \in C_k; d_T(s, t) \le \delta} d_E(X_s, X_t)^p \right]
  \\
  &\le 4^p M \left(4\left(\frac{16}{\log 2} \delta \log N^{int}_{\delta/4}(J) \right)^q  N^{int}_{\delta/4}(J)
    + c_1 \varepsilon_0^{q - d} 2^{d - m(q-d)}\left( \frac{1 - 2^{- (k - m) (q -d)/p}}{1 - 2^{- (q -d)/p}}\right)^p\right)
  \: .
\end{align*}
TODO: it remains to argue that w.l.o.g. $\varepsilon_0 2^{-m} \le \delta$ as in another lemma above (todo: refactor that).
\end{proof}


\section{Kolmogorov-Chentsov Theorem}



\begin{theorem}[Continuous version; Kolmogorov, Chentsov] \label{thm:kolchen_general}
  \uses{def:HasBoundedInternalCoveringNumber, def:IsTodoProcess}
  \notready
Let $(I, d_I)$ be a compact metric space.
Suppose that $I$ has bounded internal covering number with constant $c_1>0$ and exponent $d \in \mathbb{N}$, i.e. for all $\varepsilon > 0$ smaller than $\mathrm{diam}(I)$, $N^{int}_\varepsilon(I) \le c_1 \varepsilon^{-d}$.
Assume that $X = (X_t)_{t\in I}$ is an $E$-valued stochastic process and there are $\alpha, \beta, c_2>0$ such that $X$ is a $(\alpha, d+\beta)$-todo process with constant $c_2$.
% \begin{align*}
%   \mathbb{E}[d_E(X_s, X_t)^\alpha]
%   \le c_2 d_I(s,t)^{d+\beta}, \qquad s,t\in I
%   \: .
% \end{align*}
Then, there exists a version $Y = (Y_t)_{t\in I}$ of $X$ such that, for some random variables $H>0$ and $K<\infty$,
\begin{align*}
  \mathbb{P}\Big(\sup_{s\neq t, d_I(s,t) \leq H} d_E(Y_s, Y_t)/d_I(s,t)^\gamma
  \le K\Big) = 1
  \: ,
\end{align*}
for every $\gamma\in(0,\beta/\alpha)$.
In particular, $Y$ almost surely is locally Hölder of all orders $\gamma \in (0,\beta/\alpha)$, and has continuous paths.
\end{theorem}

\begin{proof}\uses{thm:finite_set_bound}

\end{proof}

\section{Brownian motion}
