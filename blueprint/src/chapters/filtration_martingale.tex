\chapter{{Filtrations, processes and martingales}}\label{chap:filtration_martingale}


First, recall the definitions of a filtration, an adapted process, a (sub)martingale, a stopping time and a stopped process, which are already in Mathlib.


\begin{definition}[Filtration]\label{def:filtration}
  \mathlibok
  \lean{MeasureTheory.Filtration}
A filtration on a measurable space $(\Omega, \mathcal{A})$ with measure $P$ indexed by a preordered set $T$ is a family of sigma-algebras $\mathcal{F} = (\mathcal{F}_t)_{t \in T}$ such that for all $i \le j$, $\mathcal{F}_i \subseteq \mathcal{F}_j$ and for all $t \in T$, $\mathcal{F}_t \subseteq \mathcal{A}$.
\end{definition}


\begin{definition}\label{def:adapted}
  \uses{def:filtration}
  \mathlibok
  \lean{MeasureTheory.Adapted}
A process $X : T \to \Omega \to E$ is said to be adapted with respect to a filtration $\mathcal{F}$ if for all $t \in T$, $X_t$ is $\mathcal{F}_t$-measurable.
\end{definition}


\begin{definition}[Martingale]\label{def:Martingale}
  \uses{def:adapted}
  \mathlibok
  \lean{MeasureTheory.Martingale}
Let $\mathcal{F}$ be a filtration on a measurable space $\Omega$ with measure $P$ indexed by $T$.
A family of functions $M : T \to \Omega \to E$ is a martingale with respect to a filtration $\mathcal{F}$ if $M$ is adapted with respect to $\mathcal{F}$ and for all $i \le j$, $P[M_j \mid \mathcal{F}_i] = M_i$ almost surely.
\end{definition}


\begin{definition}[Submartingale]\label{def:Submartingale}
  \uses{def:adapted}
  \mathlibok
  \lean{MeasureTheory.Submartingale}
Let $\mathcal{F}$ be a filtration on a measurable space $\Omega$ with measure $P$ indexed by $T$.
A family of functions $M : T \to \Omega \to E$ is a submartingale with respect to a filtration $\mathcal{F}$ if $M$ is adapted with respect to $\mathcal{F}$ and for all $i \le j$, $P[M_j \mid \mathcal{F}_i] \ge M_i$ almost surely.
\end{definition}


\begin{definition}[Stopping time]\label{def:IsStoppingTime}
  \uses{def:filtration}
  \mathlibok
  \lean{MeasureTheory.IsStoppingTime}
A stopping time with respect to some filtration $\mathcal{F}$ indexed by $T$ is a function $\tau : \Omega \to T$ such that for all $i$, the preimage of $\{j \mid j \le i\}$ along $\tau$ is measurable with respect to $\mathcal{F}_i$.
\end{definition}


\begin{definition}[Stopped process]\label{def:stoppedProcess}
  \mathlibok
  \lean{MeasureTheory.stoppedProcess}
Let $X : T \to \Omega \to E$ be a stochastic process and let $\tau : \Omega \to T$.
The stopped process with respect to $\tau$ is defined by
\begin{align*}
  (X^{\tau})_t = \begin{cases}
    X_t & \text{if } t \le \tau \\
    X_{\tau} & \text{otherwise}
  \end{cases}
\end{align*}
\end{definition}


We now give the definition of a filtered probability space satisfying the usual conditions.


\begin{definition}\label{def:leftRightLimitFiltration}
  \uses{def:filtration}
For $\mathcal{F}$ a filtration indexed by $T$ and $t \in T$, we define $\mathcal{F}_{t-} = \bigsqcup_{s < t} \mathcal{F}_s$ (if that supremum is nonempty: we set $\mathcal{F}_{\bot-} = \mathcal{F}_\bot$) and $\mathcal{F}_{t+} = \bigsqcap_{s > t} \mathcal{F}_s$.

Note that $\bigsqcap$ and $\bigsqcup$ denote the infimum and supremum in the lattice of sigma-algebras on $\Omega$.
\end{definition}


\begin{definition}[Right-continuous filtration]\label{def:rightContinuous}
  \uses{def:leftRightLimitFiltration}
We say that the filtration is right-continuous if for all $t \in T$, $\mathcal{F}_t = \mathcal{F}_{t+}$.
\end{definition}


\begin{definition}[Usual conditions]\label{def:usualConditions}
  \uses{def:rightContinuous}
We say that a filtered probability space $(\Omega, \mathcal{F}, P)$ satisfies the usual conditions if the filtration is right-continuous and if $\mathcal{F}_0$ contains all the $P$-null sets.
\end{definition}


\begin{definition}\label{def:predictableMeasurableSpace}
  \uses{def:filtration}
Let $\mathcal{F}$ be a filtration on a measurable space indexed $\Omega$ by a linearly ordered set $T$.
Let $S = \{\{\bot\} \times A \mid A \in \mathcal{F}_\bot\}$ if $T$ has a bottom element and $S = \emptyset$ otherwise.
The predictable sigma-algebra on $T \times \Omega$ is the sigma-algebra generated by the set of sets $\{(t, \infty] \times A \mid t \in T, \: A \in \mathcal{F}_t\} \cup S$.
\end{definition}


\begin{definition}[Predictable process]\label{def:predictable}
  \uses{def:predictableMeasurableSpace}
A process $X : T \to \Omega \to E$ is said to be predictable with respect to a filtration $\mathcal{F}$ if it is measurable with respect to the predictable sigma-algebra on $T \times \Omega$.
\end{definition}


\begin{lemma}\label{lem:Predictable.progressive}
  \uses{def:predictable}
A predictable process is progressively measurable.
\end{lemma}

\begin{proof}

\end{proof}


\begin{lemma}\label{lem:predictable_nat_iff}
  \uses{def:predictable}
Let $X : \mathbb{N} \to \Omega \to E$ be a stochastic process and let $\mathcal{F}$ be a filtration indexed by $\mathbb{N}$.
Then $X$ is predictable if and only if $X_0$ is $\mathcal{F}_0$-measurable and for all $n \in \mathbb{N}$, $X_{n+1}$ is $\mathcal{F}_n$-measurable.
\end{lemma}

\begin{proof}

\end{proof}
