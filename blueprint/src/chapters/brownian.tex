\chapter{Brownian motion}
\label{chap:brownian}


\section{Stochastic process with continuous paths}

\begin{definition}[pre-Brownian process]\label{def:preBrownian}
  \uses{def:gaussianLimit}
  \leanok
  \lean{ProbabilityTheory.preBrownian}
Let $\Omega = \mathbb{R}^{\mathbb{R}_+}$ and consider the probability space $(\Omega, P_B)$ (where $P_B$ is the measure defined in Definition~\ref{def:gaussianLimit}).
The identity on that space is a function $\Omega \to \mathbb{R}_+ \to \mathbb{R}$.
We reorder the arguments to define a stochastic process $X : \mathbb{R}_+ \to \Omega \to \mathbb{R}$, which we call the pre-Brownian process.
\end{definition}


\begin{lemma}\label{lem:isGaussianProcess_preBrownian}
  \uses{def:preBrownian, def:IsGaussianProcess}
  \leanok
  \lean{ProbabilityTheory.isGaussianProcess_preBrownian}
  The pre-Brownian process $X$ of Definition~\ref{def:preBrownian} is a Gaussian process.
\end{lemma}

\begin{proof}\leanok
  \uses{lem:isGaussian_multivariateGaussian}
For any $t_1, \ldots, t_n \in \mathbb{R}_+$, the distribution of $(X_{t_1}, \ldots, X_{t_n})$ is given by a finite-dimensional distribution of $P_B$, and therefore is Gaussian.
\end{proof}


\begin{lemma}\label{lem:map_sub_preBrownian}
  \uses{def:preBrownian}
  \leanok
  \lean{ProbabilityTheory.map_sub_preBrownian}
Let $X$ be the pre-Brownian process of Definition~\ref{def:preBrownian}.
Then, for all $s, t \in \mathbb{R}_+$, the random variable $X_t - X_s$ is a Gaussian random variable with mean $0$ and variance $|t - s|$.
\end{lemma}

\begin{proof}\leanok
  \uses{lem:isGaussianProcess_preBrownian, lem:isGaussian_map}
The map
\begin{align*}
  L : \mathbb{R}^2 &\to \mathbb{R} \\
  (x_1, x_2) &\mapsto x_2 - x_1
\end{align*}
is a continuous linear map, and $(X_s, X_t)$ is Gaussian by Lemma~\ref{lem:isGaussianProcess_preBrownian}, therefore $X_t - X_s$ is Gaussian by Lemma~\ref{lem:isGaussian_map}. By definition of $P_B$, we have
$$\mathbb{E}[X_t - X_s] = \mathbb{E}[X_t] - \mathbb{E}[X_s] = 0 - 0 = 0,$$
and
$$\mathrm{Var}(X_t - X_s) = \mathrm{Var}(X_t) - 2\mathrm{Cov}(X_t, X_s) + \mathrm{Var}(X_s) = t - 2(t \land s) + s = t \lor s - t \land s = |t - s|.$$
Therefore $X_t - X_s$ is Gaussian with mean $0$ and variance $|t - s|$.
\end{proof}


\begin{lemma}\label{lem:isKolmogorovProcess_preBrownian}
  \uses{def:preBrownian}
  \leanok
  \lean{ProbabilityTheory.isMeasurableKolmogorovProcess_preBrownian}
The pre-Brownian process $X$ of Definition~\ref{def:preBrownian} satisfies the Kolmogorov condition for exponents $(2n,n)$ with constant $(2n - 1)!!$ for all $n \in \mathbb{N}$.
That is, for all $s, t \in \mathbb{R}_+$, we have
\begin{align*}
  \mathbb{E} \left[ |X_t - X_s|^{2n} \right] \le (2n - 1)!! |t - s|^n
  \: .
\end{align*}
\end{lemma}

\begin{proof}\leanok
  \uses{lem:centralMoment_two_mul_gaussianReal, lem:map_sub_preBrownian}
$X_t - X_s$ is a Gaussian random variable with mean $0$ and variance $|t - s|$ (Lemma~\ref{lem:map_sub_preBrownian}).
Thus, by Lemma~\ref{lem:centralMoment_two_mul_gaussianReal}, we have
\begin{align*}
  \mathbb{E} \left[ |X_t - X_s|^{2n} \right]
  = (2n - 1)!! |t - s|^n
  \: .
\end{align*}
\end{proof}


\begin{definition}[Brownian motion]\label{def:brownian}
  \uses{thm:localized_holder_modification_sup, def:preBrownian, lem:isKolmogorovProcess_preBrownian, lem:hasBoundedCoveringNumberCover_nnreal}
  \leanok
  \lean{ProbabilityTheory.brownian}
By Theorem~\ref{thm:localized_holder_modification_sup}, there exists a modification $B$ of the pre-Brownian process such that all the paths of $B$ are Hölder continuous of all orders $\gamma \in (0, 1/2)$.
We call $B$ the \emph{Brownian motion} on $\mathbb{R}_+$.
\end{definition}


\begin{lemma}\label{lem:isGaussianProcess_brownian}
  \uses{def:brownian, def:IsGaussianProcess}
  \leanok
  \lean{ProbabilityTheory.isGaussianProcess_brownian}
The Brownian motion is a Gaussian process.
\end{lemma}

\begin{proof}\leanok
  \uses{lem:isGaussianProcess_of_modification, lem:isGaussianProcess_preBrownian}
The pre-Brownian process is a Gaussian process by Lemma~\ref{lem:isGaussianProcess_preBrownian}.
The Brownian motion is a modification of the pre-Brownian process by Definition~\ref{def:brownian}.
Thus, the Brownian motion is a Gaussian process as well by Lemma~\ref{lem:isGaussianProcess_of_modification}.
\end{proof}


\begin{lemma}\label{lem:isHolderWith_brownian}
  \uses{def:brownian}
  \leanok
  \lean{ProbabilityTheory.memHolder_brownian}
The paths of the Brownian motion are locally Hölder continuous of all orders $\gamma \in (0, 1/2)$.
\end{lemma}

\begin{proof}\leanok
  \uses{lem:hasBoundedCoveringNumberCover_nnreal, lem:isKolmogorovProcess_preBrownian, thm:localized_holder_modification_sup}
Consider the cover of $\mathbb{R}_+$ given by $T_n := [0, n + 1)$. By Lemma~\ref{lem:hasBoundedCoveringNumberCover_nnreal}, this cover has bounded covering numbers with exponent $1$. Moreover, by Lemma~\ref{lem:isKolmogorovProcess_preBrownian}, the pre-Brownian process satisfies the Kolmogorov condition with exponents $(2n + 4, n + 2)$ for all $n \in \mathbb{N}$.
In particular, for all $n \in \mathbb{N}$, $2n + 4 > 1$. Applying Theorem~\ref{thm:localized_holder_modification_sup} we deduce that the modification used in Definition~\ref{def:brownian} has locally Hölder continuous paths for all orders $\gamma \in (0, \sup_n (n + 1) / (2n + 4))$.
Clearly for all $n \in \mathbb{N}$ we have $(n + 1) / (2n + 4) \leqslant 2$, and this quantity tends to $2$ as $n$ goes to infinity. Therefore the Brownian motion has Hölder continuous paths of order $\gamma$ for all $\gamma \in (0, 1 / 2)$.
\end{proof}


\begin{lemma}\label{lem:continuous_brownian}
  \uses{def:brownian}
  \leanok
  \lean{ProbabilityTheory.continuous_brownian}
The paths of the Brownian motion are continuous.
\end{lemma}

\begin{proof}\leanok
  \uses{lem:isHolderWith_brownian}
The paths are $1/4$-Hölder continuous by Lemma~\ref{lem:isHolderWith_brownian} because $0 < 1/4 < 1/2$, therefore they are continuous.
\end{proof}


\begin{lemma}\label{lem:measurePreserving_brownian_apply}
  \uses{def:brownian}
  \leanok
  \lean{ProbabilityTheory.measurePreserving_brownian_apply}
For $t \in \mathbb{R}_+$, the law of $B_t$ (the Brownian motion at time $t$) is the real Gaussian measure $\mathcal{N}(0,t)$.
\end{lemma}

\begin{proof}\leanok
  \uses{def:gaussianLimit}
The Brownian motion $B$ is a modification of the pre-Brownian process, and therefore has same finite dimensional distributions. By Definition~\ref{def:gaussianLimit}, $B_t$ has law $\mathcal{N}(0,t)$.
\end{proof}


\begin{lemma}\label{lem:measurePreserving_brownian_sub}
  \uses{def:brownian}
  \leanok
  \lean{ProbabilityTheory.measurePreserving_brownian_sub}
For $s, t \in \mathbb{R}_+$, the law of $B_t - B_s$ is the real Gaussian measure $\mathcal{N}(0,\vert t - s \vert)$.
\end{lemma}

\begin{proof}\leanok
  \uses{lem:map_sub_preBrownian}
The Brownian motion $B$ is a modification of the pre-Brownian process, and therefore has same finite dimensional distributions. By Lemma~\ref{lem:map_sub_preBrownian}, $B_t - B_s$ has law $\mathcal{N}(0,\vert t - s \vert)$.
\end{proof}


\begin{definition}\label{def:HasIndepIncrements}
  \leanok
  \lean{ProbabilityTheory.HasIndepIncrements}
We say that a stochastic process $X : T \to \Omega \to E$ has independent increments if for all $t_1, \ldots, t_n \in T$ with $t_1 \le t_2 \le \cdots \le t_n$, the random variables $X_{t_2} - X_{t_1}, X_{t_3} - X_{t_2}, \ldots, X_{t_n} - X_{t_{n-1}}$ are independent.
\end{definition}


\begin{lemma}\label{lem:hasIndepIncrements_brownian}
  \uses{def:brownian, def:HasIndepIncrements}
  \leanok
  \lean{ProbabilityTheory.hasIndepIncrements_brownian}
The Brownian motion has independent increments.
\end{lemma}

\begin{proof}\leanok
Let $t_1 \le t_2 \le \ldots \le t_n$ be $n$ times in $\mathbb{R}_+$.
Then $(B_{t_2} - B_{t_1}, B_{t_3} - B_{t_2}, \ldots, B_{t_n} - B_{t_{n-1}})$ is a linear transformation of $(B_{t_1}, \ldots, B_{t_n})$, hence it is Gaussian.
We can compute its mean (which is $0$) and its covariance matrix.
The covariance matrix is diagonal, which means the the random variables are uncorrelated.
Because they are jointly Gaussian, this implies that they are independent.
\end{proof}


\section{Wiener measure on the continuous functions}

We want to turn the Brownian motion into a measure on the continuous functions $C(\mathbb{R}_+, \mathbb{R})$ with the Borel sigma-algebra generated by the compact-open topology.


\begin{definition}[Auxiliary Wiener measure]\label{def:wienerMeasureAux}
  \uses{def:brownian, def:gaussianLimit, lem:continuous_brownian}
  \leanok
  \lean{ProbabilityTheory.wienerMeasureAux}
The pushforward of the measure $P_B$ of Definition~\ref{def:gaussianLimit} by the Brownian motion $B$ is a measure on the continuous functions on $\mathbb{R}^{\mathbb{R}_+}$, with the sigma-algebra induced by the product sigma-algebra on $\mathbb{R}^{\mathbb{R}_+}$.
\end{definition}

\textbf{Lean remark}: the auxiliary Wiener measure is a measure on the subtype \texttt{\{f  // Continuous f\}}. This is not the same type as $C(\mathbb{R}_+, \mathbb{R})$.


\begin{theorem}\label{thm:ContinuousMap.borel_eq_iSup_comap_eval}
  \leanok
  \lean{ProbabilityTheory.ContinuousMap.borel_eq_iSup_comap_eval}
The Borel sigma-algebra on $C(\mathbb{R}_+, \mathbb{R})$ coming from the compact-open topology is equal to the smallest sigma-algebra for which the evaluation maps $f \mapsto f(t)$ are measurable for every $t \in \mathbb{R}_+$.
\end{theorem}

\begin{proof}\leanok
We prove that this holds for $C(X, Y)$ as long as $X$ and $Y$ are second-countable topological spaces endowed with their Borel sigma-algebra, $X$ is locally compact (i.e.\ each point of $X$ has a basis of compact neighbourhoods), and $Y$ is regular (i.e.\ for any $C \subseteq Y$ closed and $y \notin Y$, there exist disjoint open subsets $U, V \subseteq Y$ such that $C \subseteq U$ and $y \in V$). The proof is taken from \href{https://math.stackexchange.com/questions/4789531/when-does-the-borel-sigma-algebra-of-compact-convergence-coincide-with-the-pr}{this stackexchange question}.

We denote by $\mathcal{T}_1$ and $\mathcal{T}_2$ the two sigma-algebras.

First of all, the evaluation maps are continuous for the compact-open topology, and therefore measurable for $\mathcal{T}_1$. We deduce that $\mathcal{T}_2 \subseteq \mathcal{T}_1$.

Let us turn to the converse direction. For any $A \subseteq X$ and $B \subseteq Y$, denote
$$M(A, B) := {f \in C(X, Y), f(A) \subset B}.$$
The compact-open topology is generated by the sets of the form $M(K, U)$, for $K$ compact and $U$ open. Because $X$ and $Y$ are second-countable and $X$ is locally compact, $C(X, Y)$ is second-countable. Therefore it is enough to show that for any $K$ compact and $X$ open, $M(K, U) \in \mathcal{T}_2$. We now fix $K$ and $U$.

Consider $(V_n)_{n \in \mathbb{N}}$ a countable family of subsets of $Y$ which generates the topology (it exists by second-countability assumption). Clearly we have that
$$\bigcup_{\overline{V_n} \subseteq U} V_n \subseteq U.$$
Conversely, consider $y \in U$. Because $Y$ is regular, the set of $\overline{V_n}$ such that $y \in V_n$ is a basis of neighbourhoods of $y$. Indeed, consider $A$ an open neighbourhood of $y$. Then $A^c$ is a closed set which does not contain $y$. Therefore there exist disjoint open sets $B$ and $C$ such that $y \in B$ and $A^c \subseteq C$. There exists $n \in \mathbb{N}$ such that $x \in V_n \subseteq B$. Therefore $V_n \subseteq C^c$, so $\overline{V_n} \subseteq C^c \subseteq A$, and we are done. This proves that we have in fact
$$U = \bigcup_{\overline{V_n} \subseteq U} V_n = \bigcup_{\overline{V_n} \subseteq U} \overline{V_n}.$$
Consider now $f \in M(K, U)$. We then have
$$f(K) \subseteq \bigcup_{\overline{V_n} \subseteq U} V_n.$$
Because $K$ is compact and $f$ is continuous, $f(K)$ is compact. But each $V_n$ is open, so there exists a finite set $\hat{f}$ of natural integers such that for any $n \in \hat{f}$, $\overline{V_n} \subseteq U$, and
$$f(K) \subseteq \bigcup_{n \in \hat{f}} V_n \subseteq \bigcup_{n \in \hat{f}} \overline{V_n}.$$
Conversely, if there exists a finite set $\hat{f}$ such that for any $n \in \hat{f}$, $\overline{V_n} \subseteq U$, and
$$f(K) \subseteq \bigcup_{n \in \hat{f}} \overline{V_n},$$
then $f(K) \subseteq U$. We can therefore write that
$$M(K, U) = \bigcup_{\substack{I \subseteq \mathbb{N} \\ I \text{ finite} \\ \forall n \in I, \overline{V_n} \subseteq U}} \bigcup_{i \in I} \overline{V_i}.$$
This is a countable union, so we just have to prove that each term is measurable.

We now fix a finite set $I$ as in the union above. Because $X$ is second countable, there exists $Q$ a countable dense subset of $K$. For any $f$ continuous, because $Q$ is dense and $\bigcup_{i \in I} \overline{V_i}$ is closed, we have
$$f(K) \subset \bigcup_{i \in I} \overline{V_i} \Longleftrightarrow f(Q) \subset \bigcup_{i \in I} \overline{V_i}.$$
In other words, $M\left(K, \bigcup_{i \in I} \overline{V_i}\right) = M\left(Q, \bigcup_{i \in I} \overline{V_i}\right)$. We can write
$$M\left(Q, \bigcup_{i \in I} \overline{V_i}\right) = \bigcap_{q \in Q} (f \mapsto f q)^{-1} \bigcup_{i \in I} \overline{V_i},$$
and it is enough to show that each term in the intersection is measurable. Because $\bigcup_{i \in I} \overline{V_i}$ is closed, it is measurable. It therefore remains to prove that $f \mapsto f q$ is measurable for all $q \in Q$, but this is obvious from the definition of $\mathcal{T}_2$. This concludes the proof.
\end{proof}


\begin{definition}\label{def:MeasurableEquiv.continuousMap}
  \uses{thm:ContinuousMap.borel_eq_iSup_comap_eval}
  \leanok
  \lean{ProbabilityTheory.MeasurableEquiv.continuousMap}
The identity is a measurable equivalence between the continuous functions of $\mathbb{R}^{\mathbb{R}_+}$ with the subset sigma-algebra obtained from the product sigma-algebra, and $C(\mathbb{R}_+, \mathbb{R})$ with the Borel sigma-algebra coming from the compact-open topology.

Mathematically this says nothing more than the equality of sigma-algebras of Theorem~\ref{thm:ContinuousMap.borel_eq_iSup_comap_eval} but in Lean we have two different types so we need an equivalence.
\end{definition}


\begin{definition}[Wiener measure]\label{def:wienerMeasure}
  \uses{def:MeasurableEquiv.continuousMap, def:wienerMeasureAux}
  \leanok
  \lean{ProbabilityTheory.wienerMeasure}
The Wiener measure on $C(\mathbb{R}_+, \mathbb{R})$ with the Borel sigma-algebra is the map of the auxiliary Wiener measure by the measurable equivalence of definition~\ref{def:MeasurableEquiv.continuousMap}.
\end{definition}


TODO: add the main properties of the Brownian motion and the Wiener measure.
We need to be able to tell that we have built the correct objects.
